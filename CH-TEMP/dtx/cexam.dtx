% \iffalse meta-comment ^^M 此行的\iffalse 和 \end{document}后的那个\fi配对,用以在获得说明文档时越过.ins 文件
% !TeX program  = XeLaTeX
% !TeX encoding = UTF-8%
%
% <+title+>.dtx
%
%<*internal>
\iffalse
%</internal>
%
%<*readme>
% 
%</readme>
%
%<*internal>
\fi
%</internal>
%
%<*internal>
\begingroup
  \def\temp{LaTeX2e}
\expandafter\endgroup\ifx\temp\fmtname\else 
\csname fi\endcsname
%</internal>
%
%<*install>
\input ctxdocstrip
\askforoverwritefalse
\keepsilent
%^^M 设置宏包前言
\declarepreamble\cexampreamble
    Copyright (C) 2017--2020
    <+author+> and any individual authors listed elsewhere in this file.
    ----------------------------------------------------------------------
    This work may be distributed and/or modified under the
    conditions of the LaTeX Project Public License, either
    version 1.3c of this license or (at your option) any later
    version. This version of this license is in
       http://www.latex-project.org/lppl/lppl-1-3c.txt
    and the latest version of this license is in
       http://www.latex-project.org/lppl.txt
    and version 1.3 or later is part of all distributions of
    LaTeX version 2005/12/01 or later.
    This work has the LPPL maintenance status `maintained'.
    The Current Maintainers of this work is <+author+>.
------------------------------------------------------------------------------
\endpreamble
%
%^^M 设置 install.sh脚本前言
\declarepreamble\shpreamble 
\endpreamble
%
%^^M 设置所有宏包后言
\postamble
%
\endpostamble
%
% ^^M 设置生成文件放置目录
\immediate\write18{if [ ! -e INSTALL ]; then mkdir INSTALL ; fi}
\BaseDirectory{./}
\DeclareDir{CexamDir}{INSTALL}
\usedir{CexamDir}
%
%^^M 生成文件设置
\generate  
{
   \usepreamble\cexampreamble
    \file{\jobname.sty}	            {\from{\jobname.dtx}{package}}
%</install>
%<*internal>
    \file{\jobname.ins}             {\from{\jobname.dtx}{install}}
%</internal>
%<*install>
    \nopreamble\nopostamble
    \file{README.md}                {\from{\jobname.dtx}{readme}}
}
%
\catcode32=12\space
%
\Msg{***************************************************************}
\Msg{                                                               }
\Msg{ To finish the installation you have to move the following     }
\Msg{ file into proper directories searched by TeX:                 }
\Msg{                                                               }
\Msg{       \jobname.sty                                                        }
\Msg{                                                               }
\Msg{ The recommended directory is TDS:                             }
\Msg{                                                               }
\Msg{/usr/local/texlive/texmf-local/tex/latex/local/\jobname.sty       }
\Msg{                                                               }
\Msg{ For convinencely you can  chmod +x  install.sh then run it.   }
\Msg{								    }
\Msg{ To produce the documentation run the file cexam.dtx	    }
\Msg{ through XeLaTeX.						    }
\Msg{								    }
\Msg{ Happy TeXing!						    }
\Msg{								    }
\Msg{***************************************************************}
%
\endbatchfile 
%</install>
%<*internal>
\fi  
%</internal>
%
%<package|ctrlwarning|colornote>\NeedsTeXFormat{LaTeX2e}
%<package|ctrlwarning|colornote>\RequirePackage{expl3,l3keys2e,xparse,l3draw}
%<colornote>\RequirePackage{tcolorbox}
%<package>\RequirePackage{pifont}
%<package>\GetIdInfo$Id: cexam.dtx v3.4.0(alpha)  2022-04-02 ZhenhuaFeng  <fengzhenhua@outlook.com> $ {For Chinese middle school examination}
%<ctrlwarning>\GetIdInfo$Id: ctrlwarning.sty  v1.1  2021-02-05  ZhenhuaFeng  <fengzhenhua@outlook.com> $ {countrl fontwarning}
%<colornote>\GetIdInfo$Id: colornote.sty  v1.1  2021-02-05  ZhenhuaFeng  <fengzhenhua@outlook.com> $ {clone the note environment of hexo Next theme}
%<package|ctrlwarning|colornote>\ProvidesExplPackage{\ExplFileName}{\ExplFileDate}{\ExplFileVersion}{\ExplFileDescription}
%
%<*driver>
%\PassOptionsToPackage{AutoFakeSlant = true , AutoFakeBold = true}{xeCJK} ^^M 开启宋体伪粗体,但是本文件中不打算开启,在此处仅作为拓展
\documentclass{l3doc}
\usepackage[fontset=windows,UTF8, punct=kaiming, heading, linespread=1.2, sub3section]{ctex}
\usepackage[toc]{multitoc}
\usepackage{geometry}
\usepackage{tabularx}
\usepackage{makecell}
\usepackage{threeparttable}
\usepackage{siunitx}
\usepackage{unicode-math}
\usepackage{xcolor}
\usepackage{caption}
\usepackage{fancyvrb-ex}
\usepackage{zref-base}
\usepackage{hologo}
\usepackage[user=teacher,option=random,sourcecolor=blue]{cexam} ^^M 引入宏包,用以生成说明文档,展示排版效果
\usepackage{tikz}
\usepackage{amsthm} ^^M 此处宏包引入为了展示证明题的输入,当引入此宏包时加入证明结束符号,不引入则不加入证明结束符号
\ctexset{
  abstractname   = 简介,
  indexname      = 代码索引,
  section/format = \Large\bfseries\raggedright,
  section/name   = {第,节},
}
\def\glossaryname{版本历史}
\GlossaryPrologue{\section{\glossaryname}}
\IndexPrologue{%
  \section{\indexname}
  \textit{意大利体的数字表示描述对应索引项的页码;
  带下划线的数字表示定义对应索引项的代码行号;
  罗马字体的数字表示使用对应索引项的代码行号。}}
\geometry{includemp, hmargin={0mm,15mm}, vmargin={25mm,15mm}, footskip=7mm}
\hypersetup{pdfstartview=FitH, bookmarksdepth=subparagraph}
\setcounter{secnumdepth}{4}
\setcounter{tocdepth}{2}
\EnableCrossrefs
\CodelineIndex
\RecordChanges
%\OnlyDescription
\begin{document}
  \DocInput{\jobname.dtx}
  % \IndexLayout
  \newpage
  \PrintChanges
  \newpage
  \PrintIndex
\end{document}
%</driver>
%
% \fi ^^M 此\fi 和第1行的\iffalse 配对
% 
% \changes{v1.0.0}{2019/04/09}{开始使用 \hologo{LaTeX3}{}构建宏包}
%
% \CheckSum{0}
%
% \CharacterTable
%  {Upper-case    \A\B\C\D\E\F\G\H\I\J\K\L\M\N\O\P\Q\R\S\T\U\V\W\X\Y\Z
%   Lower-case    \a\b\c\d\e\f\g\h\i\j\k\l\m\n\o\p\q\r\s\t\u\v\w\x\y\z
%   Digits        \0\1\2\3\4\5\6\7\8\9
%   Exclamation   \!     Double quote  \"     Hash (number) \#
%   Dollar        \$     Percent       \%     Ampersand     \&
%   Acute accent  \'     Left paren    \(     Right paren   \)
%   Asterisk      \*     Plus          \+     Comma         \,
%   Minus         \-     Point         \.     Solidus       \/
%   Colon         \:     Semicolon     \;     Less than     \<
%   Equals        \=     Greater than  \>     Question mark \?
%   Commercial at \@     Left bracket  \[     Backslash     \\
%   Right bracket \]     Circumflex    \^     Underscore    \_
%   Grave accent  \`     Left brace    \{     Vertical bar  \|
%   Right brace   \}     Tilde         \~}
%
% \title{中文试题排版 cexam 宏包手册}
% \author{冯振华}
% \date{\ExplFileDate\qquad\ExplFileVersion\thanks{fengzhenhua@outlook.com}}
% \maketitle
%
%
% \begin{abstract}
%
% \end{abstract}
%
%  \tableofcontents
%
% \clearpage
% \setlength{\parskip}{0.8ex}
%
% \begin{documentation}
%
% \end{documentation}
%
% \section{介绍}
%
%
% \section{宏包的安装}
% 
% 由于宏包中的\pkg{解析}和\pkg{答案}是针对中文题型设计的,所以需要使用\pkg{xetex}\footnote{\pkg{xetex}是支持中文的,同时\pkg{xelatex}执行时程序名为\pkg{latex2e},而\pkg{xetex}与之不同,于是实现了二合一的文件。}和\pkg{xelatex}编译\pkg{cexam.dtx}。
% \begin{enumerate}
%     \item 生成 \pkg{cexam.ins} 和  \pkg{cexam.sty}, 执行命令
%         \begin{verbatim}
%         $ xetex --shell-escape cexam.dtx
%         \end{verbatim}
%     \item 生成说明文档 \pkg{cexam.pdf}, 执行命令
%         \begin{verbatim}
%         $ xelatex cexam.dtx
%         \end{verbatim}
% \end{enumerate}
% 
% 
% 考虑到每年texlive都会有一个更新,但是此宏包尚未计划进入texlive,所以不把宏包安装到对应年份目录下,而按装到默认的路径下,此宏包和说明文档安装位置分别为
% 
% \begin{verbatim}
% # cp cexam.sty /usr/local/texlive/texmf-local/tex/latex/local/cexam.sty
% # cp cexam.pdf /usr/local/texlive/texmf-local/doc/local/cexam.pdf
% # texhash 更新包(类)数据库
% \end{verbatim}
% 
% 将文件复制到对应文件夹后,由于使用的是 TexLive 所以还需要执行一下更新命令,让系统正确识别新安装的宏包和说明档,这样就可以使用 \pkg{texdoc} \pkg{cexam}来查找说明档。
% 
%<*install.sh> 
% \begin{macro}[added=2020/12/29]{install.sh}
% \changes{v3.3.3}{2020/12/29}{新增安装脚本}
% \changes{v3.4.0}{2022/04/02}{设置texlive变量}
% 为了提高效率,设置了安装脚本。
%    \begin{macrocode}
#!/bin/bash
# 2023年12月03日星期日多云北京市
echo "version: 1.2"
echo "Author: Feng Zhenhua(冯振华)"
printf "Date: "
date
%    \end{macrocode}
% 检测系统版本
%    \begin{macrocode}
printf "System Information:"
uname -a
%    \end{macrocode}
% 定义安装路径
%    \begin{macrocode}
LaTeX_STY="/usr/share/texmf-dist/tex/latex/cexam"
LaTeX_DOC="/usr/share/texmf-dist/doc/latex/cexam"
%    \end{macrocode}
% 发出执行命令
%    \begin{macrocode}
if [ ! -d $LaTeX_STY ]; then
    sudo mkdir $LaTeX_STY
fi
if [ ! -d $LaTeX_DOC ]; then
    sudo mkdir $LaTeX_DOC
fi
echo  "cexam.sty , cexam.pdf, colornote.sty and ctrlwarning.sty is installing... ..."
if [ -f ./cexam.sty ]; then
    sudo cp ./cexam.sty ${LaTeX_STY}/cexam.sty
else
    echo  "I can't find the file cexam.sty in the directory ./"
fi
if [ -f ./colornote.sty ]; then
    sudo cp ./colornote.sty ${LaTeX_STY}/colornote.sty
else
    echo  "I can't find the file cexam.sty in the directory ./"
fi
if [ -f ../cexam.pdf ]; then
    sudo cp ../cexam.pdf ${LaTeX_DOC}/cexam.pdf
else
    echo  "I can't find the file cexam.pdf in the directory ../"
fi
if [ -f ./ctrlwarning.sty ];then
    sudo cp ./ctrlwarning.sty  ${LaTeX_STY}/ctrlwarning.sty
else
    echo  "I can't find the file ctrlwarning.sty in the directory ./"
fi
sudo texhash
echo  "macro package: cexam.sty , colornote.sty and ctrlwarning.sty had been installed."
echo  "document: cexam.pdf had been installed."
%    \end{macrocode}
% \end{macro}
%</install.sh> 
% 
% \section{宏包选项}
% 
% 
% \begin{function}[added=2019-09-19]{cexam / option}
%  宏包根据所编写书籍的使用者设置了一个选项\pkg{user},当设置其为 \pkg{student}时将生成答案和题目分离,使用\cs{makeanswer} 在书籍的最后面生成答案.如果不指明\pkg{user}则默认为\pkg{teacher}.
% \begin{syntax}
% \cs{usepackage}\oarg{user=student}\Arg{cexam}
% \cs{usepackage}\oarg{user=teacher}\Arg{cexam}
% \cs{usepackage}\Arg{cexam}
% \end{syntax}
% \end{function}
% 
% \section{各题型输入格式}
% 
% 如果在所写的题型中不希望给图片编号,则在题号前加入*号(不加*号,则表示默认为图片编号,以编号取代图片的位置).各环境以[exp]标志是否为例题环境,如果是例题环境则题号
% 前加字``例'' ,同时只缩进这一个字符的宽度.一般而言我们在题干中输入图片时都是一幅,但是也不排除会有多幅图的可能,这时我们给出一个方法,不是直接用 | \includegraphics| 来录入,而是使用专门的分隔符号 <BeginPicture> 和<EndPicture>来分隔图片,二者内的所有部分将作为整体视为一个图片排版。如果题目中出现多个表格并排时,以<BeginTabular> 和 <EndTabular> 来分隔表格,二者内的所有部分将作为一个表格排版。
% 
% 
% \subsection{选择题环境choices}
% 
%  
% \begin{function}[added=2019-09-22]{choices , xuanze}
% choices 环境(和xuanze 环境相同,只是名称不同而己).如果不加选项[exp],则题目以正常格式排版,如果加上这个选项则以例题来排版,答案不输出到学生模式.在编写程序中考虑到了这一点,这个选项可以是任何作者认为可行的符号,只要给出了选项,则以例题排版. 
% 
% \end{function}
% 
% 在v3.3.4版中,我定义了新的输入方式,其输入结构更加可靠,同时也能提供选项随机生成功能。这一功能考虑的是在给学生的考试中,如果第一次考试则不必开启,但是过一段时间检测学生掌握情况的时候这些选择题就可以不变换题目,只需打开option=random 选项,就会随机生成一份选项不同的试题,这可以充分检测学生是否真正掌握了对应知识点,同时也为教师节省了大量的时间。同时考虑到它可以用加“*”的方式来随机排列选项,在v3.3.5 版中对学生模式答案输入也提供了随机支持,所以取消了原来直接的输入选项后自动获取选项的设定,同时使用\cs{choice}也可以避免个别题目中含有 \pkg{A.}等字符造成的麻烦。其使用方法如下
% 
% \begin{verbatim} 
% \begin{choices}[exp]
%  1.选择题题干,如果插入图片,则图片应当如\includegraphics{picture}所示.
%    从下面四个选项中选出正确的选项
%   \choice[A] 错误的选项
%   \choice[B] 错误的选项
%   \choice*[C] 正确的选项
%   \choice*[D] 正确的选项
% 
%  a.*
% 
%  e.关于选择题正确答案的解析,如果分析到正确选项为\refc{}和\refd{},
%    那么为了配合随机模式,请使用这里列出的选项引用方式.
% 
%  ee.如果解析中有多幅图片,则需要按图片分开来写解析,这一部分是补充,所以没有解析标志.
%
%  2. \source[4]{2020}{陕西省商洛市模拟} 行驶中的汽车遇到红灯刹车后做匀减速直线运动直到停
%    止,等到绿灯亮时又重新启动开始做匀加速直线运动直到恢复原来的速度继续匀速行驶,则从刹
%    车到继续匀速行驶这段过程,位移随速度变化的关系图像描述正确的是
%    \choice[P] \includegraphics{1.png}
%
% \end{choices}
%
% \end{verbatim}
% 
% \subsection{填空题环境blanks}
% 
% \begin{function}[added=2019-09-22]{blanks,tiankong}
% blanks 环境(和tiankong 环境相同,只是名称不同而己).如果不加选项[exp],则题目以正常格式排版,如果加上这个选项则以例题来排版,答案不输出到学生模式.在编写程序中考虑到了这一点,这个选项可以是任何作者认为可行的符号,只要给出了选项,则以例题排版.在填空题中以\cs{blank}\Arg{答案} 来标出答案,程序会自动转换成可换行的下划线,同时自动生成答案.在答案输入时以星号*代答案就可以获得正确的答案.
% \end{function}
% 
% \begin{verbatim} 
% \begin{blanks}[exp]
%  1.填空题题干,如果插入图片,则图片应当如\includegraphics{picture}所示.\blank{答案一}和
%    \blank{答案二}是填空题中需要留出的空白.
% 
%  a.*
% 
%  e.关于填空题正确答案的解析.
% 
%  ee.如果解析中有多幅图片,则需要按图片分开来写解析,这一部分是补充,所以没有解析标志.
% 
% \end{blanks}
% \end{verbatim}
% 
% \subsection{判断题环境judgements}
% 
% \begin{function}[added=2019-09-22]{judgements,panduan}
% judgements 环境(和panduan 环境相同,只是名称不同而己).如果不加选项[exp],则题目以正常格式排版,如果加上这个选项则以例题来排版,答案不输出到学生模式.在编写程序中考虑到了这一点,这个选项可以是任何作者认为可行的符号,只要给出了选项,则以例题排版.答案应当以对应的文字给出,比如:对,错等.
% \end{function}
% 
% \begin{verbatim} 
% \begin{judgements}[exp]
%  1.判断题题干,如果插入图片,则图片应当如\includegraphics{picture}所示.此问题是正确还是错误
% 
%  a.正确
% 
%  e.关于判断题正确答案的解析.
% 
%  ee.如果解析中有多幅图片,则需要按图片分开来写解析,这一部分是补充,所以没有解析标志.
% 
% \end{judgements}
% \end{verbatim}
% 
% \subsection{计算题环境calculations}
% 
% \begin{function}[added=2019-09-22]{calculations,jisuan}
% calculations 环境(和jisuan 环境相同,只是名称不同而己).如果不加选项[exp],则题目以正常格式排版,如果加上这个选项则以例题来排版,答案不输出到学生模式.在编写程序中考虑到了这一点,这个选项可以是任何作者认为可行的符号,只要给出了选项,则以例题排版.答案应当以对应的文字给出即可.考虑到有的问题有多个小问,则类比列表环境命令\cs{item}定义了计算题的各小问命令\cs{qitem},这一命令中的$q$ 指的是 \pkg{question}.
%
% 在v3.3.6版本中,加入了可选的小问参数,此参数用以来标记第几个小问,其可以定义对应的小问引用,所以在编写解析时可以直接以对应的题号进引用所解析的小问号。此处小问号可以自动修正,在作者编写时不需要修改可选号,则其引用对应值也会自动修正。所以源文件作出如下修改。
% 
% \end{function}
% 
% \begin{verbatim} 
% \begin{calculations}[exp]
%  1.计算题题干,如果插入图片,则图片应当如\includegraphics{picture}所示.请求解以下各问题
%  \qitem[1] 第一问的内容
%  \qitem[2] 第二问的内容
%  \qitem[3] ...
% 
%  a.计算题的答案
% 
%  e.关于计算题正确答案的解析,其中\refitem[2]是第二问的内容,\refitem[3]是第三问的内容.
% 
%  ee.如果解析中有多幅图片,则需要按图片分开来写解析,这一部分是补充,所以没有解析标志.
% 
%  11.\source[3]{2021}{德州一模}从斜面上某一位置每隔0.1s释放一颗小球,在连续释放几颗后
%  ,对斜面上正在运动着的小球拍下部分照片,如\includegraphics[scale=1]{./9.png}所示。现
%   测得$x_{AB}=15cm$,$x_{BC}=20cm$,已知小球在斜面上做匀加速直线运动,且加速度大小相同。
%   \qitem[1] 求小球的加速度。
%   \qitem[2] 求拍摄时B球的速度。
%   \qitem[4] A球上面正在运动着的小球共有几颗?
%   \qitem[3] D、C两球相距多远?
%
%   e.所以第 \refitem[2] 小问的解析为:求拍摄时B球的速度。
%
%
% \end{calculations}
% \end{verbatim}
% 
% \subsection{证明题环境proofs}
% 
% \begin{function}[added=2020-07-24]{proofs,zhengming}
% proofs环境(和zhengming 环境相同,只是名称不同而己).如果不加选项[exp],则题目以正常格式排版,如果加上这个选项则以例题来排版,答案不输出到学生模式.在编写程序中考虑到了这一点,这个选项可以是任何作者认为可行的符号,只要给出了选项,则以例题排版.答案应当以对应的文字给出即可.考虑到有的作者有可能引入\pkg{amsthm}宏包来输入证明题,所以这种情况下需要考虑到\pkg{amsthm}的格式中包含结束标志,同时又需要符合\pkg{cexam}的本身设定,于是此命令的设置兼容了\pkg{amsthm},当引入此宏包时自动追加上结束标志,如果不引入此宏包则统一为不加结束标志。
%
% 在v3.3.6版本中,加入了可选的小问参数,此参数用以来标记第几个小问,其可以定义对应的小问引用,所以在编写解析时可以直接以对应的题号进引用所解析的小问号。此处小问号可以自动修正,在作者编写时不需要修改可选号,则其引用对应值也会自动修正。所以源文件作出如下修改。
%
% \begin{verbatim} 
% \begin{proofs}[exp]
%  1.证明题题干,如果插入图片,则图片应当如\includegraphics{picture}所示.请求解以下各问题
%  \qitem[1] 第一问的内容
%  \qitem[2] 第二问的内容
%  \qitem[3]  ...
% 
%  p.证明过程,可以包含一幅图片
% 
%  pp.如果解析中有多幅图片,则需要按图片分开来写证明,这一部分是补充,所以没有证明标志.
% 
% \end{proofs}
% \end{verbatim}
% \end{function}
%
% \subsection{首字母下沉命令\cs{lettersink}}
% 
% \begin{function}[added=2019-09-22]{\lettersink}
%  这是一条附加命令,在写完程序后我发现实现这个效果不难,同时该命令支持数学公式的输出,可以实现含数学文本的首字母下沉.
% \end{function}
% 
% \begin{verbatim}
%  \lettersink[首字母高度][首字母与文本间距][首字母颜色]{首字母}
%  其余部分文字,注意这部分文字应当有足够的高度以实现与首字母的绕排.
%  同时默认的首字母高度为2cm,默认与文本间距5pt,默认首字母颜色黑色.
% 
% \end{verbatim}
% 
% \section{各题型排版效果展示}
% 
% \subsection{选择题展示}
%
% \begin{choices}[exp]
% 1.刻舟求剑的故事家喻户晓,“舟己行矣,而剑不行”这句话所选用的参考系是
% \choice[A]舟
% \choice*[B]地面
% \choice[C]舟上的人
% \choice[D]流动的水
% 
% a.*
% 
% e.此题考查参考系这一基本概念.舟相对于地行,而剑相对于地静止,所以这句话所选参考系应当为地面.
% 
% 1.刻舟求剑的故事家喻户晓,“舟己行矣,而剑不行”这句话所选用的参考系是
% \choice[A]舟
% \choice*[B]地面
% \choice[C]舟上的人
% \choice[D]流动的水
% 
% a.*
% 
% e.此题考查参考系这一基本概念.舟相对于地行,而剑相对于地静止,所以这句话所选参考系应当为地面.
% 2.某学校田径运动场 $400m$标准跑道如
% \begin{tikzpicture}
%  \draw (-1,0.6)--(1,0.6);
%  \draw (-1,-0.6)--(1,-0.6);
%  \draw (1,-0.6) arc (-90:90:0.6);
%  \draw (-1,0.6) arc (90:270:0.6);
%  \filldraw (-1,0.6) circle [radius=1pt] node [anchor=north] {\small B};
%  \filldraw (1,0.6) circle [radius=1pt]  node [anchor=north] {\small A};
% \end{tikzpicture}
% 所示,$100m$ 赛跑的起跑点在$A$点,终点在$B$点,$400m$ 赛跑的起跑点和终点都在$B$ 点.在校运动会中,甲、乙两位同学分别参加了$100m$ 、$400m$ 项目的比赛,关于甲、
% 乙两位同学运动的位移大小和路程的说法中正确的是
% \choice[A]甲、乙的位移大小相等
% \choice[B]甲、乙的路程相等
% \choice*[C]甲的位移比乙大
% \choice[D]甲的路比乙大
% 
% a.*
% 
% e.位移是指从初位置到末位置的有向线段,其大小就是有向线段的大小.而路指物体移动轨迹的长度,它是一个标量,所以此题不难考虑出来答案为\refc{}.
% 
% 
% 2.某学校田径运动场 $400m$标准跑道如
% \begin{tikzpicture}
%  \draw (-1,0.6)--(1,0.6);
%  \draw (-1,-0.6)--(1,-0.6);
%  \draw (1,-0.6) arc (-90:90:0.6);
%  \draw (-1,0.6) arc (90:270:0.6);
%  \filldraw (-1,0.6) circle [radius=1pt] node [anchor=north] {\small B};
%  \filldraw (1,0.6) circle [radius=1pt]  node [anchor=north] {\small A};
% \end{tikzpicture}
% 所示,$100m$ 赛跑的起跑点在$A$点,终点在$B$点,$400m$ 赛跑的起跑点和终点都在$B$ 点.在校运动会中,甲、乙两位同学分别参加了$100m$ 、$400m$ 项目的比赛,关于甲、
% 乙两位同学运动的位移大小和路程的说法中正确的是
% \choice[A]甲、乙的位移大小相等
% \choice[B]甲、乙的路程相等
% \choice*[C]甲的位移比乙大
% \choice[D]甲的路比乙大
% 
% a.*
% 
% e.位移是指从初位置到末位置的有向线段,其大小就是有向线段的大小.而路指物体移动轨迹的长度,它是一个标量,所以此题不难考虑出来答案为\refc{}.
%
%  3.下列关于质点的说法中,正确的是
% \choice[A]质点是一个理想化的模型,实际上并不存在,所以引入这个概念没有多大意义
% \choice*[B]体积很小的物体不一定能够看做质点
% \choice[C]凡轻小的物体,皆可看做质点
% \choice*[D]当物体的形状和大小对所研究的问题属于无关或者次要因素时,即可把物体看成质点
% 
%  a.*
% 
%  e.建立理想模型是物理中的重要的研究方法,对于复杂问题的研究有重大意义,\refa{}错误;一个物体能否看成质点不以轻重而论,\refc{}错误;物体能否看成质点取决于其大小和形状对所研究的问题是否属于无关或次要因素,若是就可以看成质点,\refd{}正确.
%
%  3.下列关于质点的说法中,正确的是
% \choice[A]质点是一个理想化的模型,实际上并不存在,所以引入这个概念没有多大意义
% \choice*[B]体积很小的物体不一定能够看做质点
% \choice[C]凡轻小的物体,皆可看做质点
% \choice*[D]当物体的形状和大小对所研究的问题属于无关或者次要因素时,即可把物体看成质点
% 
%  a.*
% 
%  e.建立理想模型是物理中的重要的研究方法,对于复杂问题的研究有重大意义,\refa{}错误;一个物体能否看成质点不以轻重而论,\refc{}错误;物体能否看成质点取决于其大小和形状对所研究的问题是否属于无关或次要因素,若是就可以看成质点,\refd{}正确.
%
%   2.\source[4]{2020}{陕西省商洛市模拟} 行驶中的汽车遇到红灯刹车后做匀减速直线运动直到停止,等到绿灯亮时又重新启动开始做匀加速直线运动直到恢复原来的速度继续匀速行驶,则从刹车到继续匀速行驶这段过程,位移随速度变化的关系图像描述正确的是
%   \choice[P] \includegraphics{1.png}
%   a.C
%
%   e.汽车在匀减速过程中由速度和位移的关系可知:$ v^2 - v_0^2 =2a_1x$,可得$ x=\frac{v^2-v_0^2}{ 2a_1}$,$ a_1$为负值,故$ x-v$图像应为开口向下的二次函数图像;汽车重新启动,速度由零开始增大时,$ v^2=2a_2(x-x_0)$,$ x_0 $是停止时的位移,可得$x=\frac{v^2}{2a_2}+x_0$,$a_2$为正值,故 $x-v$ 图像为开口向上的二次函数图像。故C正确,A、B、D错误。
%
% \end{choices}
%
% \subsection{填空题展示}
% 
% \begin{blanks}[exp]
%  1.打点计时器是记录做直线运动物体的\blank{位移}和\blank{时间}的仪器,电火花计时器是其中的一种,其工作电压是\blank{220v},电火花计时器靠电火花和墨粉打点,当交流电的频率为$50Hz$ 时,它每隔\blank{0.02}秒打一次点。
% 
% a.*
% 
% e.此题考察打点计时器的应用与操作,打点计时器采用打点的方式在纸带上留下点迹,通过测量点迹间的距离可以确定位移。同时使用的电流一定是交流电,它每隔一段时间打一次点,通常频率为$50Hz$ 的交流电,每秒打点50次,所以每两次的间隔为$0.02s$.
% 
% \end{blanks}
% 
% \begin{blanks}
%  1.用 $v-t$ 图像表示小车的运动情况时,以速度$v$ 为\blank{纵轴}、时间 $t$ 为\blank{横轴} 建立直角坐标系,用描点法画出小车的 $v-t$ 图象,图线的 \blank{斜率} 表示加速度的大小,如果 $v-t$ 图象是一条倾斜的直线,说明小车的速度是\blank{均匀变化}的。
% 
% a.*
% 
% e.此题考察 $v-t$ 图象的意义,通过 $v-t$ 图象识别加速度和判断物体运动特征。
% 
% \end{blanks}
% 
% \subsection{判断题展示}
% 
% \begin{judgements}[exp]
%  1.建立直线坐标系时,一定要规定运动方向为正方向
% 
%  a.错误 
% 
%  e.坐标系的建立具有任意性,可以选择任何一 个方向为正方向。但是通常在解决一个实际问题时会根据方便而选择坐标系的方向。
% 
% \end{judgements}
%
% \begin{judgements}
%  2.时间变化量一定为正值
% 
%  a.正确
% 
%  e.变化量指的是末时刻的物理量减去初时刻的物理量,所以时间的变化量一定为正的。
% 
%  3.物体的平均速度为零,则物体一定处于静止状态
% 
%  a.错误 
% 
%  e.当物体转一圈又回到原点时,物体的平均速度为零,但是它却不处于静止状态。
% 
% \end{judgements}
% 
% \subsection{计算题展示}
% 
% \begin{calculations}[exp]
% 
%  2.一物体做匀加速直线运动,通过一段位移$\Delta x$ 所用的时间为$t_1$ ,紧接着通过下一段位移$\Delta x$ 所用时间为$t_2$ ,求物体运动的加速度.
%
%  a.$\cfrac{2\Delta x(t_1-t_2)}{t_1t_2(t_1+t_2)}$
%
%  e.物体通过第一段位移中间时刻的瞬时速度为$v_1=\frac{\Delta x}{t_1}$ ,通过第二段位移中间时刻的瞬时速度为$v_2=\frac{\Delta x}{t_2}$ ,由$v_1$ 变到$v_2$ 所需的时间显然为 $\Delta t=\frac{t_1+t_2}{2}$ ,由加速度定义得
%  $$a=\cfrac{v_2-v_1}{\Delta t}=\cfrac{2\Delta x(t_1-t_2)}{t_1t_2(t_1+t_2)}$$
% 
% 
% \end{calculations}
% 
% \begin{calculations}
% 
%    11.\source[3]{2021}{德州一模}从斜面上某一位置每隔0.1s释放一颗小球,在连续释放几颗后,对斜面上正在运动着的小球拍下部分照片,如\includegraphics[scale=1]{./9.png}所示。现测得$x_{AB}=15cm$,$x_{BC}=20cm$,已知小球在斜面上做匀加速直线运动,且加速度大小相同。
%    \qitem[1] 求小球的加速度。
%    \qitem[2] 求拍摄时B球的速度。
%    \qitem[4] A球上面正在运动着的小球共有几颗?
%    \qitem[3] D、C两球相距多远?
%
%    e.所以第 \refitem[2] 小问的解析为:求拍摄时B球的速度。
% 
% 
% \end{calculations}
% 
% \subsection{证明题展示}
% 
% \begin{proofs}
%  1. 设$x_1,x_2,\dots,x_n\geqslant 0$,且$x_1+x_2+\dots+x_n=\frac{1}{2}$ , 求证:
%    \[
%       (1-x_1)(1-x_2)\dots(1-x_n)\geqslant \frac{1}{2}
%    \]
%
%    p.由伯努利不等式得
%    \[
%       (1-x_1)(1-x_2)\dots(1-x_n)
%       \geqslant
%       1-(x_1+x_2+\dots+x_n)
%       =1-\frac{1}{2}=\frac{1}{2}
%    \]
%
%   >pp.此题也是轮换不等式。由多元函数各偏微分为零可得,当$x_1=x_2=\dots=x_n$时,此多元函数取极值,即
%    $x_i=\frac{1}{2n}$ ,于是
%    \[
%       f_m= (1-\frac{1}{2n})^n
%    \]
%    显然当$n$ 增加时,$fm$增加,同时由特殊值不可以写出此极值为极小值。同时极小值的极小值为$n=1$时,即 
%    \[
%       (1-x_1)(1-x_2)\dots(1-x_n)
%       \geqslant
%       (1-\frac{1}{2n})^n
%       \geqslant
%       \frac{1}{2}
%    \]
% \end{proofs}
% 
% \subsection{首字母下沉展示}
% 
% \lettersink[2cm][2pt][magenta]
% 物理学的发展,推动了工业、农业和信息技术等方面的进步,引发了一次次的产业革命,改变了人类的生产和生活方式。技术的进步又为物理学的研究提供了更为强大的手段,并引发了人们对物理问题进行更深入的思考,从而反过来促进物理学的发展。
% 创立于17世纪的牛顿力学,被广泛地应用于工程技术,大大推动了社会的发展。18 ~ 19世纪,工程上对蒸汽机的改进需求,又迫使人们对热的问题进行深入研究,引发了热力学的巨大进步。
%  19 ~20世纪初,电磁学的发展,直接导致发电机和无线电通信的诞生,使电能被广泛利用。电走进了千家万户,世界被电灯点亮,电话和电报把各地的人们连接起来,人类从此进入了电气时代。
%
% \section{纯文本和数学文本分离}
% 
% 
% \begin{function}[added=2019-09-19]{\cexam_sep:n}
% 使用此程序来分离纯文本和数学文本,它可以自动探测输入数学文本的模式,支持标准的数学输入格式.分离后将获得三部分:\cs{sep_hd_tl} , \cs{sep_bd_tl}和\cs{sep_tl_tl}.其分别对应头部,数学部和尾部(用来继续生成新的头部和尾部).
% \begin{syntax}
% \cs{cexam_sep:n} \meta{text} \cs{scan_stop:}
% \end{syntax}
% \end{function}
% 
% 
% \section{获得指定宽度文本行数和高度}
% 
% \begin{function}[added=2019-09-19]{\get_par_row:nnn}
% 此程序用来获得文本行数,文本行数存储在所用的\Arg{hang}计数器中.
% 
% \begin{syntax}
% \cs{get_par_row:nnn} \Arg{hang}\Arg{text~width}\Arg{text}
% \end{syntax}
% \end{function}
% 
% \begin{function}[added=2019-09-19]{\get_par_ht:nnn}
% 此程序用来获得文本行数,文本高度存储在所用的\Arg{dim}长度中.
% 
% \begin{syntax}
% \cs{get_par_ht:nnn} \Arg{dim}\Arg{text~width}\Arg{text}
% \end{syntax}
% \end{function}
%
% 
% \begin{function}[added=2019-09-19]{\get_par_rowht:nnnn}
% 此程序用来获得文本行数,文本行数存储在所用的\Arg{hang}计数器中,文本高度存储在所用的\Arg{dim}长度中.
% 
% \begin{syntax}
% \cs{get_par_rowht:nnnn} \Arg{hang}\Arg{dim}\Arg{text~width}\Arg{text}
% \end{syntax}
% \end{function}
% 
% \section{段落形状生成}
% 
% 
% \begin{function}[added=2019-09-20]{\cexam_sha_add:n}
% 用来追加到段落形状中的缩进或者行宽.
% \begin{syntax}
% \cs{cexam_sha_add:n}\Arg{dim}
% \end{syntax}
% \end{function}
% 
% \begin{function}[added=2019-09-20]{\cexam_sha_mk:nnn}
% 此程序用来生成指定缩进和行宽的形状.
% \begin{syntax}
% \cs{cexam_sha_mk:nnn}\Arg{int}\Arg{leftindent}\Arg{linewidth}
% \end{syntax}
% \end{function}
%
% 
% \begin{function}[added=2019-09-20]{\cexam_shad_set:n}
% 设定段落的总行数
% \begin{syntax}
% \cs{cexam_shad_set:n}\Arg{int}
% \end{syntax}
% \end{function}
% 
% 
% \begin{function}[added=2019-09-20]{\cexam_lwr_set:nnnn}
% 设置图片位置及左右缩进.
% 
% \begin{syntax}
% \cs{cexam_lwr_set:nnnn}\Arg{l~or~r}\Arg{picwd}\Arg{lindent}\Arg{rindent}
% \end{syntax}
% \end{function}
% 
% \section{图片格式化}
% 
% 
% \begin{function}[added=2019-09-20]{\cexam_fmt_pic:nnnn}
% 格式化图片命令.
% \begin{syntax}
% \cs{cexam_fmt_pic:nnnn}\Arg{l~or~r}\Arg{pic}\Arg{lindent}\Arg{rindent}
% \end{syntax}
% \end{function}
% 
% \section{基本排版程序}
% 
% 
% \begin{function}[added=2019-09-20]{\cexam_type_i:nnnnnnn}
% 此程序用以排版文本以行宽减图宽排版时高度大于图高的情况,其第一级和第二级缩进可以单独设置,由于没有第三级缩进所以此不能用于排版选择题含长选项的情况,因为长选项是第三级部分,其是需要缩进的.但是当选项不是长选项时,其不需要缩进,则要以此程序排版.
% \begin{syntax}
% \cs{cexam_type_i:nnnnnnn}
% \Arg{l~or~r}\Arg{pic}
% \Arg{lind}\Arg{rind}
% \Arg{sublind}\Arg{subrind}
% \Arg{text}
% \end{syntax}
% \end{function}
% 
% 
% \begin{function}[added=2019-09-21]{\cexam_type_ii:nnnnnnnnn}
% 此程序用以排版文本以行宽减图宽排版时高度大于图高的情况,其第一级,第二级和第三级缩进可以单独设置,于排版选择题含长选项的情况,因为长选项是第三级部分,其是需要缩进的.其也可以排版选择题短选项,即第三级缩进同第二级缩进相同的情况,但是这样会执行更多的代码,对于短选项部分使用\cs{cexam_type_i:nnnnnnn} 排版更加合理.
% \begin{syntax}
% \cs{cexam_type_ii:nnnnnnnnn}
%  \Arg{l~or~r}\Arg{pic}
%  \Arg{lind}\Arg{rind}
%  \Arg{sublind}\Arg{subrind}
%  \Arg{subsublind}\Arg{subsubrind}
%  \Arg{text}
% \end{syntax}
% \end{function}
% 
% 
% \begin{function}[added=2019-09-21]{\cexam_type_iii:nnnnnnn}
% 此程序用以排版题干,图片居中,选项依次排版的情况.
% \begin{syntax}
% \cs{cexam_type_iii:nnnnnnn}
%  \Arg{l~or~r}\Arg{pic}
%  \Arg{lind}\Arg{rind}
%  \Arg{sublind}\Arg{subrind}
%  \Arg{text}
% \end{syntax}
% \end{function}
% 
% 
% \begin{function}[added=2019-09-21]{\cexam_type_iv:nnnnnnnn}
% 此程序用以排版含图模式,其含有二部分文本,第一部分为题干第二部分为选项(选择题长选项)且这二级缩进可以单独设置.
% \begin{syntax}
% \cs{cexam_type_iv:nnnnnnnn}
%  \Arg{l~or~r}\Arg{pic}
%  \Arg{lind}\Arg{rind}
%  \Arg{sublind}\Arg{subrind}
%  \Arg{text}\Arg{subtext}
% \end{syntax}
% \end{function}
% 
% 
% \begin{function}[added=2019-09-21]{\cexam_type_v:nnnnn}
% 此程序用以排版无图模式,包含二级缩进,这二级的左右缩进可以单独设置.
% \begin{syntax}
% \cs{cexam_type_v:nnnnn}
%  \Arg{lind}\Arg{rind}
%  \Arg{sublind}\Arg{subrind}
%  \Arg{text}
% \end{syntax}
% \end{function}
% 
% \StopEventually{}
% 
%\begin{implementation}
% \clearpage
% \section{cexam.sty代码实现}
%
% \subsection{缩写列表}
% 
% 由于编写过程中需要对函数命名,如果为了清析则可以使用全称来命名,但是这样做会导致程序的名字过长,输入不便同时会影响逻辑结构的表达清析.但是用过短的简写来命名,对于维护来说不是很方便,这也是我在此处列出缩写列表的目的所在,两者兼顾,同时所生成的宏包还不容易被破译.
%
%  \begin{center}
%  \begin{tabular}{|*{12}{c|}}
%  \hline
%  简写&英文&中文&&简写&英文&中文&&简写&英文&中文\\
%  \hline
%  by& body&主体&& mk& make& 生成&& rec&rectangle&矩形\\
%  \hline
%  hd&head&头部&& sha&shape&形状&&sep&separate&分离\\
%  \hline
%  tl&tail&尾部&&txt &text &文本 &&mat &math &数学 \\
%  \hline
%  sub&subtraction&减去&&ps &parshape&形状&&equ &equation &公式 \\
%  \hline
%  \end{tabular}
%  \end{center}
%
% \subsection{布尔值设置}
%    \begin{macrocode}
%<*package>
%    \end{macrocode}
%
%    \begin{macrocode}
%<@@=cexam>
%    \end{macrocode}
%  \pkg{expl3}和\pkg{l3keys2e}检测,设置此检测的目的是:随着\pkg{cexam}的开发,将来有可能用到这两个宏包的新增功能,而旧版有可能不包含新的功能,所以要检测一下版本日期,确保存在需要的新功能,为了不依赖于\pkg{ctex}这里直接借用其检测代码。
%
% \begin{macro}[added=2020/05/01]{l3-too-old}
% \changes{v3.2.6}{2020/05/01}{新增版本检测}
%    \begin{macrocode}
\msg_new:nnnn {cexam}{l3-too-old}
{Support~package~#1~too~old.}
{
   Please~update~an~up~to~date~version~of~the~bundles\\\\
   'l3kernel'~and~'l3packages'\\\\
   using~your~TeX~package~manager~or~from~CTAN
}
\@ifpackagelater{expl3}{2019/03/05}{}
{\msg_error:nnn {cexam}{l3-too-old}{expl3}}
\@ifpackagelater{l3keys2e}{2015/12/20}{}
{\msg_error:nnn {cexam}{l3-too-old}{l3keys2e}}
%    \end{macrocode}
% \end{macro}
% 
%    \begin{variable}[added=2019-05-11]{\g_@@_sep_bd_bool}
%  这个布尔值在数学分离模式中标志数学模式是否文本串中有数学公式,字符串分离后尾部是否为空.
%    \begin{macrocode}
\bool_new:N \g_@@_sep_bd_bool
%    \end{macrocode}
%    \end{variable}
%
%    \begin{variable}[added=2019-05-11]{\g_@@_sep_tl_bool}
%  这个布尔值在数学分离模式中标志数学模式是否文本串中有数学公式,字符串分离后尾部是否为空.
%    \begin{macrocode}
\bool_new:N \g_@@_sep_tl_bool
%    \end{macrocode}
%    \end{variable}
%
%    \begin{variable}[added=2019-08-25]{\cexam_nopic_bool}
%  此布尔值用来判断图片与文字分离时,题干中是否存在图片(或表格).如果为真则无图片(或表格),如果为假,则有图片.
%    \begin{macrocode}
\bool_new:N \cexam_nopic_bool
%    \end{macrocode}
%    \end{variable}
%
%    \begin{variable}[added=2019-08-25]{\cexam_notab_bool}
%  此布尔值用来判断图片与文字分离时,题干中是否存在图片(或表格).如果为真则无图片(或表格),如果为假,则有图片.
%    \begin{macrocode}
\bool_new:N \cexam_notab_bool
%    \end{macrocode}
%    \end{variable}
%
%    \begin{variable}[added=2019-08-29]{\cexam_fmt_bool}
% \changes{v3.1.2}{2019/08/29}{增加图片格式化判断布尔值}
%  此布尔值用来判断图片是否需要格式化,即带上下标号.
%
%    \begin{macrocode}
\bool_new:N \cexam_fmt_bool
%    \end{macrocode}
%    \end{variable}
%
%    \begin{variable}[added=2019-08-29]{\cho_opt_maxed_bool}
% \changes{v3.1.2}{2019/08/29}{增加长选项判断布尔值}
%  此布尔值用来判断选择题选项是否是按长行依次排列.
%
%    \begin{macrocode}
\bool_new:N \cho_opt_maxed_bool 
%    \end{macrocode}
%    \end{variable}
%
%    \begin{variable}[added=2019-09-18]{\answer_student_bool}
% \changes{v3.1.6}{2019/09/18}{增加学生模式答案写出布尔值}
%  此布尔值用来判断是否是学生模式,当为学生模式时答案不在原题显示,而在书籍后面生成单独的答案.
%
%    \begin{macrocode}
\bool_new:N \answer_student_bool
%    \end{macrocode}
%    \end{variable}
% 
%
% \begin{variable}[added=2020-07-24]{\ctrl_end_bool}
%  此布尔值用来控制解析证明的结束符号是否显示。
%
%    \begin{macrocode}
\bool_new:N \ctrl_end_bool
%    \end{macrocode}
% \end{variable}
% 
% \begin{variable}[added=2021-02-05]{\cho_optrand_bool}
% \changes{v3.3.4}{2021/02/05}{新增命令}
%  此布尔值用来控制选择题选项是否开启随机排布选项模式。
%
%    \begin{macrocode}
\bool_new:N \cho_optrand_bool 
%    \end{macrocode}
% \end{variable}
% 
% \begin{variable}[added=2021-02-05]{\cho_optstar_bool}
% \changes{v3.3.4}{2021/02/05}{新增命令}
%  此布尔值用来控制选择题选项自动生成答案,且答案自动跟随选项生成。
%
%    \begin{macrocode}
\bool_new:N \cho_optstar_bool
%    \end{macrocode}
% \end{variable}
% 
% \begin{variable}[added=2021-02-05]{\choice_oldopt_bool}
% \changes{v3.3.4}{2021/02/05}{新增命令}
%  此布尔值为了兼容之前版本的选择题输入格式,实现不同的答案类型输出。
%
%    \begin{macrocode}
\bool_new:N \choice_oldopt_bool
%    \end{macrocode}
% \end{variable}
%
% \begin{variable}[added=2021-02-05]{\cexam_choice_bool,\cexam_blank_bool,\cexam_calculate_bool,\cexam_judgement_bool}
% \changes{v3.3.4}{2021/02/06}{新增命令}
% 标志各题型所处的模式
%
%    \begin{macrocode}
\bool_new:N \cexam_choice_bool
\bool_new:N \cexam_blank_bool
\bool_new:N \cexam_judgement_bool
\bool_new:N \cexam_calculate_bool 
%    \end{macrocode}
% \end{variable}
%
% \begin{variable}[added=2021-02-25]{\source_display_bool,\source_star_bool,\source_year_bool}
% \changes{v3.3.6}{2021/02/25}{新增命令,控制题源显示}
% \changes{v3.3.6}{2021/02/26}{新增命令,控制题源显示}
% 用来控制题源显示与否,及单独控制星级和年份
%
%    \begin{macrocode}
\bool_new:N \source_display_bool
\bool_new:N \source_star_bool
\bool_new:N \source_year_bool
%    \end{macrocode}
% \end{variable}
%
% \begin{variable}[added=2022-03-29]{\cexam_env_add_bool}
% \changes{v3.3.9}{2022/03/29}{新增命令,控制各题型环境判断}
% 在此版中更改为更加兼容的模式,方便统一改进各题型排版
%    \begin{macrocode}
\bool_new:N \cexam_env_add_bool
%    \end{macrocode}
% \end{variable}
% \subsection{盒子设置}
%
% \begin{variable}[added=2019-05-14]{\cexam_txtht_box}
% 此盒子用来在计算行数时获得对应文字的高度,其应用于测量高度时接收\cs{parbox}的预排版. 
%    \begin{macrocode}
\box_new:N \cexam_txtht_box
%    \end{macrocode}
% \end{variable}
%
% \begin{variable}[added=2019-08-14]{\cexam_picture_box}
% 此盒子用来存储图片,以获得图片的各种尺寸.
%    \begin{macrocode}
\box_new:N \cexam_picture_box 
%    \end{macrocode}
% \end{variable}
%
% \begin{variable}[added=2019-08-24]{\cho_option_box}
% \changes{v3.0.9}{2019/08/24}{新增选择题选项最大长度获取盒子}
% 此盒子用来存储选择题中的选项,以获得选项单行排版时的宽度.
%    \begin{macrocode}
\box_new:N \cho_option_box
%    \end{macrocode}
% \end{variable}
%
% \begin{variable}[added=2019-08-25]{\cexam_option_box}
% \changes{v3.1.0}{2019/08/25}{新增选项格式化盒子}
% 此盒子用来存储格式化的选项,用来的最终排版时生成对应的段落格式.
%    \begin{macrocode}
\box_new:N \cexam_option_box
%    \end{macrocode}
% \end{variable}
%
% \begin{variable}[added=2019-08-25]{\sep_temp_box}
% \changes{v3.1.0}{2019/08/25}{新增图片分离临时盒子}
% 此盒子用来在分离图片和文本时临时存储图片,以判定图片是否为空.
%    \begin{macrocode}
\box_new:N \sep_temp_box
%    \end{macrocode}
% \end{variable}
%
% \begin{variable}[added=2019-08-29]{\cho_optpic_box}
% \changes{v3.1.2}{2019/08/29}{新增判定选项排版格式盒子}
%
% 此盒子用来存储决定选项排版时,图片的各尺寸,为了防止与图片格式化时的付值影响图片格式化,所以此处单独设置一个盒子.
%    \begin{macrocode}
\box_new:N \cho_optpic_box
%    \end{macrocode}
% \end{variable}
%
% \begin{variable}[added=2019-08-29]{\cexam_number_box}
% \changes{v3.1.2}{2019/08/29}{新增题号格式尺寸获得盒子}
%
%    \begin{macrocode}
\box_new:N \cexam_number_box
%    \end{macrocode}
% \end{variable}
%
% \begin{variable}[added=2019-09-03]{\blank_wd_box}
% \changes{v3.1.3}{2019/09/03}{新增填空题空白长度测量盒子}
%
%    \begin{macrocode}
\box_new:N \blank_wd_box
%    \end{macrocode}
% \end{variable}
%
% \begin{variable}[added=2020-03-22]{\fmt_picture_box,\fmt_picture_vbox,\fmt_picture_hbox,\fmt_pic_vbox,\fmt_pic_hbox,\fmt_pic_r_hbox,\fmt_pic_r_vbox,\fmt_pic_t_vbox}
% \changes{v3.2.4}{2020/03/22}{新增前缀盒子}
%  
% 此盒子是在前缀设置命令\cs{cexam_fmt_pic:nnnn}中为了取代之前的\cs{parbox}命令而专门设置了,它们用来构建图片的放置位置。
%
%    \begin{macrocode}
\box_new:N \fmt_picture_box
\box_new:N \fmt_picture_vbox
\box_new:N \fmt_picture_hbox
\box_new:N \fmt_pic_vbox
\box_new:N \fmt_pic_hbox
\box_new:N \fmt_pic_r_hbox
\box_new:N \fmt_pic_r_vbox
\box_new:N \fmt_pic_t_vbox
%    \end{macrocode}
% \end{variable}
%
% \begin{variable}[added=2020-03-27]{\ind_hat_vbox ,\ind_hat_hbox ,\ind_hat_box}
% \changes{v3.2.5}{2020/03/27}{新增前缀盒子}
%  
% 此盒子是在前缀设置命令\cs{cexam_ind_hat:nnnn}中为了取代之前的\cs{parbox}命令而专门设置了,它们用来构建前缀。
%    \begin{macrocode}
\box_new:N \ind_hat_vbox
\box_new:N \ind_hat_hbox
\box_new:N \ind_hat_box
%    \end{macrocode}
% \end{variable}
%
% \subsection{长度设置}
%
% \begin{variable}[added=2019-07-31]{\rec_tempht_dim}
% 此长度变量用来在计算行数时,临时存储文本的高度.
%    \begin{macrocode}
\dim_new:N \rec_tempht_dim
%    \end{macrocode}
% \end{variable}
%
% \begin{variable}[added=2019-07-31]{\cexam_psrin_dim,\cexam_pslin_dim,\cexam_pswd_dim}
% 此三个变量用来在形状生成程序中存储右缩进,左缩进,行宽.
%    \begin{macrocode}
\dim_new:N \cexam_psrin_dim 
\dim_new:N \cexam_pslin_dim 
\dim_new:N \cexam_pswd_dim 
%    \end{macrocode}
% \end{variable}
%
% \begin{variable}[added=2019-08-14]{\cexam_picht_dim,\cexam_picwd_dim}
% 此三个变量用来处理图片时记录图片的高,宽.第三个长是为了进行三级排版时设置的.
%    \begin{macrocode}
\dim_new:N \cexam_picht_dim
\dim_new:N \cexam_picwd_dim
%    \end{macrocode}
% \end{variable}
%
% \begin{variable}[added=2019-08-24]{\cho_lmax_dim}
% \changes{v3.0.9}{2019/08/24}{选择题最大选项长度}
% 此长度用来存储选择题中四个选项的最大长度
%    \begin{macrocode}
\dim_new:N \cho_lmax_dim
\dim_set:Nn \cho_lmax_dim {0pt}
%    \end{macrocode}
% \end{variable}
%
% \begin{variable}[added=2019-10-10]{\cho_lmax_i_dim ,\cho_lmax_ii_dim}
% \changes{v3.1.9}{2019/10/10}{新增\cs{cho_lmax_i_dim}}
% \changes{v3.2.0}{2019/10/13}{新增\cs{cho_lmax_ii_dim}}
% \cs{cho_lmax_i_dim}来存储选择题中A选项和B选项中的最大宽度,
% \cs{cho_lmax_ii_dim}来存储选择题中C选项和D选项中的最大宽度
%
%    \begin{macrocode}
\dim_new:N \cho_lmax_i_dim
\dim_new:N \cho_lmax_ii_dim
\dim_set:Nn \cho_lmax_i_dim {0pt}
\dim_set:Nn \cho_lmax_ii_dim {0pt}
%    \end{macrocode}
% \end{variable}
%
% \begin{variable}[added=2019-08-25]{\cho_optwd_dim,\cho_optwd_i_dim}
% \changes{v3.1.0}{2019/08/25}{选择题选项的行宽}
% 第一个长度用来存储选择题四个选项排版时的行宽,默认值为\cs{linewidth}
% 第二个长度用来确定每个选项的排版宽度
%
%    \begin{macrocode}
\dim_new:N \cho_optwd_dim
\dim_new:N \cho_optwd_i_dim
%    \end{macrocode}
% \end{variable}
%
% \begin{variable}[added=2019-08-29]{\sep_HD_ht}
% \changes{v3.1.2}{2019/08/29}{新增长度}
% 此长度用来存储已经排过版的内容的高度,用以辅助生成文本高度和行数.
%    \begin{macrocode}
\dim_new:N \sep_HD_ht  
%    \end{macrocode}
% \end{variable}
%
% \begin{variable}[added=2019-08-29]{\cho_optpic_wd_dim,\cho_optpic_ht_dim,\cho_optpic_hti_dim}
% \changes{v3.1.2}{2019/08/29}{新增长度}
% 第一个长度用来存储选择题选项的宽度,第二个用来存储选项的高度,第三个用来存储判断高度
%    \begin{macrocode}
\dim_new:N \cho_optpic_ht_dim
\dim_new:N \cho_optpic_hti_dim
\dim_new:N \cho_optpic_wd_dim
%    \end{macrocode}
% \end{variable}
%
% \begin{variable}[added=2019-08-29]{\cexam_indent_dim,\cexam_indent_i_dim}
% \changes{v3.1.2}{2019/08/29}{新增长度}
% 第一个长度用来存储一级缩进,第二个用来存储二级缩进.
%    \begin{macrocode}
\dim_new:N \cexam_indent_dim
\dim_new:N \cexam_indent_i_dim
%    \end{macrocode}
% \end{variable}
%
% \begin{variable}[added=2019-09-01]{\cexam_pictxt_skip,\cexam_numtxt_skip}
% \changes{v3.1.2}{2019/09/01}{新增长度}
% 第一个长度用来存储图片与文本的间距,第二个用来存储题号与文本的间距.默认值都是5pt.
%    \begin{macrocode}
\dim_new:N \cexam_pictxt_skip 
\dim_set:Nn \cexam_pictxt_skip{5pt}
\dim_new:N \cexam_numtxt_skip
\dim_set:Nn \cexam_numtxt_skip{5pt}
%    \end{macrocode}
% \end{variable}
%
% \begin{variable}[added=2019-09-03]{\cexam_pic_linwd_dim}
% \changes{v3.1.3}{2019/09/03}{新增长度}
% 此长度为格式化图片时的行宽.
%
%    \begin{macrocode}
\dim_new:N \cexam_pic_linwd_dim
%    \end{macrocode}
% \end{variable}
%
% \begin{variable}[added=2019-09-03]{\blank_wd_dim}
% \changes{v3.1.3}{2019/09/03}{新增长度}
% 此长度为填空题生成空白的答案的长度.
%
%    \begin{macrocode}
\dim_new:N \blank_wd_dim
%    \end{macrocode}
% \end{variable}
%
% \begin{variable}[added=2019-09-10]{\get_rec_linewd_dim}
% \changes{v3.1.3}{2019/09/10}{新增长度}
% 此长度为生成矩形行数时的专有长度,不与其它程序共用.
%
%    \begin{macrocode}
\dim_new:N \get_rec_linewd_dim
%    \end{macrocode}
% \end{variable}
% 
% \begin{variable}[added=2019-09-27]{\cexam_picwd_limit}
% \changes{v3.1.9}{2019/09/27}{新增长度}
% 限制图片宽度,设置为行宽的一半,若超过一半则使用 \cs{cexam_type_iii:nnnnnnn} 排版.
%
%    \begin{macrocode}
\dim_new:N \cexam_picwd_limit 
%    \end{macrocode}
% \end{variable}
% 
% \begin{variable}[added=2020-05-01]{\cexam_ccwd_dim}
% \changes{v3.2.6}{2020/05/01}{新增长度\cs{cexam_ccwd_dim},取消对\pkg{ctex}的依赖}
%
%    \begin{macrocode}
\dim_new:N \cexam_ccwd_dim
\cs_if_exist:NTF \ccwd 
{\dim_set:Nn \cexam_ccwd_dim {\ccwd}}
{\dim_set:Nn \cexam_ccwd_dim {1em}}
%    \end{macrocode}
% \end{variable}
% 
% \begin{variable}[added=2019-10-20]{\cho_hat_dim , \cho_hat_wd_dim,\cho_hat_ht_dim}
% \changes{v3.2.1}{2019/10/20}{新增长度}
% \changes{v3.2.5}{2020/03/27}{新增长度\cs{cho_hat_ht_dim}}
% 此命令用来设置选择题四个选项与A,B,C,D的间隔。不论何种排版可以达到一致的效果。
%
%    \begin{macrocode}
\dim_new:N \cho_hat_dim
\dim_new:N \cho_hat_wd_dim
\dim_set:Nn \cho_hat_dim {.3\cexam_ccwd_dim}
\dim_set:Nn \cho_hat_wd_dim {1.2\cexam_ccwd_dim}
\dim_add:Nn \cho_hat_wd_dim {\cho_hat_dim}
\dim_new:N \cho_hat_ht_dim 
\dim_set:Nn \cho_hat_ht_dim {.7\cexam_ccwd_dim}
%    \end{macrocode}
% \end{variable}
%
% \begin{variable}[added=2020-02-22]{\fmt_pic_t_xdim,\fmt_pic_t_ydim,\fmt_picture_xdim,\fmt_picture_ydim}
% \changes{v3.2.4}{2020/03/22}{新增长度用来在格式化图片时定位图片位置}
%
%    \begin{macrocode}
\dim_new:N \fmt_pic_t_xdim
\dim_new:N \fmt_pic_t_ydim 
\dim_new:N \fmt_picture_xdim
\dim_new:N \fmt_picture_ydim
%    \end{macrocode}
% \end{variable}
% 
% \subsection{计数器设置}
%
% \begin{variable}[added=2019-08-29]{\cexam_number_int}
% \changes{v3.1.2}{2019/08/29}{新增题号计数器}
%
%    \begin{macrocode}
\int_new:N \cexam_number_int
%    \end{macrocode}
% \end{variable}
%
% \begin{variable}[added=2019-08-03]{\cexam_equ_int}
% 此计数器用来在测行时,数学公式的计数器会增加,所以此计数器对数学公式部分取得高度后数学公式计数器的还原.
%
%    \begin{macrocode}
\int_new:N \cexam_equ_int
%    \end{macrocode}
% \end{variable}
%
% \begin{variable}[added=2019-08-14]{\cexam_numtemp_int}
% 此计数器在计算行数时,临时使用.
%
%    \begin{macrocode}
\int_new:N  \cexam_numtemp_int
%    \end{macrocode}
% \end{variable}
%
% \begin{variable}[added=2019-08-14]{\cexam_picmath_int,\cexam_totalnum_int}
% 此二计数器分别记录图片的高度所生成的行数,图片之后一级,二级缩进的总行数,
%
%    \begin{macrocode}
\int_new:N \cexam_picmath_int
\int_new:N \cexam_totalnum_int
%    \end{macrocode}
% \end{variable}
%
% \begin{variable}[added=2019-09-03]{\cexam_qitem_int}
% \changes{v3.1.3}{2019/09/03}{新增计算题小问计数器}
%
%    \begin{macrocode}
\int_new:N \cexam_qitem_int
%    \end{macrocode}
% \end{variable}
% 
% \begin{variable}[added=2019-09-19]{\example_number_int,\cexam_numold_int}
% \changes{v3.1.7}{2019/09/19}{新增存储题号计数器和例题环境题号计数器}
%
%    \begin{macrocode}
\int_new:N \example_number_int
\int_new:N \cexam_numold_int
%    \end{macrocode}
% \end{variable}
%
% \begin{variable}[added=2021-02-05]{\choice_opta_int,\choice_optb_int,\choice_optc_int,\choice_optd_int ,\choice_optabcd_int}
% \changes{v3.3.4}{2021/02/05}{新增命令}
% 四个计数器用来存储随机排版选择题选项时的四个随机数。
% 
%    \begin{macrocode}
\int_new:N \choice_opta_int
\int_new:N \choice_optb_int
\int_new:N \choice_optc_int
\int_new:N \choice_optd_int
\int_new:N \choice_optabcd_int
%    \end{macrocode}
% \end{variable}
%
% \begin{variable}[added=2021-02-06]{\choice_option_it}
% \changes{v3.3.5}{2021/02/06}{新增命令}
% 存储选择题选项的个数。
% 
%    \begin{macrocode}
\int_new:N \choice_option_int
%    \end{macrocode}
% \end{variable}
%
% \begin{variable}[added=2021-02-25]{\source_star_int}
% \changes{v3.3.6}{2021/02/25}{新增命令,显示题目星级}
% 题目评级星号数量。
% 
%    \begin{macrocode}
\int_new:N \source_star_int
%    \end{macrocode}
% \end{variable}
%
% \subsection{字符串变量}
% 
% \begin{macro}[added=2019-02-16]{\sep_hd_tl , \sep_bd_tl ,\sep_tl_tl}
% \changes{v3.2.2}{2019/02/16}{新增字符串变量}
% 
% 此处所设置字符串变量用于数学文本和常规文本的分离中,及生成矩形行数时累加字符串头部内容。
%    \begin{macrocode}
\tl_new:N\sep_hd_tl
\tl_new:N\sep_bd_tl
\tl_new:N\sep_tl_tl
\tl_new:N\sep_HD_tl
%    \end{macrocode}
% \end{macro}
% 
% \begin{macro}[added=2019-02-16]{\cho_fmt_tl}
% \changes{v3.2.2}{2019/02/16}{新增字符串变量}
% 选择题格式化时所加空白
% 
%    \begin{macrocode}
\tl_new:N\cho_fmt_tl
%    \end{macrocode}
% \end{macro}
% 
% 
% \begin{macro}[added=2019-02-16]{\cexam_number_tag_tl ,\cexam_number_tag_i_tl}
% \changes{v3.2.2}{2019/02/16}{新增字符串变量}
% 此处字符串为题目的编号 
%    \begin{macrocode}
\tl_new:N \cexam_number_tag_tl
\tl_new:N \cexam_number_tag_i_tl
%    \end{macrocode}
% \end{macro}
% 
% 
% \begin{macro}[added=2019-02-16]{\ans_tag_tl ,\ana_tag_tl , \ans_tag_i_tl ,\ans_tag_i_tl}
% \changes{v3.2.2}{2019/02/16}{新增字符串变量}
% 此处字符串为答案和解析的格式
% 
%    \begin{macrocode}
\tl_new:N \ans_tag_tl
\tl_new:N \ans_tag_i_tl
\tl_new:N \ana_tag_tl
\tl_new:N \ana_tag_i_tl
%    \end{macrocode}
% \end{macro}
% 
% 
% \begin{macro}[added=2020-07-14]{\prf_tag_tl,\prf_tag_i_tl}
% \changes{v3.2.8}{2020/07/14}{新增证明题标签}
% \changes{v3.3.0}{2020/07/27}{删除\cs{prf_end_tl}}
%
%    \begin{macrocode}
\tl_new:N \prf_tag_tl
\tl_new:N \prf_tag_i_tl
%    \end{macrocode}
% \end{macro}
%
% \begin{macro}[added=2019-02-16]{\cexam_anspub_tl ,\cexam_quad_tl ,\choice_ans_tl}
% \changes{v3.2.2}{2019/02/16}{新增字符串变量}
% \changes{v3.3.4}{2021/02/05}{修改\cs{cexam_blank_tl}为\cs{cexam_anspub_tl},选择题和填空题共用}
% \changes{v3.3.4}{2021/02/06}{增加选择题专用答案存储字符串\cs{choice_ans_tl}}
% 存储填空题答案
%    \begin{macrocode}
\tl_new:N \cexam_anspub_tl
\tl_new:N \choice_ans_tl
\tl_const:Nn\cexam_quad_tl {\rule[-2pt]{\cexam_ccwd_dim}{0.4pt}}
%    \end{macrocode}
% \end{macro}
%
% \begin{macro}[added=2019-08-15]{\cexam_fmt_tag_tl,\cexam_picture_tl}
%
% 此字符串存储了图片编号的格式,如果需要修改,则可以修改这个命令.
%    \begin{macrocode}
\tl_new:N \cexam_fmt_tag_tl
\tl_new:N \cexam_picture_tl
%    \end{macrocode}
% \end{macro}
%
% \begin{macro}[added=2020-03-28]{\cexam_shape_tl}
%
% 此字符串存储了段落的形状,曾经使用\cs{cs_new:Nn}来写的,此处定义更加合理。
%    \begin{macrocode}
\tl_new:N \cexam_shape_tl
%    \end{macrocode}
% \end{macro}
% 
% 
% \begin{macro}[added=2020-07-24]{\cexam_end_tl,\ctrl_end_tl,\cexam_env_end_tl}
% 
% 此字符串用来设置题目的解析和证明的结束标志,默认为空,以后可以根据具体题型来设置。第二个\cs{cexam_env_end_tl}用来记录最后一段中并入\cs{end}时的情况,以保证最后一段不必与\cs{end}多一个空行。
%    \begin{macrocode}
\tl_new:N \cexam_end_tl
\tl_new:N \ctrl_end_tl
\tl_new:N \cexam_env_end_tl
%    \end{macrocode}
% \end{macro}
% 
% 
% \begin{macro}[added=2021-02-05]{\cho_opta_tl,\cho_optb_tl,\cho_optc_tl,\cho_optd_tl}
% 此四个字符串用来存储选择题的四个选项,以实现随机排列选项之功能。
% 
%    \begin{macrocode}
\tl_new:N\cho_opta_tl
\tl_new:N\cho_optb_tl
\tl_new:N\cho_optc_tl
\tl_new:N\cho_optd_tl
%    \end{macrocode}
% \end{macro}
%
% \begin{macro}[added=2021-02-25]{\source_color_tl}
% \changes{v3.3.6}{2021/02/25}{新增命令,题源颜色}
% 题源颜色,默认设置为黑色。
% 
%    \begin{macrocode}
\tl_new:N \source_color_tl
\tl_new:N \cexam_source_tl
\tl_set:Nn \source_color_tl {black}
%    \end{macrocode}
% \end{macro}
%
% \subsection{宏包选项}
% 
% \begin{macro}[added=2019-09-18]{\answer_write}
% \changes{v3.1.6}{2019/09/18}{新增答案写出}
%
% 答案写出命令
%    \begin{macrocode}
\iow_new:N \answer_write
%    \end{macrocode}
% \end{macro}
% 
% \begin{macro}[added=2019-09-18]{cexam/option}
% \changes{v3.1.6}{2019/09/18}{新增宏包选项}
% \changes{v3.3.4}{2021/02/05}{规范化选项设置,增加选择题随机选项控制选项option}
% \changes{v3.3.6}{2021/02/25}{增加题源显示及其颜色设置}
% \changes{v3.3.6}{2021/02/26}{增加单独控制星级和年份开关}
% 宏包选项,学生模式为答案单独写出,老师模式为不写出答案而在原题显示.
%
%    \begin{macrocode}
\keys_define:nn {cexam / option}
{
   user .choice:, 
   user / student .code:n =
   \bool_set_true:N \answer_student_bool
   \iow_open:Nn \answer_write {\jobname.ans},
   user / teacher .code:n =
   \bool_set_false:N \answer_student_bool,
   user / unknown .code:n = 
   \bool_set_false:N \answer_student_bool,
   option .choice:,
   option / random .code:n =
   \bool_set_true:N \cho_optrand_bool,
   option / unknown .code:n = 
   \bool_set_false:N \cho_optrand_bool,
   source .choice:,
   source / off .code:n =
   \bool_set_true:N \source_display_bool,
   source / unknown .code:n =
   \bool_set_false:N \source_display_bool,
   sourcecolor .choice:,
   sourcecolor / red  .code:n =
   \tl_set:Nn \source_color_tl {red},
   sourcecolor / blue .code:n =
   \tl_set:Nn \source_color_tl {blue},
   sourcecolor / green .code:n =
   \tl_set:Nn \source_color_tl {green},
   sourcecolor / unknown .code:n =
   \tl_set:Nn \source_color_tl {black},
   sourceyear .choice:,
   sourceyear / off .code:n =
   \bool_set_true:N \source_year_bool,
   sourceyear / unknown .code:n =
   \bool_set_false:N \source_year_bool,
   sourcestar .choice:,
   sourcestar / off .code:n =
   \bool_set_true:N \source_star_bool,
   sourcestar / unknown .code:n =
   \bool_set_false:N \source_star_bool,
}
\ProcessKeysOptions {cexam / option}
%    \end{macrocode}
% \end{macro}
%
% \subsection{文本和数学分离}
% 
% \begin{macro}[added=2019-05-10]{\cexam_sep_i:n , \cexam_sep_ii:n, \cexam_sep_iii:n}
%
% 三个基本数学模式分离,数学模式符号不处于字符串两端的处理
% \changes{v3.2.2}{2019/02/16}{用\hologo{LaTeX3}{}中的数据格式tokenlist 重写了数据分析结构}
%
%    \begin{macrocode}
\cs_new:Npn \cexam_sep_i:n  #1$$#2$$#3\scan_stop:
{
  \tl_set:Nn \sep_hd_tl {#1}
  \tl_set:Nn \sep_bd_tl {$$#2$$}
  \tl_set:Nn \sep_tl_tl {#3}
}
%
\cs_new:Npn \cexam_sep_ii:n #1\[#2\]#3\scan_stop:
{
  \tl_set:Nn \sep_hd_tl {#1}
  \tl_set:Nn \sep_bd_tl {\[#2\]}
  \tl_set:Nn \sep_tl_tl {#3}
}
%
\cs_new:Npn \cexam_sep_iii:n #1\begin#2\end#3#4\scan_stop:
{
  \tl_set:Nn \sep_hd_tl {#1}
  \tl_set:Nn \sep_bd_tl {\begin#2\end{#3}}
  \tl_set:Nn \sep_tl_tl {#4}
}
%    \end{macrocode}
% \end{macro}
%
% \begin{macro}[added=2019-05-10]{\cexam_sep_mk:n}
% 将三个数学模式合并为一个处理程序
%    \begin{macrocode}
\cs_new:Npn \cexam_sep_mk:n #1\scan_stop:
{
  \str_if_in:nnTF {#1}{$$}%$$
  {\cexam_sep_i:n #1\scan_stop:}
  {
    \str_if_in:nnTF {#1}{\[}%\]
      {\cexam_sep_ii:n #1\scan_stop:}
      {
	  \str_if_in:nnTF {#1}{\begin}
	  {\cexam_sep_iii:n #1\scan_stop:}
	  {}
      }
  }
}
%    \end{macrocode}
% \end{macro}
% \begin{macro}[added=2019-05-10]{\cexam_sep_isin:nn}
% 加入三个数学模式符号处于字符串两端的处理
% \changes{v3.2.2}{2019/02/16}{用\hologo{LaTeX3}{}中的数据格式tokenlist 重写了数据分析结构}
%    \begin{macrocode}
  \cs_new:Npn \cexam_sep_isin:nn #1#2
  {
    \str_if_in:nnTF {*#1}{*#2}
    {
      \bool_set_true:N \g_@@_sep_bd_bool
      \str_if_in:nnTF {#1*}{#2*}
      {
	 \tl_set:Nn \sep_hd_tl {}
	 \tl_set:Nn \sep_bd_tl {}
	 \tl_set:Nn \sep_tl_tl {}
	 \bool_set_false:N \g_@@_sep_tl_bool
      }
      {
	\cexam_sep_mk:n *#1\scan_stop:
	\tl_set:Nn \sep_hd_tl {}
	\bool_set_true:N \g_@@_sep_tl_bool
      }
    }
    {
      \str_if_in:nnTF {#1*}{#2*}
      {
	\bool_set_true:N \g_@@_sep_bd_bool
	\cexam_sep_mk:n #1*\scan_stop:
	\tl_set:Nn \sep_hd_tl {}
	\bool_set_false:N \g_@@_sep_tl_bool
      }
      {
	\str_if_in:nnTF {#1}{#2}
	{
	  \bool_set_true:N \g_@@_sep_bd_bool
	  \cexam_sep_mk:n #1\scan_stop:
	  \bool_set_true:N \g_@@_sep_tl_bool
	}{}
      }
    }
  }
%    \end{macrocode}
% \end{macro}
% \begin{macro}[added=2019-05-10]{\cexam_sep:n}
% 加入数学和纯文本模式混合时的分离功能,自动判断是否存在数学模式,尾部是否为空
% \changes{v3.2.2}{2019/02/16}{用\hologo{LaTeX3}{}中的数据格式tokenlist 重写了数据分析结构}
%    \begin{macrocode}
  \cs_new:Npn \cexam_sep:n #1 \scan_stop:
  {
    \str_if_in:nnTF {#1}{$$}%$$
    {
      \cexam_sep_isin:nn {#1}{$$}%$$
    }
    {
      \str_if_in:nnTF {#1}{\[}%\]
	{
	  \cexam_sep_isin:nn {#1}{\[}%\]
	  }
	  {
	    \str_if_in:nnTF {#1}{\begin}%\end
	    {
	      \cexam_sep_isin:nn {#1}{\begin}%\end
	    }
	    {
	       \tl_set:Nn \sep_hd_tl {#1}
	       \tl_set:Nn \sep_bd_tl {}
	       \tl_set:Nn \sep_tl_tl {}
	      \bool_set_false:N \g_@@_sep_tl_bool
	      \bool_set_false:N \g_@@_sep_bd_bool
	    }
	  }
      }
  }
%    \end{macrocode}
% \end{macro}
%
% \subsection{行数测定}
%
% \begin{macro}[added=2019-07-30]{\cexam_get:nNnN}
% \changes{v3.1.2}{2019/08/30}{排版中已经不再使用该程序累加行数,保留备用}
% 四个参数依次为:1计数器增量,2计数器,3行减量,4总减行高.这样设计的依据是,使待求量尽量放在前面,则在后面使用时可以在追加资料的情况下,不同程序中相同位置表示相同的量,这样可以增加程序的可读性.
% 2019年8月30日由于获得了最新的测行程序,所以大幅度对原始排版代码进行了改写,不再使用此处的行数累加程序,但是考虑到以后可能会有用,暂时保留下来.
% 
%    \begin{macrocode}
  \cs_new:Npn \cexam_get:nNnN #1#2#3#4
  {
    \dim_while_do:nNnn {#4}>{0pt}
    {
      \dim_sub:Nn {#4}{#3}
      \int_add:Nn {#2}{#1}
    }
  }
%    \end{macrocode}
% \end{macro}
%
% \subsection{排版文本高度和行数获得}
%
% \begin{macro}[added=2019-08-30]{\get_par_row:nnn}
% \changes{v3.1.2}{2019/08/30}{新增程序}
%
% 三个参量:1行数(返回),2文本宽,3文本.
% 此程序用来获得文本行数.
%    \begin{macrocode}
  \cs_new:Npn \get_par_row:nnn #1#2#3
  {
    \int_set:Nn \cexam_equ_int {\int_use:N\c@equation}
    \hbox_set:Nn \cexam_txtht_box
    {\parbox{#2}{#3\par\int_gset:Nn #1{\int_use:N \prevgraf}\quad}}
    \int_set:Nn \c@equation {\int_use:N \cexam_equ_int}
  }
%    \end{macrocode}
% \end{macro}
%
% \begin{macro}[added=2019-08-30]{\get_par_ht:nnn}
% \changes{v3.1.2}{2019/08/30}{新增程序}
%
% 三个参量:1行高(返回),2文本宽,3文本
% 此程序用来获得指定文本宽度时文本高度.
%
%    \begin{macrocode}
  \cs_new:Npn \get_par_ht:nnn #1#2#3
  {
    \int_set:Nn \cexam_equ_int {\int_use:N\c@equation}
    \hbox_set:Nn \cexam_txtht_box
    {\parbox{#2}{#3}}
    \int_set:Nn \c@equation {\int_use:N \cexam_equ_int}
    \dim_set:Nn {#1}{\box_dp:N \cexam_txtht_box}
    \dim_add:Nn {#1}{\box_ht:N \cexam_txtht_box}
  }
%    \end{macrocode}
% \end{macro}
%
% \begin{macro}[added=2019-08-30]{\get_par_rowht:nnnn}
% \changes{v3.1.2}{2019/08/30}{新增程序}
%
% 四个参量分别为:1行数(返回),2行高(返回),3文本宽,4文本高.
% 此程序获得行数和文本高
%    \begin{macrocode}
  \cs_new:Npn \get_par_rowht:nnnn #1#2#3#4 
  {
    \get_par_row:nnn {#1}{#3}{#4}
    \get_par_ht:nnn  {#2}{#3}{#4}
  }
%    \end{macrocode}
% \end{macro}
% 
% \subsection{矩形行数获得} 
%
% \begin{macro}[added=2019-07-31]{\cexam_get_rec:nnnnnn}
% \changes{v3.0.4}{2019/08/14}{修改为六参量函数}
% \changes{v3.0.7}{2019/08/15}{改进数学结尾时测行}
% \changes{v3.1.0}{2019/08/25}{精简了三行代码}
% \changes{v3.1.2}{2019/08/29}{全新改写}
% \changes{v3.1.4}{2019/09/10}{以专用行宽代之前的通用行宽}
% \changes{v3.1.9}{2019/09/27}{修复生成行后,图片高度规零}
%
% 六个参量:1计数器,2矩形高,3矩形宽,4左缩进,5右缩进,6文本(含数学文本)
%
%    \begin{macrocode}
  \cs_new:Npn \cexam_get_rec:nnnnnn #1#2#3#4#5#6
  {
%    \end{macrocode}
% 置空存储头部
%    \begin{macrocode}
    \tl_set:Nn \sep_HD_tl {}
%    \end{macrocode}
% 获得排版宽度
%    \begin{macrocode}
    \dim_set:Nn \get_rec_linewd_dim{\linewidth}
    \dim_sub:Nn \get_rec_linewd_dim{#3}
    \dim_sub:Nn \get_rec_linewd_dim{#4}
    \dim_sub:Nn \get_rec_linewd_dim{#5}
    \get_par_rowht:nnnn
    {#1}
    {\sep_HD_ht}
    {\get_rec_linewd_dim}
    {#6}
    \dim_compare:nNnTF
    {\sep_HD_ht} < {#2}
    {\dim_sub:Nn {#2}{\sep_HD_ht}}
    {
      \cexam_get_rec_i:nnnnnn
      {#1}{#2}{#3}{#4}{#5}{#6}
      \dim_set:Nn {#2}{0pt}
    }
  }
%    \end{macrocode}
% \end{macro}
%
% \begin{macro}[added=2019-07-31]{\cexam_get_rec_i:nnnnnn}
% \changes{v3.0.3}{2019/08/12}{修改为7参量,增加左缩进和右缩进}
% \changes{v3.1.2}{2019/08/29}{全新改写,并减少为六个参量}
% \changes{v3.1.2}{2019/08/31}{修复逻辑错误}
% \changes{v3.1.4}{2019/09/10}{精简代码}
% \changes{v3.2.2}{2019/02/16}{去除\cs{sep_hd_old:}}
%
% 六个参量:1计数器,2矩形高,3矩形宽,4左缩进,5右缩进,6文本(含数学文本)
%
%    \begin{macrocode}
  \cs_new:Npn \cexam_get_rec_i:nnnnnn #1#2#3#4#5#6
  { 
%    \end{macrocode}
%   分离头,干,尾
%    \begin{macrocode}
    \exp_args:No \cexam_sep:n #6 \scan_stop:
%    \end{macrocode}
%  头部并入old
%    \begin{macrocode}
     \tl_put_right:No \sep_HD_tl{\sep_hd_tl}
%    \end{macrocode}
%  取得old 的高度
%    \begin{macrocode}
    \get_par_rowht:nnnn
    {#1}
    {\sep_HD_ht}
    {\get_rec_linewd_dim}
    {\sep_HD_tl}
%    \end{macrocode}
% 对比旧高与图高
%    \begin{macrocode}
    \dim_compare:nNnTF 
    {\sep_HD_ht} > {#2}
    {
      \dim_sub:Nn \sep_HD_ht {#2}
      \dim_while_do:nNnn 
      {\sep_HD_ht} > {0pt}
      {
	\int_sub:Nn #1 {1}
	\dim_sub:Nn \sep_HD_ht {\baselineskip}
      }
%    \end{macrocode}
% 当排版后的old 高度小于5pt 时追加0行,当排版后的高度大于5pt 时,追加1行.
%    \begin{macrocode}
      \dim_compare:nNnTF 
      {\dim_abs:n{\sep_HD_ht}} < {5pt}
      {\int_add:Nn #1{0}}
      {\int_add:Nn #1{1}}
    }
    {
      \bool_if:NTF \g__cexam_sep_bd_bool
      {
%    \end{macrocode}
% 并入中部
%    \begin{macrocode}
	 \tl_put_right:No \sep_HD_tl{\sep_hd_tl}
%    \end{macrocode}
% 获得行数和高
%    \begin{macrocode}
	\get_par_rowht:nnnn
	{#1}
	{\sep_HD_ht}
	{\get_rec_linewd_dim}
	{\sep_HD_tl}
%    \end{macrocode}
% 对比旧高和图高
%    \begin{macrocode}
	\dim_compare:nNnTF
	{\sep_HD_ht} > {#2}
	{
	  \c_empty_tl %for multiplie math.
	}
	{
	  \bool_if:NTF \g__cexam_sep_tl_bool
	  {
	    \cexam_get_rec_i:nnnnnn
	    {#1}{#2}{#3}{#4}{#5}{\sep_tl_tl}
	  }
	  {\c_empty_tl}
	}
      }
      {\c_empty_tl}
    }
  }
%    \end{macrocode}
% \end{macro}
%
% \subsection{形状生成}
% \changes{v3.2.6}{2020/03/28}{删除\cs{cexam_shad:}}
% \changes{v3.2.6}{2020/03/28}{删除\cs{cexam_sha_cape:}}
%
% \begin{macro}[added=2019-09-10]{\cexam_shad_add:n}
% \changes{v3.1.4}{2019/09/10}{新增程序}
% \changes{v3.2.6}{2020/03/28}{重写此程序}
% 形状累加程序. 
% 
%    \begin{macrocode}
  \cs_new:Npn \cexam_shad_add:n #1
  {
     \tl_put_right:Nn \cexam_shape_tl {~}
     \exp_args:NNx \tl_put_right:Nn \cexam_shape_tl {\dim_use:N #1}
  }
%    \end{macrocode}
% \end{macro}
%
% \changes{v3.1.4}{2019/09/10}{删除\cs{cexam_sha_mk_i:nnnn}}
% \changes{v3.1.4}{2019/09/10}{删除\cs{cexam_sha_mk_ii:nnnnnnn}}
%
% \begin{macro}[added=2019-09-10]{\cexam_sha_mk:nnn}
% 三个参数:1计数器,2左缩进,3行宽 .
% 原因在于parshape 需要指明的就是一个左缩进和一个行宽,这符合parshape的要求.
%    \begin{macrocode}
  \cs_new:Npn \cexam_sha_mk:nnn #1#2#3
  {  
    \int_while_do:nNnn {#1} > {0}
    {
      \int_sub:Nn {#1}{1}
      \cexam_shad_add:n {#2}
      \cexam_shad_add:n {#3}
    }
  }
%    \end{macrocode}
% \end{macro}
%
% \begin{macro}[added=2019-09-10]{\cexam_shad_set:n}
% \changes{v3.2.6}{2020/03/28}{重写此程序}
% 行数设定命令
%    \begin{macrocode}
  \cs_new:Npn \cexam_shad_set:n #1
  {
     \int_add:Nn {#1}{1}
     \tl_set:Nn \cexam_shape_tl {~}
     \exp_args:NNx \tl_put_right:Nn \cexam_shape_tl {\int_use:N #1}
     \int_sub:Nn {#1}{1}
  }
%    \end{macrocode}
% \end{macro}
%
% \begin{macro}[added=2019-09-10]{\cexam_lwr_set:nnnn}
%  行格式设置,四个参量1图片位置,2图片宽度,3左缩进,4右缩进
%
%    \begin{macrocode}
  \cs_new:Npn \cexam_lwr_set:nnnn #1#2#3#4
  {
    \dim_set:Nn \cexam_pslin_dim {#3}
    \dim_set:Nn \cexam_psrin_dim {#4}
    \str_if_in:nnTF {#1}{l}
    {\dim_add:Nn \cexam_pslin_dim{#2}}
    {
      \str_if_in:nnTF {#1}{r}
      {\dim_add:Nn \cexam_psrin_dim{#2}}
      {\c_empty_tl}
    }
    \dim_set:Nn \cexam_pswd_dim {\linewidth}
    \dim_sub:Nn \cexam_pswd_dim {\cexam_pslin_dim}
    \dim_sub:Nn \cexam_pswd_dim {\cexam_psrin_dim}
  }
%    \end{macrocode}
% \end{macro}
%
% \subsection{图片格式化}
%
% \begin{macro}[added=2019-08-15]{\cexam_fmt_pic:nnnn}
% \changes{v3.0.7}{2019/08/15}{支持图片带编号和左右排版}
% \changes{v3.0.9}{2019/08/24}{增加图片居中排版格式}
% \changes{v3.0.9}{2019/08/24}{图片格式化增加编号增长命令}
% \changes{v3.1.2}{2019/08/29}{修改星标控制格式化为布尔值控制}
% \changes{v3.1.2}{2019/08/29}{改为并列结构格式化图片}
% \changes{v3.1.3}{2019/09/03}{修改为三参量及排版模式}
% \changes{v3.1.4}{2019/09/10}{删除\cs{cexam_fmt_pic:nnn}}
% \changes{v3.1.4}{2019/09/10}{修改为四参量及排版模式}
% \changes{v3.1.4}{2019/09/10}{修复图片下标在题目环境中的错误}
% \changes{v3.2.3}{2019/03/21}{增加表格格式化}
% \changes{v3.2.4}{2019/03/22}{使用\pkg{l3box}重构,不再使用\pkg{parbox}}
% \changes{v3.2.9}{2020/07/24}{修复格式化图片后高度的错误}
% \changes{v3.3.5}{2021/02/06}{使用正则表达式改写部分代码}
%
%
% 此程序用来格式化图片,获得图片的宽,高,生成参与排版的零宽度盒子.最初的设计是使用盒子生成图片,虽然在\hologo{LaTeX3}{} 中能够正确运行,但是定义到用户接口的环境后,并不能正确运行,它总是产生段落开始的一大段空白.而在\LaTeX2e{} 中使用零宽度盒子能很好的解决问题,同时考虑到题目的题号宽度是动态,所以加入了一个文本左缩进量,以解决此问题.\footnote{2019年9月3日经过努力思考得到此方法.}
% 
%    \begin{macrocode}
 \cs_new:Npn \cexam_fmt_pic:nnnn #1#2#3#4
 {
%    \end{macrocode}
% 设定图片和表格计数器
%    \begin{macrocode}
    \bool_case_true:n
    {
       {\cexam_fmt_bool && !\cexam_nopic_bool && \cexam_notab_bool}
       {
	  \int_gadd:Nn \c@figure {1}
	  \tl_set:Nn\cexam_fmt_tag_tl{\figurename~\thefigure}
       }
       {\cexam_fmt_bool && \cexam_nopic_bool && !\cexam_notab_bool}
       {
	  \int_gadd:Nn \c@table {1}
	  \tl_set:Nn\cexam_fmt_tag_tl{\tablename~\thetable}
       }
       {!\cexam_fmt_bool && !\cexam_nopic_bool && \cexam_notab_bool}
       {\tl_set:Nn \cexam_fmt_tag_tl {\figurename}}
       {!\cexam_fmt_bool && !\cexam_nopic_bool && !\cexam_notab_bool}
       {\tl_set:Nn \cexam_fmt_tag_tl {\tablename}}
    }
%    \end{macrocode}
% 取得图片的总体宽和高(高加深)以备后续排版用,在图片格式化后,则加入了下标说明文字,所以需要追加一行的高度。
%    \begin{macrocode}
    \vbox_set:Nn \fmt_pic_vbox{\hbox:n{#2}}
    \dim_set:Nn {\cexam_picwd_dim}{\box_wd:N \fmt_pic_vbox}
    \dim_set:Nn {\cexam_picht_dim}{\box_ht:N \fmt_pic_vbox}
    \dim_add:Nn {\cexam_picht_dim}{\box_dp:N \fmt_pic_vbox}
    \bool_if:NTF \cexam_fmt_bool
    {\dim_add:Nn {\cexam_picht_dim}{\baselineskip}}
    {\c_empty_tl}
%    \end{macrocode}
% 图片和标题组合成一个整体
%    \begin{macrocode}
    \vbox_set:Nn \fmt_pic_t_vbox{\hbox:n{\cexam_fmt_tag_tl}}
    \dim_set:Nn {\fmt_pic_t_ydim}{\cexam_picht_dim} 
    \dim_sub:Nn {\fmt_pic_t_ydim}{0.8\cexam_ccwd_dim} 
    \dim_set:Nn \fmt_pic_t_xdim{.5\box_wd:N\fmt_pic_vbox}
    \dim_sub:Nn \fmt_pic_t_xdim {.5\box_wd:N\fmt_pic_t_vbox}
    \bool_if:NTF \cexam_fmt_bool
    {
       \vbox_set:Nn \fmt_picture_box
       {
	  \box_use:N \fmt_pic_vbox
	  \box_move_right:nn{\fmt_pic_t_xdim}{\box_use:N \fmt_pic_t_vbox}
       }
    }
    {
       \vbox_set:Nn \fmt_picture_box
       {\box_use:N \fmt_pic_vbox}
    }
%    \end{macrocode}
% 根据位置设置图片版式
%    \begin{macrocode}
    \str_if_in:nnTF {#1}{l}
    {
       \dim_set:Nn \fmt_picture_xdim{\cexam_picwd_dim}
       \dim_add:Nn \fmt_picture_xdim{#3}
    }
    {
       \str_if_in:nnTF {#1}{c}
       {
	  \dim_set:Nn \cexam_pic_linwd_dim{\linewidth}
	  \dim_sub:Nn \cexam_pic_linwd_dim {#3}
	  \dim_sub:Nn \cexam_pic_linwd_dim {#4}
	  \dim_sub:Nn \cexam_pic_linwd_dim {\cexam_picwd_dim}
	  \dim_set:Nn \fmt_picture_xdim {.5\cexam_pic_linwd_dim}
       }
       {
	  \str_if_in:nnTF {#1}{r}
	  {
	     \dim_set:Nn \cexam_pic_linwd_dim{\linewidth}
	     \dim_sub:Nn \cexam_pic_linwd_dim {#3}
	     \dim_set:Nn \fmt_picture_xdim {\cexam_pic_linwd_dim}
	     \dim_sub:Nn \fmt_picture_xdim {\box_wd:N\fmt_picture_box}
	  }
	  {\c_empty_tl}
       }
    }
    \str_if_in:nnTF {#1}{l}
    {
       \vbox_set:Nn \fmt_picture_vbox
       {\box_move_left:nn {\fmt_picture_xdim}{\box_use:N \fmt_picture_box}}
    }
    {
       \vbox_set:Nn \fmt_picture_vbox
       {\box_move_right:nn {\fmt_picture_xdim}{\box_use:N \fmt_picture_box}}
    }
    \str_if_in:nnTF {#1}{c}
    {
       \hbox_set:Nn\fmt_picture_hbox{\box_use:N\fmt_picture_vbox}
    }
    {
       \box_set_ht:Nn \fmt_picture_vbox{.8\cexam_ccwd_dim}
       \hbox_set:Nn\fmt_picture_hbox
       {\box_use:N\fmt_picture_vbox} 
       \box_set_wd:Nn \fmt_picture_hbox{0pt}
    }
%    \end{macrocode}
% 定义参考排版的图片模块,加入一个\LaTeX2e{}的零宽度盒子仅仅为了定位,待\pkg{l3box}完成相应功能后再修改为\pkg{l3box}
%    \begin{macrocode}
    \tl_set:Nn \cexam_picture_tl{\makebox[0pt][r]{}\box_use:N\fmt_picture_hbox}
 }
%    \end{macrocode}
% \end{macro}
%
% \subsection{基本排版程序}
%
% \begin{macro}[added=2019-08-14]{\cexam_type_i:nnnnnnn}
% \changes{v3.0.6}{2019/08/14}{创建二级缩排程序}
% \changes{v3.0.7}{2019/08/15}{修改为七参量函数,增加图片位置格式控制}
% \changes{v3.1.3}{2019/09/03}{修改图片放置命令}
% \changes{v3.1.4}{2019/09/10}{精简长度付值重构程序}
%
% 七个参量1.图片位置(l左,r右),2.图片,3.一级左缩进,4一级右缩进,5.二级左缩进,6二级右缩进,7文本.此程序用来处理二级缩进的排版,这是在排版试题时会遇到的大多数情况.
%
%    \begin{macrocode}
  \cs_new:Npn \cexam_type_i:nnnnnnn #1#2#3#4#5#6#7
  {
%    \end{macrocode}
%
% 格式化图片
%    \begin{macrocode}
    \cexam_fmt_pic:nnnn {#1}{#2}{#3}{#4}
%    \end{macrocode}
%
% 生成一级行数
%    \begin{macrocode}
    \cexam_get_rec:nnnnnn 
    {\cexam_picmath_int}
    {\cexam_picht_dim}{\cexam_picwd_dim}
    {#3}{#4}{#7}
%    \end{macrocode}
%
% 设定一级排版长度.
%    \begin{macrocode}
    \cexam_lwr_set:nnnn
    {#1}{\cexam_picwd_dim}{#3}{#4}
%    \end{macrocode}
%
% 生成一级排版形状
%    \begin{macrocode}
    \cexam_shad_set:n {\cexam_picmath_int}
    \cexam_sha_mk:nnn
    {\cexam_picmath_int}
    {\cexam_pslin_dim}{\cexam_pswd_dim}
%    \end{macrocode}
%
% 生成二级排版形状
%    \begin{macrocode}
    \cexam_lwr_set:nnnn 
    {}{}{#5}{#6}
    \cexam_shad_add:n {\cexam_pslin_dim}
    \cexam_shad_add:n {\cexam_pswd_dim}
%    \end{macrocode}
% 执行排版任务
%    \begin{macrocode}
    \tex_parshape:D \cexam_shape_tl
    \cexam_picture_tl
    #7
  }
%    \end{macrocode}
% \end{macro}
%
% \begin{macro}[added=2019-08-14]{\cexam_type_ii:nnnnnnnnn}
%
% \changes{v3.0.6}{2019/08/14}{增加三级缩排程序}
% \changes{v3.0.7}{2019/08/15}{增加图片左右位置控制}
% \changes{v3.0.7}{2019/08/15}{整理三级缩进代码}
% \changes{v3.1.0}{2019/08/25}{由于精简了测行程序,所以此程序也精简掉了一行代码}
% \changes{v3.1.1}{2019/08/27}{去除了若尾部为空,多一行的bug}
% \changes{v3.1.2}{2019/08/29}{全新改写}
% \changes{v3.1.2}{2019/08/30}{基于新的测行程序去除微小bug}
% \changes{v3.1.4}{2019/09/10}{精简和重构程序}
%
% 九个参量:1.图片位置(l左,r右),2图片,3一级缩进,4一级右缩进,5二级左缩进,6二级右缩进,7三级左缩进,8三级右缩进,9文本.此程序用来排版三级缩进的情况,一般遇到的较少,在选择题排版时如果题干总高度超过图片时,会遇到此处的情况.
%
% 2019年8月29日重新获得更加合理的测行程序后,发现此三级缩排的情况可以更好的处理,所以专门记录一下.
%
%    \begin{macrocode}
  \cs_new:Npn \cexam_type_ii:nnnnnnnnn #1#2#3#4#5#6#7#8#9
  {
%    \end{macrocode}
% 格式化图片
%    \begin{macrocode}
    \cexam_fmt_pic:nnnn {#1}{#2}{#3}{#4}
%    \end{macrocode}
% 获得图片的排版行数
%    \begin{macrocode}
    \cexam_get_rec:nnnnnn {\cexam_picmath_int}
    {\cexam_picht_dim}{\cexam_picwd_dim}
    {#3}{#4}{#9}
%    \end{macrocode}
% 将图片行数传给 第一次排版行数
%    \begin{macrocode}
    \int_set:Nn \cexam_numtemp_int{\int_use:N \cexam_picmath_int}
%    \end{macrocode}
% 设置试排版行数
%    \begin{macrocode}
    \cexam_shad_set:n {\cexam_numtemp_int} 
%    \end{macrocode}
% 设置一级排版行参数
%    \begin{macrocode}
    \cexam_lwr_set:nnnn
    {#1}{\cexam_picwd_dim}{#3}{#4}
%    \end{macrocode}
% 生成形状
%    \begin{macrocode}
    \cexam_sha_mk:nnn
    {\cexam_numtemp_int}
    {\cexam_pslin_dim}{\cexam_pswd_dim}
%    \end{macrocode}
% 设置二级排版行参数,并生成形状
%    \begin{macrocode}
    \cexam_lwr_set:nnnn
    {}{}{#5}{#6}
    \cexam_shad_add:n {\cexam_pslin_dim}
    \cexam_shad_add:n {\cexam_pswd_dim}
%    \end{macrocode}
% 获得图片后的排版行数
%    \begin{macrocode}
    \get_par_row:nnn
    {\cexam_totalnum_int}
    {\cexam_pswd_dim}
    {
      \tex_parshape:D \cexam_shape_tl
      #9
    }
%    \end{macrocode}
% 设置图片之后行数
%    \begin{macrocode}
    \int_set:Nn \cexam_numtemp_int {\int_use:N \cexam_totalnum_int}
    \int_sub:Nn \cexam_numtemp_int {\cexam_picmath_int}
%    \end{macrocode}
% 生成最终形状,设置总行数
%    \begin{macrocode}
    \cexam_shad_set:n {\cexam_totalnum_int} 
%    \end{macrocode}
% 生成一级行参数及形状
%    \begin{macrocode}
    \cexam_lwr_set:nnnn
    {#1}{\cexam_picwd_dim}{#3}{#4}
    \cexam_sha_mk:nnn
    {\cexam_picmath_int}
    {\cexam_pslin_dim}{\cexam_pswd_dim}
%    \end{macrocode}
%  生成二级行参数及形状
%    \begin{macrocode}
    \cexam_lwr_set:nnnn 
    {}{}{#5}{#6}
    \cexam_sha_mk:nnn
    {\cexam_numtemp_int}
    {\cexam_pslin_dim}{\cexam_pswd_dim}
%    \end{macrocode}
%生成三级行参数及形状
%    \begin{macrocode}
    \cexam_lwr_set:nnnn 
    {}{}{#7}{#8}
    \cexam_shad_add:n {\cexam_pslin_dim}
    \cexam_shad_add:n {\cexam_pswd_dim}
%    \end{macrocode}
% 排版
%    \begin{macrocode}
    \tex_parshape:D \cexam_shape_tl
    \cexam_picture_tl
    #9
  }
%    \end{macrocode}
% \end{macro}
%
% \begin{macro}[added=2019-08-24]{\cexam_type_iii:nnnnnnn}
% \changes{v3.0.9}{2019/08/24}{增加图片居中排版程序}
% \changes{v3.1.2}{2019/08/30}{使用新的测行程序改写}
% \changes{v3.1.4}{2019/09/10}{精简代码,重构部分程序}
%
% 七个参数依次为:1.图片位置,2图片,3一级左缩进,4一级右缩进,5二级左缩进,6二级右缩进,7文本
%    \begin{macrocode}
  \cs_new:Npn \cexam_type_iii:nnnnnnn #1#2#3#4#5#6#7
  {
%    \end{macrocode}
% 设置一级行参数
%    \begin{macrocode}
    \cexam_lwr_set:nnnn
    {}{}{#3}{#4}
%    \end{macrocode}
% 格式化图片
%    \begin{macrocode}
    \cexam_fmt_pic:nnnn {c}{#2}{#3}{#4}
%    \end{macrocode}
% 获得文本行数
%    \begin{macrocode}
    \get_par_row:nnn 
    {\cexam_picmath_int}
    {\cexam_pswd_dim}{#7}
%    \end{macrocode}
% 追加一行用以排版图片和后面的选项.
%    \begin{macrocode}
    \int_add:Nn \cexam_picmath_int {1}
%    \end{macrocode}
% 设置排版总行数
%    \begin{macrocode}
    \cexam_shad_set:n {\cexam_picmath_int}
%    \end{macrocode}
% 生成一级段落形状
%    \begin{macrocode}
    \cexam_sha_mk:nnn
    {\cexam_picmath_int}
    {\cexam_pslin_dim}{\cexam_pswd_dim}
%    \end{macrocode}
% 设置二级段落形状
%    \begin{macrocode}
    \cexam_lwr_set:nnnn
    {}{}{#5}{#6}
    \cexam_shad_add:n {\cexam_pslin_dim}
    \cexam_shad_add:n {\cexam_pswd_dim}
%    \end{macrocode}
% 开始排版图片和文字
%    \begin{macrocode}
    \tex_parshape:D \cexam_shape_tl
    #7
    \vspace{5pt} 
    \newline
    \cexam_picture_tl
  }
%    \end{macrocode}
% \end{macro}
%
% \begin{macro}[added=2019-08-27]{\cexam_type_iv:nnnnnnnn}
% \changes{v3.1.1}{2019/08/27}{新增图文排版,取代原纯文本排版}
% \changes{v3.1.2}{2019/08/28}{去除二级缩进的代码置0}
% \changes{v3.1.2}{2019/08/31}{使用新的测行程序重新设计了代码}
% \changes{v3.1.4}{2019/09/10}{精简并重构部分代码}
% \changes{v3.1.7}{2019/09/19}{修复二级缩进错误}
% \changes{v3.1.8}{2019/09/26}{修复二级缩进错误}
% \changes{v3.1.9}{2019/09/27}{修复题高小于图高时的自动填充空白}
% 
% 八个参数:1图片位置,2图片,3一级左缩进,4一级右缩进,5二级左缩进,6二级右缩进7主文本,8副文本.此程序用来排版当选项与题干的总高大于图高,但是题干高度低于图高的情况.
%
% 由于重新设计实现了测行程序,所以在测量行数时不需要单独置零行数计数器,故精简了一行代码.  
%  
%    \begin{macrocode}
  \cs_new:Npn \cexam_type_iv:nnnnnnnn #1#2#3#4#5#6#7#8
  {
%    \end{macrocode}
% 格式化图片,由于此模式图片居左排版相当不美观,所以取消其左排模式,凡进入者皆图片右排.
%    \begin{macrocode}
    \cexam_fmt_pic:nnnn {r}{#2}{#3}{#4}
%    \end{macrocode}
% 取得主文本行数,文本高小于图片高
%    \begin{macrocode}
    \cexam_lwr_set:nnnn
    {r}{\cexam_picwd_dim}{#3}{#4}
    \get_par_rowht:nnnn
    {\cexam_picmath_int}
    {\rec_tempht_dim}
    {\cexam_pswd_dim}
    {#7}
    \dim_sub:Nn \cexam_picht_dim{\rec_tempht_dim}
%    \end{macrocode}
% 取得副文本行数,副文本高度大于图片的剩余高度
%    \begin{macrocode}
    \cexam_get_rec:nnnnnn {\cexam_numtemp_int}
    {\cexam_picht_dim}{\cexam_picwd_dim}
    {#5}{#6}{#8}
%    \end{macrocode}
% 设置总行数 
%    \begin{macrocode}
    \int_set:Nn \cexam_totalnum_int {\int_use:N \cexam_picmath_int}
    \int_add:Nn \cexam_totalnum_int {\int_use:N \cexam_numtemp_int}
%    \end{macrocode}
% 生成主文本形状
%    \begin{macrocode}
    \cexam_shad_set:n {\cexam_totalnum_int}
    \cexam_sha_mk:nnn
    {\cexam_picmath_int}
    {\cexam_pslin_dim}{\cexam_pswd_dim}
%    \end{macrocode}
% 生成副文本形状,作为副文本其对应于选项,所以有左缩进,同时还有图片加入到右缩进.
%    \begin{macrocode}
    \cexam_lwr_set:nnnn
    {#1}{\cexam_picwd_dim}{#5}{#6}
    \cexam_sha_mk:nnn
    {\cexam_numtemp_int}
    {\cexam_pslin_dim}{\cexam_pswd_dim}
%    \end{macrocode}
% 生成尾行形状,保留左缩进和右缩进,但是余下部分不再有图片,所以去除图片宽度
%    \begin{macrocode}
    \cexam_lwr_set:nnnn
    {}{}{#5}{#6}
    \cexam_shad_add:n {\cexam_pslin_dim}
    \cexam_shad_add:n {\cexam_pswd_dim}
%    \end{macrocode}
% 准备排版图文
%    \begin{macrocode}
    \tex_parshape:D \cexam_shape_tl
    \cexam_picture_tl
    #7
    \newline
    #8
%    \end{macrocode}
% 当图高大于题高时,为了防止图片与下一题重合,则追加图高减题高一样大的空白
%    \begin{macrocode}
    \dim_compare:nNnTF 
    {\cexam_picht_dim} > {0pt}
    {\vspace{\cexam_picht_dim}}
    {\c_empty_tl}
  }
%    \end{macrocode}
% \end{macro}
%
% \begin{macro}[added=2019-08-24]{\cexam_type_v:nnnnn}
% \changes{v3.0.9}{2019/08/24}{增加无图排版模式}
% \changes{v3.1.1}{2019/08/27}{排版号由iv增加一个,变为v}
% \changes{v3.1.2}{2019/08/29}{精简两行代码}
% \changes{v3.1.2}{2019/08/30}{使用新的测行程序重新设计了代码}
% \changes{v3.1.4}{2019/09/10}{精简并重构部分代码}
%
% 五个参数依次为:1.一级左缩进,2一级右缩进,3二级左缩进,4二级右缩进,5文本
%
% 2019年8月29日重新设计了测行程序,所以借助最新的测行程序重新设计了该程序.
%
%    \begin{macrocode}
  \cs_new:Npn \cexam_type_v:nnnnn #1#2#3#4#5
  {
%    \end{macrocode}
% 设置一级行参数
%    \begin{macrocode}
    \cexam_lwr_set:nnnn
    {}{}{#1}{#2}
%    \end{macrocode}
% 获得文本行数
%    \begin{macrocode}
    \get_par_row:nnn
    {\cexam_picmath_int}{\cexam_pswd_dim}{#5}
%    \end{macrocode}
% 设定行数
%    \begin{macrocode}
    \cexam_shad_set:n {\cexam_picmath_int} 
%    \end{macrocode}
% 生成一级段落形状
%    \begin{macrocode}
    \cexam_sha_mk:nnn
    {\cexam_picmath_int}
    {\cexam_pslin_dim}{\cexam_pswd_dim}
%    \end{macrocode}
% 生成二级段落形状
%    \begin{macrocode}
    \cexam_lwr_set:nnnn
    {}{}{#3}{#4}
    \cexam_shad_add:n {\cexam_pslin_dim}
    \cexam_shad_add:n {\cexam_pswd_dim}
%    \end{macrocode}
% 开始排版图片和文字
%    \begin{macrocode}
    \tex_parshape:D \cexam_shape_tl
    #5
  }
%    \end{macrocode}
% \end{macro}
%
% \subsection{图片与文字的分离}
%
% \begin{macro}[added=2019-08-25]{\cexam_sep_pictab_tl,\cexam_sep_txt_tl}
% 此处二个命令分别用来保存图片与文字分离后的图片和文本.初始设置为空.
% \changes{v3.2.3}{2020/03/21}{将原来的控制序列修改为字符串格式}
%    \begin{macrocode}
\tl_new:N \cexam_sep_pictab_tl 
\tl_new:N \cexam_sep_txt_tl 
\tl_new:N \cexam_sep_nopic_tl
%    \end{macrocode}
% \end{macro}
%
% \begin{macro}[added=2019-09-11]{\cexam_sep_nopic_tl}
% 当图片过小或者过大时,所设置的默认方框,用以参与排版。同时messgae 在终端给出提示。
%    \begin{macrocode}
\tl_set:Nn \cexam_sep_nopic_tl
{
   \draw_begin:
   \draw_color:n {blue}
   \draw_linewidth:n {2pt}
   \draw_path_rectangle:nn
   {0cm ,0cm}
   {2.4cm ,2.4cm}
   \hcoffin_set:Nn\l_tmpa_coffin
   {\color_group_begin:\color_select:n{red}SMALL\color_group_end:}
   \draw_transform_xshift:n {1.2cm}
   \draw_transform_yshift:n {1.2cm}
   \draw_coffin_use:Nnn \l_tmpa_coffin {hc}{vc}
   \draw_path_use_clear:n {draw}
   \draw_end:
}
%    \end{macrocode}
% \end{macro}
%
%
% \begin{macro}[added=2019-08-25]{picture}
% \changes{v3.1.0}{2019/08/25}{增加图片与文本初级分离程序}
% \changes{v3.1.2}{2019/08/29}{修改无图时的提醒格式}
% \changes{v3.1.3}{2019/09/03}{图片界定符换为<<和>>}
% \changes{v3.1.5}{2019/09/11}{增加对图片尺寸的探测,并限制大图}
% \changes{v3.1.5}{2019/09/11}{加入message 提示图片太大和太小}
% \changes{v3.1.9}{2019/09/27}{允许通过较宽的图片,限制图高为半个行宽}
% \changes{v3.2.3}{2020/03/21}{删除命令\cs{cexam_sep_pictxt_i:p},同时删除定界符}
%
% 在\pkg{v3.2.3}版中删除了定界符,改成自动判断是否存在图片(或表格),这样做就不需要判断是否忘记加入图片(或表格),所以精简掉了一个警告消息。
% 在老师们输入试题时,由于选用的图片不一定清楚它的具体尺寸,所以有的时候过小有的时候过大了.在过小的时候我假定图片的宽度比 5pt 还要小,此时认为图片不存在,同时向终端发出一条警告.在图片过大时,这时我认为图宽大于 0.6\cs{baselineskip} (或图高大于此值)则图片过大, 同时向终端发出一条警告,用以提醒作者修改对应题目的图片.
%
%    \begin{macrocode}
\msg_new:nnn {cexam}{picture}
{The~picture~of~problem~ #1~too~#2~,it~will~be~replaced~by~a~rectangle.}
%    \end{macrocode}
% \end{macro}
%
% \begin{macro}[added=2020-03-21]{\cexam_sep_graphics:p}
% 此命令用来判断题目主干中是否以\pkg{graphic}或\pkg{graphicx}宏包插入了图片,由于它是含有参数的,所以将各种类型进行独立分离,最后合并成一个命令。
%    \begin{macrocode}
\cs_new:Npn \cexam_sep_pictxt_is:p #1\includegraphics*[#2][#3]#4#5\scan_stop:
{
   \tl_set:Nn \cexam_sep_txt_tl {#1\cexam_fmt_tag_tl#5}
   \tl_set:Nn \cexam_sep_pictab_tl {\includegraphics*[#2][#3]{#4}}
}
\cs_new:Npn \cexam_sep_pictxt_i:p #1\includegraphics[#2][#3]#4#5\scan_stop:
{
   \tl_set:Nn \cexam_sep_txt_tl {#1\cexam_fmt_tag_tl#5}
   \tl_set:Nn \cexam_sep_pictab_tl {\includegraphics[#2][#3]{#4}}
}
\cs_new:Npn \cexam_septxt_iis:p #1\includegraphics*[#2]#3#4\scan_stop:
{
   \tl_set:Nn \cexam_sep_txt_tl {#1\cexam_fmt_tag_tl#4}
   \tl_set:Nn \cexam_sep_pictab_tl {\includegraphics*[#2]{#3}}
}
\cs_new:Npn \cexam_septxt_ii:p #1\includegraphics[#2]#3#4\scan_stop:
{
   \tl_set:Nn \cexam_sep_txt_tl {#1\cexam_fmt_tag_tl#4}
   \tl_set:Nn \cexam_sep_pictab_tl {\includegraphics[#2]{#3}}
}
\cs_new:Npn \cexam_sep_pictxt_iii:p #1\includegraphics#2#3\scan_stop:
{
   \tl_set:Nn \cexam_sep_txt_tl {#1\cexam_fmt_tag_tl#3}
   \tl_set:Nn \cexam_sep_pictab_tl {\includegraphics{#2}}
}
\cs_new:Npn \cexam_sep_pictxt_iiis:p #1\includegraphics*#2#3\scan_stop:
{
   \tl_set:Nn \cexam_sep_txt_tl {#1\cexam_fmt_tag_tl#3}
   \tl_set:Nn \cexam_sep_pictab_tl {\includegraphics*{#2}}
}
\cs_new:Npn \cexam_sep_graphics:p #1 \scan_stop:
{
   \bool_set_false:N \cexam_nopic_bool
   \bool_set_true:N \cexam_notab_bool
   \str_if_in:nnTF {#1}{\includegraphics*}
   {
      \str_if_in:nnTF {#1}{\includegraphics*[}
	 {
	    \str_if_in:nnTF {#1}{][}
	    {\cexam_sep_pictxt_is:p #1\scan_stop:}
	    {\cexam_septxt_iis:p #1\scan_stop:}
	 }
	 {\cexam_sep_pictxt_iiis:p #1\scan_stop:}
      }
      {
	 \str_if_in:nnTF {#1}{\includegraphics[}
	    {
	       \str_if_in:nnTF {#1}{][}
	       {\cexam_sep_pictxt_i:p #1\scan_stop:}
	       {\cexam_septxt_ii:p #1\scan_stop:}
	    }
	    {\cexam_sep_pictxt_iii:p #1\scan_stop:}
	 }
      }
%    \end{macrocode}
% \end{macro}
%
% \begin{macro}[added=2020-03-21]{\cexam_sep_tikz:p}
% 此命令用来自动判断题目中是否插入了图片或者表格,同时无论图片或者表格都将判定为题目中存在图片,借用\cs{cexam_nopic_bool}来进行下一步排版的判断。这里布尔值\cs{cexam_notab_bool}仅仅用来决定在出现表格时修改表格在主文本中的替换文字为\pkg{表 xx.x} 同时表格下方的标题也修改为\pkg{表 xx.x},所以在执行图片(表格)与文本分离过程中自动设置好布尔值。
% \changes{v3.3.0}{2020/07/28}{修复无图时布尔值设置的错误}
%
%    \begin{macrocode}
      \cs_new:Npn \cexam_sep_tikz:p #1\begin#2#3\end#4#5 \scan_stop:
      {
	 \str_if_in:nnTF {#2}{tikzpicture}
	 { 
	    \bool_set_false:N \cexam_nopic_bool
	    \bool_set_true:N \cexam_notab_bool
	    \tl_set:Nn \cexam_sep_txt_tl {#1\cexam_fmt_tag_tl#5}
	    \tl_set:Nn \cexam_sep_pictab_tl {\begin{#2} #3\end{#4}}
	 }
	 {
	    \str_if_in:nnTF {#2}{tabular}
	    {
	       \bool_set_false:N \cexam_nopic_bool
	       \bool_set_false:N \cexam_notab_bool
	       \tl_set:Nn \cexam_sep_txt_tl {#1\cexam_fmt_tag_tl#5}
	       \tl_set:Nn \cexam_sep_pictab_tl {\begin{#2} #3\end{#4}}
	    }
	    {
	       \str_if_in:nnTF {#2}{arry}
	       {
		  \bool_set_false:N \cexam_nopic_bool
		  \bool_set_false:N \cexam_notab_bool
		  \tl_set:Nn \cexam_sep_txt_tl {#1\cexam_fmt_tag_tl#5}
		  \tl_set:Nn \cexam_sep_pictab_tl {\begin{#2} #3\end{#4}}
	       }
	       {
		  \bool_set_true:N \cexam_nopic_bool
		  \bool_set_true:N \cexam_notab_bool
		  \tl_set:Nn \cexam_sep_txt_tl {#1\begin{#2}#3\end{#4}#5}
		  \tl_set:Nn \cexam_sep_pictab_tl {}
	       }
	    }
	 }
      }
%    \end{macrocode}
% \end{macro}
%
% \begin{macro}[added=2020-09-23]{\cexam_sep_multiply_i:p,\cexam_sep_multiply_ii:p}
%  此命令用来解决题目中存在多图并排在一块时的情况,尽管实现了系统自动判断图片和分离图片功能,但是存在多图时如果再自动判断,则实现起来太过复杂,所以此处加入专门的分隔符号 <BeginPicture> 和<EndPicture>来分隔图片,二者内的所有部分将作为整体视为一个图片排版。如果题目中出现多个表格并排时,以<BeginTabular> 和 <EndTabular> 来分隔表格,二者内的所有部分将作为一个表格排版。
% \changes{v3.3.1}{2020/09/23}{加入多图多表并排的处理}
%    \begin{macrocode}
\cs_new:Npn \cexam_sep_multiply_i:p #1<BeginPicture>#2<EndPicture>#3 \scan_stop:
{
   \bool_set_false:N \cexam_nopic_bool
   \bool_set_true:N \cexam_notab_bool
   \tl_set:Nn \cexam_sep_txt_tl {#1\cexam_fmt_tag_tl#3}
   \tl_set:Nn \cexam_sep_pictab_tl {#2}
}
\cs_new:Npn \cexam_sep_multiply_ii:p #1<BeginTabular>#2<EndTabular>#3 \scan_stop:
{
   \bool_set_false:N \cexam_nopic_bool
   \bool_set_false:N \cexam_notab_bool
   \tl_set:Nn \cexam_sep_txt_tl {#1\cexam_fmt_tag_tl#3}
   \tl_set:Nn \cexam_sep_pictab_tl {#2}
}
%    \end{macrocode}
% \end{macro}
%
% \subsection{前缀设置}
%
% \begin{macro}[added=2019-08-25]{\cexam_ind_hat:nnnn}
% \changes{v3.1.0}{2019/08/25}{增加前缀设置程序}
% \changes{v3.1.7}{2019/09/19}{由原来的二参量改为三参量}
% \changes{v3.2.2}{2019/02/16}{修改下沉量为0.01}
% \changes{v3.2.5}{2020/03/27}{修改为四个参量,加入高度参量,重写代码,去除\cs{parbox}}
% 三个参量:1宽度,2高度,3左缩进部分,4加入到文本部分
% 此程序用来生成前缀,如题号,选择题选项前的标号等.由于去除了\cs{parbox}用\pkg{l3box}重写了代码,所以出现了专用的盒子和长度,这里不移到前面的原因也在这里。同时,由于\pkg{l3box}的原理与\LaTeX2e{}中的零宽度盒子多少有些不同,当\pkg{l3box}的零宽度盒子位于段落段开头时,它不能正确定位,所以最后加入了一个\LaTeX2e{}的零宽度盒子,纯粹为了定位,一旦\pkg{l3box}实现了同样的功能此处应当修改为\pkg{l3box}。
%
%    \begin{macrocode}
\cs_new:Npn \cexam_ind_hat:nnnn #1#2#3#4
{
   \vbox_set:Nn \ind_hat_vbox{\hbox:n{#3}}
   \box_set_ht:Nn \ind_hat_vbox{#2}
   \vbox_set:Nn \ind_hat_box{\box_move_left:nn{#1}{\box_use:N \ind_hat_vbox}}
   \hbox_set:Nn \ind_hat_hbox{\box_use:N\ind_hat_box}
   \box_set_wd:Nn \ind_hat_hbox{0pt}
   \makebox[0pt][r]{}\box_use:N\ind_hat_hbox{}#4
}
%    \end{macrocode}
% \end{macro}
%
% \begin{macro}[added=2020-07-12]{\cexam_ind_hat:nnn}
% \changes{v3.2.7}{2020/07/12}{新增命令}
%
% 在题目的标号中不需要将题号进行上下移动,所以综合考虑后决定单独设置一个命令。原因在于,在选择题的四个选项中 A,B,C,D 四个选项号,放入零盒子中时高度发生变化,所以使用四个参量以设置高度,而在此处不需要设置高度,所以重新增加了此命令。
%    \begin{macrocode}
\cs_new:Npn \cexam_ind_hat:nnn #1#2#3
{
   \vbox_set:Nn \ind_hat_vbox{\hbox:n{#2}}
   \vbox_set:Nn \ind_hat_box{\box_move_left:nn{#1}{\box_use:N \ind_hat_vbox}}
   \hbox_set:Nn \ind_hat_hbox{\box_use:N\ind_hat_box}
   \box_set_wd:Nn \ind_hat_hbox{0pt}
   \makebox[0pt][r]{}\box_use:N\ind_hat_hbox{}#3
}
%    \end{macrocode}
% \end{macro}
% 
% \subsection{选择题的排版}
% 
% \begin{macro}[added=2019-08-24]{\cho_get_lmax:nn}
% \changes{v3.0.9}{2019/08/24}{增加选择题选项最大长度获得程序}
% \changes{v3.2.0}{2019/10/13}{删除\cs{cho_get_lmax:n}}
% \changes{v3.2.0}{2019/10/13}{新增\cs{cho_get_lmax:nn}}
%
% 此程序并不复杂,在\LaTeX2e{}版本中,我曾单独写出了这支程序,但是在\hologo{LaTeX3}{} 中给出了一个标准的取得最大长度的程序\cs{dim_max:nn},所以在此版本中,我选择了这个标准的程序来获得最大选项长度.
% 
% 在2019年10月13日,考虑优化选择题选项排版时,由于二行排版选项时需要对比AB的最大长度和CD的最大长度,所以此处决定升级为双参量函数。
% 
%    \begin{macrocode}
\cs_new:Npn \cho_get_lmax:nn #1#2 
{
  \hbox_set:Nn \cho_option_box{#2}
  \dim_set:Nn #1{\dim_max:nn {#1}{\box_wd:N \cho_option_box}}
}
%    \end{macrocode}
% \end{macro}
% 
% 
% \begin{macro}[added=2019-10-20]{\cho_fmt_tl}
% \changes{v3.2.1}{2019/10/20}{新增命令}
% \changes{v3.2.2}{2019/02/16}{修改为\cs{cho_fmt_tl}}
%  此命令用来规范选择题四个选项中A.B.C.D.与其内容的间隔,默认值已经在长度定义时设置。
%    \begin{macrocode}
\tl_set:Nn\cho_fmt_tl{\raisebox{-0.2pt}{.}\hspace*{\cho_hat_dim}}
%    \end{macrocode}
% \end{macro}
%
% \begin{macro}[added=2019-08-25]{\cho_opt_type_i:nnnn}
% \changes{v3.1.0}{2019/08/25}{增加选择题短选项一行排版}
% \changes{v3.1.2}{2019/08/29}{追加了每个选项的排版宽度}
% \changes{v3.1.2}{2019/08/30}{改写了排版选项,以解决水平盒子偶然过宽问题}
% \changes{v3.2.0}{2019/10/12}{优化了单行排版}
% \changes{v3.2.1}{2019/10/20}{规范了选项间隔}
%
%    \begin{macrocode}
\cs_new:Npn \cho_opt_type_i:nnnn #1#2#3#4
{
   A\cho_fmt_tl#1\hfill
   B\cho_fmt_tl#2\hfill
   C\cho_fmt_tl#3\hfill
   D\cho_fmt_tl#4\hspace*{\cho_optwd_i_dim}
}
%    \end{macrocode}
% \end{macro}
%
% \begin{macro}[added=2019-08-25]{\cho_opt_type_ii:nnnn}
% \changes{v3.1.0}{2019/08/25}{增加选择题中选项二行排版}
% \changes{v3.1.2}{2019/08/29}{追加了每个选项的排版宽度}
% \changes{v3.1.2}{2019/08/30}{改写了排版选项,以解决水平盒子偶然过宽问题}
% \changes{v3.2.1}{2019/10/20}{规范了选项间隔}
%
%    \begin{macrocode}
\cs_new:Npn \cho_opt_type_ii:nnnn #1#2#3#4
{
   \makebox[\cho_optwd_i_dim][l]{A\cho_fmt_tl#1}
   B\cho_fmt_tl#2
   \newline
   \makebox[\cho_optwd_i_dim][l]{C\cho_fmt_tl#3}
   D\cho_fmt_tl#4
}
%    \end{macrocode}
% \end{macro}
%
% \begin{macro}[added=2019-08-25]{\cho_opt_type_iii:nnnn}
% \changes{v3.1.0}{2019/08/25}{增加选择题长选项多行排版}
% \changes{v3.2.1}{2019/10/20}{规范了选项间隔}
% \changes{v3.2.5}{2020/03/27}{由于修改了\cs{cexam_ind_hat:nnnn},所以此处也对应做了修改}
%
%    \begin{macrocode}
\cs_new:Npn \cho_opt_type_iii:nnnn #1#2#3#4
{
   \cexam_ind_hat:nnnn {\cho_hat_wd_dim}{\cho_hat_ht_dim}{A\cho_fmt_tl}{}#1
   \newline
   \cexam_ind_hat:nnnn {\cho_hat_wd_dim}{\cho_hat_ht_dim}{B\cho_fmt_tl}{}#2
   \newline
   \cexam_ind_hat:nnnn {\cho_hat_wd_dim}{\cho_hat_ht_dim}{C\cho_fmt_tl}{}#3
   \newline
   \cexam_ind_hat:nnnn {\cho_hat_wd_dim}{\cho_hat_ht_dim}{D\cho_fmt_tl}{}#4
}
%    \end{macrocode}
% \end{macro}
%
% \begin{macro}[added=2019-08-25]{\cexam_fmt_opt_cho:nnnn}
% \changes{v3.1.0}{2019/08/25}{增加选择题选项格式化程序}
% \changes{v3.1.2}{2019/09/01}{去除选项排版不对齐bug}
% \changes{v3.1.5}{2019/09/13}{去除多题排版时,上一题选项最长对下一题的影响}
% \changes{v3.1.5}{2019/09/13}{恢复二行选项排版时,每项宽为半个行宽.}
% \changes{v3.1.9}{2019/10/10}{优化选项双行排版}
% \changes{v3.2.0}{2019/10/12}{优化选项单行排版}
% \changes{v3.2.0}{2019/10/13}{再次优化选项双行排版}
%
% 此程序用来在选择题排版之前将选项先格式化,最后参与排版.
%
%    \begin{macrocode}
\cs_new:Npn \cexam_fmt_opt_cho:nnnn #1#2#3#4
{
   \dim_set:Nn \cho_lmax_i_dim {0pt}
   \cho_get_lmax:nn {\cho_lmax_i_dim}{#1}
   \cho_get_lmax:nn {\cho_lmax_i_dim}{#3}
   \dim_set:Nn \cho_lmax_ii_dim {0pt}
   \cho_get_lmax:nn {\cho_lmax_ii_dim}{#2}
   \cho_get_lmax:nn {\cho_lmax_ii_dim}{#4}
   \dim_set:Nn \cho_lmax_dim{\dim_max:nn {\cho_lmax_i_dim}{\cho_lmax_ii_dim}}
   \dim_add:Nn \cho_lmax_dim{\cexam_ccwd_dim} 
%    \end{macrocode}
% 上述取得了选项的最大长度,但是排版时由于各选项要有一定间隔,所以加入一个字符的宽度,以保证确定选项时不会发生微小的错误。
%    \begin{macrocode}
   \dim_compare:nNnTF {\cho_lmax_dim} < {.25\cho_optwd_dim}
   {
      \bool_set_false:N \cho_opt_maxed_bool
      \dim_set:Nn \cho_optwd_i_dim {.25\cho_optwd_dim} 
      \dim_sub:Nn \cho_optwd_i_dim {\cho_lmax_dim} 
      \hbox_set:Nn \cexam_option_box {\cho_opt_type_i:nnnn {#1}{#2}{#3}{#4}}
   }
   {
      \dim_compare:nNnTF {\cho_lmax_dim} < {.5\cho_optwd_dim}
      {
	 \bool_set_false:N \cho_opt_maxed_bool
	 \dim_set:Nn \cho_optwd_i_dim {.5\cho_optwd_dim}
	 \hbox_set:Nn \cexam_option_box {\cho_opt_type_ii:nnnn {#1}{#2}{#3}{#4}}
      }
      {
%    \end{macrocode}
% 此处追加了一步判断,如果四个选项最大宽度大于0.5倍的行宽,但是AB选项中的最大宽度和四个选项的最大宽度之和有可能小于行宽,此时使用二行排版选项也是合理的。
%    \begin{macrocode}
	 \dim_add:Nn \cho_lmax_i_dim {\cho_lmax_ii_dim}
	 \dim_add:Nn \cho_lmax_i_dim {2\cexam_ccwd_dim}
	 \dim_compare:nNnTF {\cho_lmax_i_dim} < {\cho_optwd_dim}
	 {
	    \bool_set_false:N \cho_opt_maxed_bool
	    \dim_set:Nn \cho_optwd_i_dim {\cho_optwd_dim}
	    \dim_sub:Nn \cho_optwd_i_dim {\cho_lmax_ii_dim}
	    \dim_sub:Nn \cho_optwd_i_dim {\cexam_ccwd_dim}
	    \hbox_set:Nn \cexam_option_box {\cho_opt_type_ii:nnnn {#1}{#2}{#3}{#4}}
	 }
	 {
	    \bool_set_true:N \cho_opt_maxed_bool
	    \hbox_set:Nn \cexam_option_box {\cho_opt_type_iii:nnnn {#1}{#2}{#3}{#4}}
	 }
      }
   }
}
%    \end{macrocode}
% \end{macro}
%
% \begin{macro}[added=2019-08-29]{\cexam_sep_pictxt:n}
% \changes{v3.1.2}{2019/08/29}{新增图片与文字分离程序}
% \changes{v3.2.3}{2020/03/21}{完全改写此命令}
% \changes{v3.3.1}{2020/09/23}{加入多图或多表模式}
%
%    \begin{macrocode}
\cs_new:Npn \cexam_sep_pictxt:n #1 
{
   \str_if_in:nnTF {#1}{<BeginPicture>}
   {\cexam_sep_multiply_i:p \c_empty_tl #1\c_empty_tl\scan_stop:}
   {
      \str_if_in:nnTF {#1}{<BeginTabular>}
      {\cexam_sep_multiply_ii:p \c_empty_tl #1\c_empty_tl\scan_stop:}
      {
	 \str_if_in:nnTF {#1}{\includegraphics}
	 {\cexam_sep_graphics:p \c_empty_tl #1\c_empty_tl\scan_stop:}
	 {
	    \str_if_in:nnTF {#1}{\begin}
	    {\cexam_sep_tikz:p \c_empty_tl #1\c_empty_tl\scan_stop:}
	    {
	       \bool_set_true:N \cexam_nopic_bool
	       \bool_set_true:N \cexam_notab_bool
	       \tl_set:Nn \cexam_sep_txt_tl {#1}
	       \tl_set:Nn \cexam_sep_pictab_tl {}
	    }
	 }
	 \bool_if:NTF \cexam_nopic_bool
	 {\c_empty_tl}
	 {
	    \hbox_set:Nn \sep_temp_box {\cexam_sep_pictab_tl}
	    \dim_compare:nNnTF {\box_wd:N \sep_temp_box} < {5pt}
	    {
	       \msg_warning:nnxx{cexam}{picture}
	       {\int_use:N \cexam_number_int}{small}
	       \tl_set:Nn \cexam_sep_pictab_tl {\cexam_sep_nopic_tl}
	    }
	    {
	       \dim_compare:nNnTF {\box_wd:N \sep_temp_box} > {\linewidth}
	       {
		  \msg_warning:nnxx {cexam}{picture}
		  {\int_use:N \cexam_number_int}{wide}
		  \tl_set:Nn \cexam_sep_pictab_tl{\cexam_sep_nopic_tl}
	       }
	       {
		  \dim_compare:nNnTF {\box_ht:N \sep_temp_box} > {.5\linewidth}
		  {
		     \msg_warning:nnxx {cexam}{picture}
		     {\int_use:N \cexam_number_int}{high}
		     \tl_set:Nn \cexam_sep_pictab_tl{\cexam_sep_nopic_tl}
		  }
		  {\c_empty_tl}
	       }
	    }
	 }
      }
   }
}
%    \end{macrocode}
% \end{macro}
%
% \begin{macro}[added=2019-09-19]{\cexam_number_tag_tl,\cexam_number_tag_i_tl}
% \changes{v3.1.7}{2019/09/19}{新增程序}
% \changes{v3.2.2}{2019/02/16}{修改命令为tokenlist}
% 此二命令为题目编号,也可以修改用以生成例题模式
%    \begin{macrocode}
\tl_set:Nn \cexam_number_tag_tl{\int_use:N \cexam_number_int .}
\tl_set:Nn \cexam_number_tag_i_tl{}
%    \end{macrocode}
% \end{macro}
% 
% 
% \begin{macro}[added=2021-02-05]{\choice,\refa,\refb,\refc,\refd}
% \changes{v3.3.4}{2021/02/05}{新增命令}
% \changes{v3.3.4}{2021/02/06}{新增随机模式下选项引用命令}
% 此命令无实际作用,单纯在正文中输入选项定界。
%    \begin{macrocode}
\NewDocumentCommand {\choice}{}{} 
\NewDocumentCommand {\refa}{}{A}
\NewDocumentCommand {\refb}{}{B}
\NewDocumentCommand {\refc}{}{C}
\NewDocumentCommand {\refd}{}{D}
%    \end{macrocode}
% \end{macro}
% 
% 
% \begin{macro}[added=2021-02-05]{\choice_option_set_i:p}
% \changes{v3.3.4}{2021/02/05}{新增命令}
% 此命令用来定义选择题的四个选项。
%    \begin{macrocode}
%
\cs_new:Npn \choice_option_set_i:p #1[#2]#3\scan_stop:
{
   \str_case:nnTF {#2}
   {
      {A}{\tl_set:Nn \cho_opta_tl{#3}}
      {B}{\tl_set:Nn \cho_optb_tl{#3}}
      {C}{\tl_set:Nn \cho_optc_tl{#3}}
      {D}{\tl_set:Nn \cho_optd_tl{#3}}
   }
   {
      \str_if_in:nnTF {#1}{*}
      {\tl_put_right:No \choice_ans_tl{#2}\bool_set_true:N\cho_optstar_bool}
      {\c_empty_tl}
   }
   {
      \tl_clear:N \cho_opta_tl
      \tl_clear:N \cho_optb_tl
      \tl_clear:N \cho_optc_tl
      \tl_clear:N \cho_optd_tl
   }
}
%    \end{macrocode}
% \end{macro}
% 
% 
% \begin{macro}[added=2021-02-05]{\choice_randint_make:}
% \changes{v3.3.4}{2021/02/05}{新增命令}
% 生成选择题随机选项的四个随机数码,辅助生成四个随机选项。
% 
%    \begin{macrocode}
\cs_new:Nn \choice_randint_make:
{
   \int_set:Nn \choice_opta_int{\int_rand:n {4}}
   \int_set:Nn \choice_optb_int{\int_rand:n {4}}
   \int_do_while:nNnn {\choice_optb_int}={\choice_opta_int}
   {\int_set:Nn \choice_optb_int{\int_rand:n {4}}}
   \int_set:Nn \choice_optc_int{\int_rand:n {4}}
   \int_do_while:nNnn {\choice_optc_int}={\choice_opta_int}
   {
      \int_set:Nn \choice_optc_int{\int_rand:n {4}}
      \int_do_while:nNnn {\choice_optc_int}={\choice_optb_int}
      {\int_set:Nn \choice_optc_int{\int_rand:n {4}}}
   }
   \int_set:Nn \choice_optd_int{\int_rand:n {4}}
   \int_do_while:nNnn {\choice_optd_int}={\choice_opta_int}
   {
      \int_set:Nn \choice_optd_int{\int_rand:n {4}}
      \int_do_while:nNnn {\choice_optd_int}={\choice_optb_int}
      {
	 \int_set:Nn \choice_optd_int{\int_rand:n {4}}
	 \int_do_while:nNnn {\choice_optd_int}={\choice_optc_int}
	 {\int_set:Nn \choice_optd_int{\int_rand:n {4}}}
      }
   }
   \exp_args:NNx \int_set:Nn \choice_optabcd_int 
   {
      \int_use:N\choice_opta_int
      \int_use:N\choice_optb_int
      \int_use:N\choice_optc_int
      \int_use:N\choice_optd_int
   }
}
%    \end{macrocode}
% \end{macro}
% 
% 
% \begin{macro}[added=2021-02-06]{\choice_optref_set:nn}
% 生成选择题开启选项随机模式后,在编写解析时其选项也对应的随机同步变化。
%    \begin{macrocode}
\cs_new:Npn \choice_optref_set:nn #1#2
{
   \int_case:nn {#1}
   {
      {1}{\exp_args:Nx \RenewDocumentCommand {\use:c{ref#2}}{}{A}}
      {2}{\exp_args:Nx \RenewDocumentCommand {\use:c{ref#2}}{}{B}}
      {3}{\exp_args:Nx \RenewDocumentCommand {\use:c{ref#2}}{}{C}}
      {4}{\exp_args:Nx \RenewDocumentCommand {\use:c{ref#2}}{}{D}}
   }
}
%    \end{macrocode}
% \end{macro}
% 
% 
% \begin{macro}[added=2021-02-06]{\choice_optref_set:p}
% 
%    \begin{macrocode}
\cs_new:Npn \choice_optref_set:p #1#2#3#4\scan_stop:
{
   \choice_optref_set:nn{#1}{a}
   \choice_optref_set:nn{#2}{b}
   \choice_optref_set:nn{#3}{c}
   \choice_optref_set:nn{#4}{d}
}
%    \end{macrocode}
% \end{macro}
% 
% 
% \begin{macro}[added=2021-02-06]{\choice_optref_set:}
% 
%    \begin{macrocode}
\cs_new:Nn \choice_optref_set:
{
   \choice_optref_set:p
   \choice_opta_int\choice_optb_int\choice_optc_int\choice_optd_int
   \scan_stop:
}
%    \end{macrocode}
% \end{macro}
% 
% 
% \begin{macro}[added=2021-02-06]{\RandRefabcd}
% 设置四个引用选项。
% 
%    \begin{macrocode}
\NewDocumentCommand {\RandRefabcd}{m}{\choice_optref_set:p#1\scan_stop:}
%    \end{macrocode}
% \end{macro}
% 
% 
% \begin{macro}[added=2021-02-05]{\choice_option_set_ii:p}
% \changes{v3.3.4}{2021/02/05}{新增命令}
% 随机选项生成。
% 
%    \begin{macrocode}
\cs_new:Npn \choice_option_set_ii:p [#1]#2[#3]#4\scan_stop:
{
   \int_case:nn {#1}
   {
      {1}{\choice_option_set_i:p #2[A]#4\scan_stop:}
      {2}{\choice_option_set_i:p #2[B]#4\scan_stop:}
      {3}{\choice_option_set_i:p #2[C]#4\scan_stop:}
      {4}{\choice_option_set_i:p #2[D]#4\scan_stop:}
   }
}
%    \end{macrocode}
% \end{macro}
% 
% 
% \begin{macro}[added=2021-02-05]{\choice_ans_order:n}
% \changes{v3.3.4}{2021/02/05}{新增命令}
% 由于在随机选项排列中,答案的生成不再按照顺序来,这样子的答案比较不舒服,所以用它重排选择题答案。
% 
%    \begin{macrocode}
\cs_new:Npn \choice_ans_order:n #1
{
   \tl_clear:N \cexam_anspub_tl
   \str_if_in:nnTF {#1}{A}{\tl_put_right:Nn\cexam_anspub_tl{A}}{}
   \str_if_in:nnTF {#1}{B}{\tl_put_right:Nn\cexam_anspub_tl{B}}{}
   \str_if_in:nnTF {#1}{C}{\tl_put_right:Nn\cexam_anspub_tl{C}}{}
   \str_if_in:nnTF {#1}{D}{\tl_put_right:Nn\cexam_anspub_tl{D}}{}
}
%    \end{macrocode}
% \end{macro}
% 
% 
% \begin{macro}[added=2021-02-05]{\choice_option_set:nnnn}
% \changes{v3.3.4}{2021/02/05}{新增命令}
% \changes{v3.3.4}{2021/02/06}{使用布尔表达式优化实现过程}
% 选择题选项内容确定程序。
% 
%    \begin{macrocode}
\cs_new:Npn \choice_option_set:nnnn #1#2#3#4
{
   \bool_if:nTF {!\choice_oldopt_bool && \cho_optrand_bool}
   {
      \choice_randint_make:
      \choice_option_set_ii:p [\choice_opta_int]\c_empty_tl #1\scan_stop:
      \choice_option_set_ii:p [\choice_optb_int]\c_empty_tl #2\scan_stop:
      \choice_option_set_ii:p [\choice_optc_int]\c_empty_tl #3\scan_stop:
      \choice_option_set_ii:p [\choice_optd_int]\c_empty_tl #4\scan_stop:
   }
   {
      \choice_option_set_i:p \c_empty_tl #1\scan_stop:
      \choice_option_set_i:p \c_empty_tl #2\scan_stop:
      \choice_option_set_i:p \c_empty_tl #3\scan_stop:
      \choice_option_set_i:p \c_empty_tl #4\scan_stop:
   }
   \choice_optref_set:
   \exp_args:Nx\choice_ans_order:n {\tl_use:N\choice_ans_tl}
}
%    \end{macrocode}
% \end{macro}
% 
% 
% \begin{macro}[added=2019-08-29]{\choice_type_i:p}
% \changes{v3.1.2}{2019/08/29}{新增选择题排版程序}
% \changes{v3.1.3}{2019/09/03}{更名}
% \changes{v3.1.4}{2019/09/10}{修复环境排题时图片下标格式错误}
% \changes{v3.1.4}{2019/09/10}{加入选择题空白括号}
% \changes{v3.1.7}{2019/09/19}{增强题号功能,配合生成例题模式}
% \changes{v3.1.9}{2019/09/27}{增加图片超半个行宽时的排版}
% \changes{v3.2.1}{2019/10/20}{规范了选项间隔}
% \changes{v3.2.7}{2020/07/12}{修改了题号命令为三个参量}
% \changes{v3.3.4}{2021/02/05}{修改选项间隔符号为\cs{option}及选项处理}
%
%    \begin{macrocode}
\cs_new:Npn \choice_type_i:p 
#1.#2 \choice #3\choice #4\choice #5\choice #6\scan_stop:
{
%    \end{macrocode}
% 设置题号决定的缩进
%    \begin{macrocode}
   \bool_set_false:N \cho_optstar_bool 
   \tl_clear:N\choice_ans_tl
   \choice_option_set:nnnn {#3}{#4}{#5}{#6} 
   \int_gadd:Nn \cexam_number_int {1}
   \hbox_set:Nn \cexam_number_box {\cexam_number_tag_tl} 
   \dim_set:Nn \cexam_indent_dim{\box_wd:N \cexam_number_box}
   \dim_add:Nn \cexam_indent_dim{\cexam_numtxt_skip}
   \dim_set:Nn \cexam_indent_i_dim {\cexam_indent_dim}
   \dim_add:Nn \cexam_indent_i_dim {\cho_hat_wd_dim} 
%    \end{macrocode}
% 分离图片和文字
%    \begin{macrocode}
   \cexam_sep_pictxt:n 
   {
      \cexam_ind_hat:nnn
      {\cexam_indent_dim}{\cexam_number_tag_tl}{\cexam_number_tag_i_tl}
      #2 
      \hfill\mbox{(\quad)}
   } 
   \dim_set:Nn \cho_optwd_dim {\linewidth}
   \dim_sub:Nn \cho_optwd_dim {\cexam_indent_dim}
   \dim_sub:Nn \cho_optwd_dim {\cexam_pictxt_skip} 
%    \end{macrocode}
%  获得图片宽高
%    \begin{macrocode}
   \bool_if:NTF \cexam_nopic_bool
   {\c_empty_tl}
   {
      \hbox_set:Nn \cho_optpic_box{\cexam_sep_pictab_tl}
      \dim_set:Nn {\cho_optpic_wd_dim}{\box_wd:N \cho_optpic_box}
      \dim_set:Nn {\cho_optpic_ht_dim}{\box_ht:N \cho_optpic_box}
      \dim_add:Nn {\cho_optpic_ht_dim}{\box_dp:N \cho_optpic_box}
%    \end{macrocode}
% 加入*号图片不加标号
%    \begin{macrocode}
      \bool_set_true:N \cexam_fmt_bool
      \str_if_in:nnTF {#1}{*}
      {\bool_set_false:N \cexam_fmt_bool}
      { 
	 \dim_add:Nn \cho_optpic_ht_dim {\baselineskip}
      }
   }
%    \end{macrocode}
% 据图片给出排版依据的高度
%    \begin{macrocode}
   \bool_if:NTF \cexam_nopic_bool
   {
      \cexam_fmt_opt_cho:nnnn 
      {\cho_opta_tl}{\cho_optb_tl}{\cho_optc_tl}{\cho_optd_tl} 
   }
   { 
%    \end{macrocode}
% 测试图片宽度如果大于半个行宽,则置零判断高度
%    \begin{macrocode}
      \dim_set:Nn \cexam_picwd_limit {.5\linewidth}
      \dim_sub:Nn \cexam_picwd_limit {.5\cexam_indent_dim}
      \dim_compare:nNnTF {\cho_optpic_wd_dim} > {\cexam_picwd_limit} 
      {
	 \dim_set:Nn \cho_optpic_hti_dim {0pt} 
	 \cexam_fmt_opt_cho:nnnn 
	 {\cho_opta_tl}{\cho_optb_tl}{\cho_optc_tl}{\cho_optd_tl} 
      }
%    \end{macrocode}
% 当图宽小于半个行宽,则取得文本高度
%    \begin{macrocode}
      {
	 \dim_sub:Nn \cho_optwd_dim {\cho_optpic_wd_dim}
	 \cexam_fmt_opt_cho:nnnn 
	 {\cho_opta_tl}{\cho_optb_tl}{\cho_optc_tl}{\cho_optd_tl} 
	 \get_par_ht:nnn  
	 {\cho_optpic_hti_dim}
	 {\cho_optwd_dim}
	 {\cexam_sep_txt_tl}
      }
   }
%    \end{macrocode}
% 准备排版
%    \begin{macrocode}
   \bool_if:NTF \cexam_nopic_bool
%    \end{macrocode}
% 排版无图模式
%    \begin{macrocode}
   { 
      \bool_if:NTF \cho_opt_maxed_bool
      {\c_empty_tl}
      {
	 \dim_set:Nn \cexam_indent_i_dim {\cexam_indent_dim}
      }
      \cexam_type_v:nnnnn
      {\cexam_indent_dim}{0pt}
      {\cexam_indent_i_dim}{0pt}
      {\cexam_sep_txt_tl}
      \newline
      \hbox_unpack:N \cexam_option_box
   }
%    \end{macrocode}
% 排版含图模式
%    \begin{macrocode}
   {
      \dim_compare:nNnTF {\cho_optpic_hti_dim} > {\cho_optpic_ht_dim}
      {
%    \end{macrocode}
% 给选项宽付值
%    \begin{macrocode}
	 \dim_add:Nn \cho_optwd_dim {\cho_optpic_wd_dim}
	 \cexam_fmt_opt_cho:nnnn 
	 {\cho_opta_tl}{\cho_optb_tl}{\cho_optc_tl}{\cho_optd_tl} 
%    \end{macrocode}
% 开始排版
% 四个选项缩进排版,三级缩进为2\cs{cexam_ccwd_dim}
% 四个选项无缩进排版,三级缩进为\cs{cexam_ccwd_dim}
%    \begin{macrocode}
	 \bool_if:NTF \cho_opt_maxed_bool
	 {\c_empty_tl}
	 {\dim_set:Nn \cexam_indent_i_dim {\cexam_indent_dim}}
	 \cexam_type_ii:nnnnnnnnn
	 {r}{\cexam_sep_pictab_tl}
	 {\cexam_indent_dim}{\cexam_pictxt_skip} 
	 {\cexam_indent_dim}{0pt}
	 {\cexam_indent_i_dim}{0pt}
	 {\cexam_sep_txt_tl}
	 \newline
	 \hbox_unpack:N \cexam_option_box
      }
      {
%    \end{macrocode}
% 判断排版格式
%    \begin{macrocode}
	 \dim_sub:Nn \cho_optpic_ht_dim {\cho_optpic_hti_dim}
	 \bool_if:NTF \cho_opt_maxed_bool
	 { 
%    \end{macrocode}
% 如果图宽大于半个行宽则不需要选项高度直接进入\cs{cexam_type_iv:nnnnnn}排版,若图宽小于半个行宽,则获得选项的高度,以进一步判断排版模式
%    \begin{macrocode}
	    \dim_compare:nNnTF {\cho_optpic_wd_dim} > {\cexam_picwd_limit} 
	    {\c_empty_tl}
	    {
	       \get_par_ht:nnn
	       {\cho_optpic_hti_dim}
	       {\cho_optwd_dim}
	       {\hbox_unpack:N \cexam_option_box}
	    }
%    \end{macrocode}
% 加入定义,以防止进入测定行数程序时的第一次展开
%    \begin{macrocode}
	    \dim_compare:nNnTF
	    {\cho_optpic_ht_dim}< {\cho_optpic_hti_dim}
	    {
	       \cs_set:Nn \cexam_seped_txt_i:
	       {\hbox_unpack:N \cexam_option_box}
	       \dim_add:Nn \cho_optwd_dim {\cho_optpic_wd_dim}
	       \cexam_type_iv:nnnnnnnn
	       {r}{\cexam_sep_pictab_tl}
	       {\cexam_indent_dim}{\cexam_pictxt_skip} 
	       {\cexam_indent_i_dim}{0pt}
	       {\cexam_sep_txt_tl}{\cexam_seped_txt_i:}
	    }
	    {
	       \dim_add:Nn \cho_optwd_dim {\cho_optpic_wd_dim}
	       \cexam_fmt_opt_cho:nnnn 
	       {\cho_opta_tl}{\cho_optb_tl}{\cho_optc_tl}{\cho_optd_tl} 
	       \cexam_type_iii:nnnnnnn 
	       {c}{\cexam_sep_pictab_tl}
	       {\cexam_indent_dim}{0pt}
	       {\cexam_indent_i_dim}{0pt}
	       {\cexam_sep_txt_tl}
	       \newline
	       \hbox_unpack:N \cexam_option_box
	    }
	 }
	 {
%    \end{macrocode}
% 进入图片居于题干和选项之间居中排版
%    \begin{macrocode}
	    \dim_compare:nNnTF {\cho_optpic_wd_dim} > {\cexam_picwd_limit} 
	    {
	       \cexam_fmt_opt_cho:nnnn 
	       {\cho_opta_tl}{\cho_optb_tl}{\cho_optc_tl}{\cho_optd_tl} 
	       \cexam_type_iii:nnnnnnn
	       {c}{\cexam_sep_pictab_tl}
	       {\cexam_indent_dim}{0pt}
	       {\cexam_indent_dim}{0pt}
	       {\cexam_sep_txt_tl}
	       \newline
	       \hbox_unpack:N \cexam_option_box
	    }
%    \end{macrocode}
% 进入图片居于题干和选项之间居中排版
%    \begin{macrocode}
	    {
	       \dim_compare:nNnTF {\cho_optpic_ht_dim} > {2\baselineskip}
	       {
		  \dim_add:Nn \cho_optwd_dim {\cho_optpic_wd_dim}
		  \cexam_fmt_opt_cho:nnnn 
		  {\cho_opta_tl}{\cho_optb_tl}{\cho_optc_tl}{\cho_optd_tl}
		  \cexam_type_iii:nnnnnnn
		  {c}{\cexam_sep_pictab_tl}
		  {\cexam_indent_dim}{0pt}
		  {\cexam_indent_dim}{0pt}
		  {\cexam_sep_txt_tl}
		  \newline
		  \hbox_unpack:N \cexam_option_box
	       }
	       {
%    \end{macrocode}
% 进入选项无缩进排列
%    \begin{macrocode}
		  \dim_add:Nn \cho_optpic_wd_dim{\cexam_pictxt_skip} 
		  \cexam_type_i:nnnnnnn
		  {r}{\cexam_sep_pictab_tl}
		  {\cexam_indent_dim}{\cexam_pictxt_skip} 
		  {\cexam_indent_dim}{\cho_optpic_wd_dim}
		  {\cexam_sep_txt_tl}
		  \newline
		  \hbox_unpack:N \cexam_option_box
	       }
	    }
	 }
      }
   }
}
%    \end{macrocode}
% \end{macro}
% 
% 
% \begin{macro}[added=2021-02-06]{\choice_option_total:p}
% \changes{v3.3.5}{2021/02/06}{新增命令}
% 此命令用以取消之前版本选择题选项输入格式后对选择题结构的探测。
% 
%    \begin{macrocode}
\cs_new:Npn \choice_option_total:p #1\choice#2\scan_stop:
{
   \str_if_in:nnTF {#2}{\choice}
   {
      \choice_option_total:p\c_empty_tl#2\c_empty_tl\scan_stop:
      \int_add:Nn \choice_option_int {1}
   }
   {\c_empty_tl}
}
%    \end{macrocode}
% \end{macro}
%
% \begin{macro}[added=2019-09-03]{\choice_warning:}
% \changes{v3.1.3}{2019/09/03}{新增命令}
% \changes{v3.3.5}{2021/02/06}{修改了提示内容}
% 此程序用来探测选择题结构,如果选择题没有四个选项,则不排版而输出红色警告文字.
%
%    \begin{macrocode}
\cs_new:Nn \choice_warning: 
{
  \color_group_begin:
  \color_select:n {red} Choice~option~lost.
  \color_group_end:
}
%    \end{macrocode}
% \end{macro}
%
% \begin{macro}[added=2019-09-03]{\choice_type:p}
% \changes{v3.1.3}{2019/09/03}{重新定义选择题排版程序}
% \changes{v3.3.4}{2021/02/05}{加入了随机选项排列功能,同时兼容之前的选项输入格式}
% \changes{v3.3.5}{2021/02/06}{去除了对旧输入格式的兼容}
% \changes{v3.3.7}{2022/03/26}{增加了选项为图片时的排版}
% 选择题排版程序.
%
%    \begin{macrocode}
\cs_new:Npn \choice_type:p #1.#2 \par
{
   \str_if_in:nnTF {#2}{\choice[P]}
   {\choice_type_ii:p #1.#2 \scan_stop:}
   {
      \str_if_in:nnTF {#2}{\choice}
      {
	 \int_set:Nn \choice_option_int {1}
	 \choice_option_total:p #2\scan_stop:
	 \int_compare:nNnTF {\choice_option_int} < 4
	 {\choice_warning:}
	 {
	    \bool_set_false:N \choice_oldopt_bool
	    \choice_type_i:p #1.#2 \scan_stop:
	 }
      }
      {\choice_warning:}
   }
   \par
}
\cs_new:Npn \choice_type_ii:p #1.#2\choice[P]#3 \scan_stop:
{
   \blank_type:p #1.#2\hfill\mbox{(\quad)}\par
   s.\hspace*{\cexam_indent_dim}\hfill #3\hfill\quad
}
%    \end{macrocode}
% \end{macro}
% 
% \subsection{填空题的排版}
%
% \begin{macro}[added=2019-09-03]{\blank_type_i:p}
% \changes{v3.1.3}{2019/09/03}{新增命令}
% \changes{v3.1.4}{2019/09/10}{修复环境排题时图片下标格式错误}
% \changes{v3.1.9}{2019/09/27}{增加宽图排版}
% \changes{v3.3.2}{2020/09/29}{修复题干高于图高时排版时调用排版模式的错误}
%
% 填空题初级排版程序,由于此程序在答案,解析,判断等题中有重复应用,所以将这一部分共同的排版程序,提取出来.
% 
% 在填空题类型的排版中,由于不涉及选项的排版,所以不应当调用三级缩进排版,而三级缩进排版仅当在选择题中和计算题中出现小问的时候调用。这一问题遇到的不多,在2020年9月排版《高中物理讲义》时初次遇到这个情况,于是在2020年9月29日进行了模式调用的讨论,进而以为调用了二级缩进来排图。
%
%    \begin{macrocode}
\cs_new:Npn \blank_type_i:p #1.#2 \par
{
%    \end{macrocode}
% 分离图文
%    \begin{macrocode}
   \cexam_sep_pictxt:n {#2}
%    \end{macrocode}
% 判断图片格式化
%    \begin{macrocode}
   \bool_if:NTF \cexam_nopic_bool
   {\c_empty_tl}
   {
      \hbox_set:Nn \cho_optpic_box{\cexam_sep_pictab_tl}
      \dim_set:Nn {\cho_optpic_wd_dim}{\box_wd:N \cho_optpic_box}
      \dim_set:Nn {\cho_optpic_ht_dim}{\box_ht:N \cho_optpic_box}
      \dim_add:Nn {\cho_optpic_ht_dim}{\box_dp:N \cho_optpic_box}
      \bool_set_true:N \cexam_fmt_bool
      \str_if_in:nnTF {#1}{*}
      {\bool_set_false:N \cexam_fmt_bool}
      {\dim_add:Nn \cho_optpic_ht_dim {\baselineskip}}
   }
%    \end{macrocode}
% 基础排版
%    \begin{macrocode}
   \bool_if:NTF \cexam_nopic_bool
   {
%    \end{macrocode}
% 无图排版
%    \begin{macrocode}
      \cexam_type_v:nnnnn
      {\cexam_indent_dim}{0pt}
      {\cexam_indent_i_dim}{0pt}
      {\cexam_sep_txt_tl}
   }
   {
%    \end{macrocode}
% 图宽大于半个行宽时,直接以\cs{cexam_type_iv:nnnnnnn} 排版
%    \begin{macrocode}
      \dim_set:Nn \cexam_picwd_limit {.5\linewidth}
      \dim_sub:Nn \cexam_picwd_limit {.5\cexam_indent_dim}
      \dim_compare:nNnTF {\cho_optpic_wd_dim} > {\cexam_picwd_limit} 
      {
	 \cexam_type_iii:nnnnnnn
	 {c}{\cexam_sep_pictab_tl}
	 {\cexam_indent_dim}{0pt}
	 {\cexam_indent_dim}{0pt}
	 {\cexam_sep_txt_tl}
      }
%    \end{macrocode}
% 当图宽小于半个行宽时,获得文本以行宽减图宽排版时的测量高度
%    \begin{macrocode}
      {
	 \dim_set:Nn \cho_optwd_dim {\linewidth}
	 \dim_sub:Nn \cho_optwd_dim {\cexam_indent_dim}
	 \dim_sub:Nn \cho_optwd_dim {\cexam_pictxt_skip}
	 \dim_sub:Nn \cho_optwd_dim {\cho_optpic_wd_dim}
	 \get_par_ht:nnn
	 {\cho_optpic_hti_dim}
	 {\cho_optwd_dim}
	 {\cexam_sep_txt_tl}
%    \end{macrocode}
% 决定排版
%    \begin{macrocode}
	 \dim_compare:nNnTF {\cho_optpic_hti_dim} > {\cho_optpic_ht_dim}
	 {
	    \cexam_type_i:nnnnnnn
	    {r}{\cexam_sep_pictab_tl}
	    {\cexam_indent_dim}{\cexam_pictxt_skip}
	    {\cexam_indent_dim}{0pt}
	    {\cexam_sep_txt_tl}
	 }
	 {
	    \dim_sub:Nn \cho_optpic_ht_dim {\cho_optpic_hti_dim}
	    \dim_compare:nNnTF
	    {\cho_optpic_ht_dim} < {\baselineskip}
	    {
	       \cexam_type_i:nnnnnnn
	       {r}{\cexam_sep_pictab_tl}
	       {\cexam_indent_dim}{\cexam_pictxt_skip}
	       {\cexam_indent_dim}{\cho_optpic_wd_dim}
	       {
		  \cexam_sep_txt_tl
	       }
	       \vspace{\cho_optpic_ht_dim}
	    }
	    {
	       \cexam_type_iii:nnnnnnn
	       {c}{\cexam_sep_pictab_tl}
	       {\cexam_indent_dim}{0pt}
	       {\cexam_indent_dim}{0pt}
	       {\cexam_sep_txt_tl}
	    }
	 }
      }
   }
   \par
}
%    \end{macrocode}
% \end{macro}
%
% \begin{macro}[added=2019-09-03]{\blank_type:p}
% \changes{v3.1.3}{2019/09/03}{新增命令}
% \changes{v3.2.7}{2020/7/12}{修改了题号命令为三个参量}
% 填空题排版程序.
%
%    \begin{macrocode}
\cs_new:Npn \blank_type:p #1.#2 \par
{
  \int_gadd:Nn \cexam_number_int {1}
  \hbox_set:Nn \cexam_number_box {\cexam_number_tag_tl} 
  \dim_set:Nn \cexam_indent_dim{\box_wd:N \cexam_number_box}
  \dim_add:Nn \cexam_indent_dim{\cexam_numtxt_skip}
  \blank_type_i:p #1.
  \cexam_ind_hat:nnn
  {\cexam_indent_dim}{\cexam_number_tag_tl}{\cexam_number_tag_i_tl}
  #2 
  \par
}
%    \end{macrocode}
% \end{macro}
% 
% \begin{macro}[added=2019-09-03]{\cexam_anspub_tl,\cexam_quad_tl,\cexam_blan:n}
% 此三个命令分别旧时存储填空题答案命令,不可展开空白,答案积累和空白生成的作用.
% 此处的空白生成程序我考虑了很久,当直接代之以与答案差不多长的线段盒子时这样构成的下划线不能自动断行,但是加入一个\cs{quad}后可以自动断行,而此时各线段中又多了若干空白,所以在空白后再追加一个负宽度盒子,以抵消此空白.我们就实现了下划线自动断行的功能.
% \changes{v3.2.2}{2019/02/16}{规范填空题命令,修改为字符串命令}
%
%    \begin{macrocode}
\cs_new:Npn \cexam_blank:n #1
{
  \tl_put_right:No \cexam_anspub_tl{~#1\quad}
  \hbox_set:Nn \blank_wd_box {#1}
  \dim_set:Nn \blank_wd_dim {\box_wd:N \blank_wd_box}
  \dim_add:Nn \blank_wd_dim {2\cexam_ccwd_dim}
  \hspace{3pt}
  \dim_while_do:nNnn
  {\blank_wd_dim} > {0pt}
  { 
    \dim_sub:Nn \blank_wd_dim {\cexam_ccwd_dim}
    \cexam_quad_tl 
    \hspace{-13pt}
    \quad
  }
  \hspace{6pt}
}
%    \end{macrocode}
% \end{macro}
%
% \subsection{判断题的排版}
%
% \begin{macro}[added=2019-09-03]{\judge_type:p}
% \changes{v3.1.3}{2019/09/03}{新增命令}
% 判断题排版程序.
%    \begin{macrocode}
\cs_new:Npn \judge_type:p #1.#2\par
{\blank_type:p #1.#2\hfill\mbox{(\quad)}\par}
%    \end{macrocode}
% \end{macro}
%
% \subsection{计算题的排版}
%
% \begin{macro}[added=2019-09-03]{\cexam_qitem:}
% \changes{v3.1.3}{2019/09/03}{新增命令}
% \changes{v3.2.5}{2020/03/27}{使用\cs{cexam_ind_hat:nnnn}改写,删除了原来的\cs{parbox}}
% \changes{v3.2.7}{2020/7/12}{修改了题号命令为三个参量}
% 类比列表环境中的\cs{item},此处为问题(question)的小问,所以在\cs{item}前加以一个q以示区别.
%    \begin{macrocode}
\cs_new:Nn \cexam_qitem:
{
   \cexam_ind_hat:nnn
   {1.4\cexam_ccwd_dim}{(\int_use:N \cexam_qitem_int)}{}
}
%    \end{macrocode}
% \end{macro}
%
%
% \begin{macro}[added=2019-09-03]{\calculate_type_i:p}
% \changes{v3.1.3}{2019/09/03}{新增命令}
% \changes{v3.1.4}{2019/09/10}{修复环境排版时图片下标格式错误}
% \changes{v3.1.9}{2019/09/27}{增加图宽大于半个行宽的排版}
% \changes{v3.1.9}{2019/10/08}{置零计数器改为标准的\cs{int_zero:N}}
% \changes{v3.2.7}{2020/07/12}{修改了题号命令为三个参量}
% \changes{v3.3.6}{2021/02/25}{由于重定义了\cs{qitem},所以修改了参数变量为三个}
% 排版计算题中含有若干小问的情况.
%
%    \begin{macrocode}
\cs_new:Npn \calculate_type_i:p #1.#2\qitem#3\par
{
%    \end{macrocode}
% 题号处理
%    \begin{macrocode}
   \int_gadd:Nn \cexam_number_int {1}
   \hbox_set:Nn \cexam_number_box {\cexam_number_tag_tl} 
   \dim_set:Nn \cexam_indent_dim{\box_wd:N \cexam_number_box}
   \dim_add:Nn \cexam_indent_dim{\cexam_numtxt_skip}
   \dim_set:Nn \cexam_indent_i_dim {\cexam_indent_dim}
   \dim_add:Nn \cexam_indent_i_dim {1.43\cexam_ccwd_dim}
   \int_zero:N \cexam_qitem_int
%    \end{macrocode}
% 分离图文
%    \begin{macrocode}
   \cexam_sep_pictxt:n
   {
      \cexam_ind_hat:nnn
      {\cexam_indent_dim}{\cexam_number_tag_tl}{\cexam_number_tag_i_tl}#2 
   }
%    \end{macrocode}
% 判断图片格式化
%    \begin{macrocode}
   \bool_if:NTF \cexam_nopic_bool
   {\c_empty_tl}
   {
      \hbox_set:Nn \cho_optpic_box{\cexam_sep_pictab_tl}
      \dim_set:Nn {\cho_optpic_wd_dim}{\box_wd:N \cho_optpic_box}
      \dim_set:Nn {\cho_optpic_ht_dim}{\box_ht:N \cho_optpic_box}
      \dim_add:Nn {\cho_optpic_ht_dim}{\box_dp:N \cho_optpic_box}
      \bool_set_true:N \cexam_fmt_bool
      \str_if_in:nnTF {#1}{*}
      {\bool_set_false:N \cexam_fmt_bool}
      {\dim_add:Nn \cho_optpic_ht_dim {\baselineskip}}
   }
%    \end{macrocode}
% 定义选项盒子
%    \begin{macrocode}
   \hbox_set:Nn \cexam_option_box {\qitem#3}
%    \end{macrocode}
% 排版
%    \begin{macrocode}
   \bool_if:NTF \cexam_nopic_bool
   {
      \cexam_type_v:nnnnn
      {\cexam_indent_dim}{0pt}
      {\cexam_indent_i_dim}{0pt}
      {\cexam_sep_txt_tl}
      \newline
      \hbox_unpack:N \cexam_option_box
   }
   {
      \dim_set:Nn \cexam_picwd_limit {.5\linewidth}
      \dim_sub:Nn \cexam_picwd_limit {.5\cexam_indent_dim}
      \dim_compare:nNnTF {\cho_optpic_wd_dim} > {\cexam_picwd_limit} 
      {
	 \cexam_type_iii:nnnnnnn
	 {c}{\cexam_sep_pictab_tl}
	 {\cexam_indent_dim}{0pt}
	 {\cexam_indent_dim}{0pt}
	 {\cexam_sep_txt_tl}
	 \newline
	 \hbox_unpack:N \cexam_option_box
      }
      {
	 \dim_set:Nn \cho_optwd_dim {\linewidth}
	 \dim_sub:Nn \cho_optwd_dim {\cexam_indent_dim}
	 \dim_sub:Nn \cho_optwd_dim {\cexam_pictxt_skip}
	 \dim_sub:Nn \cho_optwd_dim {\cho_optpic_wd_dim}
	 \get_par_ht:nnn
	 {\cho_optpic_hti_dim}
	 {\cho_optwd_dim}
	 {\cexam_sep_txt_tl}
	 \dim_compare:nNnTF 
	 {\cho_optpic_hti_dim}>{\cho_optpic_ht_dim}
	 {
	    \cexam_type_ii:nnnnnnnnn
	    {r}{\cexam_sep_pictab_tl}
	    {\cexam_indent_dim}{\cexam_pictxt_skip}
	    {\cexam_indent_dim}{0pt}
	    {\cexam_indent_i_dim}{0pt}
	    {\cexam_sep_txt_tl}
	    \newline
	    \hbox_unpack:N \cexam_option_box
	 }
	 {
	    \cs_set:Nn \cexam_seped_txt_i:
	    {\hbox_unpack:N \cexam_option_box}
	    \cexam_type_iv:nnnnnnnn
	    {r}{\cexam_sep_pictab_tl}
	    {\cexam_indent_dim}{\cexam_pictxt_skip}
	    {\cexam_indent_i_dim}{0pt}
	    {\cexam_sep_txt_tl}{\cexam_seped_txt_i:}
	 }
      }
   }
   \par
}
%    \end{macrocode}
% \end{macro}
%
%
% \begin{macro}[added=2019-09-03]{\calculate_type:p}
% \changes{v3.1.3}{2019/09/03}{新增命令}
% 排版计算题.
%
%    \begin{macrocode}
\cs_new:Npn \calculate_type:p #1.#2 \par
{
  \str_if_in:nnTF {#2}{\qitem}
  {\calculate_type_i:p #1.#2\par}
  {\blank_type:p #1.#2\par}
}
%    \end{macrocode}
% \end{macro}
%
% \subsection{答案和解析}
%
% \begin{macro}[added=2019-09-03]{\ans_tag_tl,\ana_tag_tl,\ans_tag_i_tl,\ana_tag_i_tl}
% \changes{v3.1.3}{2019/09/03}{新增命令}
% \changes{v3.1.6}{2019/09/18}{新增答案文件中的标签命令}
% \changes{v3.2.2}{2019/02/16}{修改命令为tokenlist}
% \changes{v3.2.8}{2020/07/14}{增加证明题标签}
% 答案和解析的标签,在v3.2.8 版中增加证明题的标签。
%    \begin{macrocode}
\tl_set:Nn \ans_tag_tl{{\bf \makebox[0pt][r]{答}案}\hspace{5pt}}
\tl_set:Nn \ana_tag_tl{{\bf \makebox[0pt][r]{解}析}\hspace{5pt}}
\tl_set:Nn \prf_tag_tl{{\bf \makebox[0pt][r]{证}明}\hspace{5pt}}
\tl_set:Nn \ans_tag_i_tl{{\heiti 答案}}
\tl_set:Nn \ana_tag_i_tl{{\heiti 解析}}
\tl_set:Nn \prf_tag_i_tl{{\heiti 证明}}
%    \end{macrocode}
% \end{macro}
%
% \begin{macro}[added=2019-09-03]{\answer_type:p,\analysis_type:p}
% \changes{v3.1.3}{2019/09/03}{新增命令}
%  答案和解析的排版.
%
%    \begin{macrocode}
\cs_new:Npn \answer_type:p #1.#2\par
{
  \dim_set:Nn \cexam_indent_dim{\cexam_ccwd_dim}
  \blank_type_i:p #1.\ans_tag_tl#2\par
}
\cs_new:Npn \analysis_type:p #1.#2\par
{
   \bool_if:nTF \answer_student_bool
   {\vspace{-\baselineskip}\par}
   {
      \str_if_in:nnTF {#1}{ee}
      {\blank_type_i:p #1.#2\par}
      {
	 \dim_set:Nn \cexam_indent_dim{\cexam_ccwd_dim}
	 \blank_type_i:p #1.\ana_tag_tl#2\par
      }
   }
}
%    \end{macrocode}
% \end{macro}
%
% \subsection{学生模式答案写出}
% 
% 
% \begin{macro}[added=2021-02-06]{\cexam_optrand_iow:n}
% \changes{v3.3.4}{2021/02/06}{新增命令}
% 用来写出学生模式随机选项模式时写出带选项引用的解析。
%    \begin{macrocode}
\cs_new:Npn \cexam_optrand_iow:n #1 
{\iow_shipout:Nn \answer_write {\RandRefabcd{#1}}}
%    \end{macrocode}
% \end{macro}
%
% \begin{macro}[added=2019-09-18]{\cexam_answer_iow:p}
% \changes{v3.1.6}{2019/09/18}{新增命令}
% \changes{v3.2.2}{2019/02/16}{修改答案写出填空题命令}
%  答案输出模块
%    \begin{macrocode}
\cs_new:Npn \cexam_answer_iow:p #1.#2\scan_stop:
{
   \bool_if:NTF \answer_student_bool
   {
      \str_if_in:nnTF {#1}{a}
      {
	 \iow_shipout:Nx \answer_write
	 {\int_use:N \cexam_number_int .\ans_tag_i_tl}
	 \str_if_in:nnTF {#2}{*}
	 {
	    \iow_shipout:Nx \answer_write {\cexam_anspub_tl}
	    \bool_if:nTF {\cexam_choice_bool && \cho_optrand_bool}
	    {\exp_args:Nx\cexam_optrand_iow:n{\int_use:N\choice_optabcd_int}}
	    {\c_empty_tl}
	 }
	 {\iow_shipout:Nn \answer_write {#2}}
	 \iow_shipout:Nn \answer_write {}
      }
      {
	 \str_if_in:nnTF {#1}{ee}
	 {
	    \iow_shipout:Nn \answer_write {ee.#2}
	    \iow_shipout:Nn \answer_write {}
	 }
	 {
	    \str_if_in:nnTF {#1}{e}
	    {
	       \iow_shipout:Nx \answer_write {ee.\ana_tag_i_tl}
	       \iow_shipout:Nn \answer_write {#2}
	       \iow_shipout:Nn \answer_write {}
	    }
	    {\c_empty_tl}
	 }
      }
   }
   {\c_empty_tl}
}
%    \end{macrocode}
% \end{macro}
%
%
% \begin{macro}[added=2019-09-18]{\cexam_answer_add:p}
% \changes{v3.1.6}{2019/09/18}{新增命令}
%  用来添加章节及环境的写出操作
%
%    \begin{macrocode}
\cs_new:Npn \cexam_answer_add:p #1\scan_stop:
{
   \bool_if:nTF \answer_student_bool
   {
      \iow_shipout:Nx \answer_write {\exp_not:N#1}
      \iow_shipout:Nn \answer_write {}
   }
   {\c_empty_tl}
}
%    \end{macrocode}
% \end{macro}
% 
% \subsection{目录的设置}
% 
% 
% 
% \begin{macro}[added=2019-10-11]{\cexam_table_bool}
% \changes{v3.1.9}{2019/10/11}{新增布尔值,修复目录错误}
% 
% 由于在使用目录时不应当打开学生答案单独写出功能,所以需要修改\cs{tableofcontents}命令,以保证目录的正常使用。
%
%    \begin{macrocode}
\bool_new:N \cexam_table_bool
\cs_if_exist:NTF \tableofcontents
{
   \bool_if:NTF \answer_student_bool 
   {
      \tex_let:D \cexam_table_contents:n \tableofcontents
      \tex_def:D \tableofcontents
      {
	 \bool_set_false:N \answer_student_bool
	 \cexam_table_contents:n
	 \bool_set_true:N \answer_student_bool
      }
   }
   {\c_empty_tl}
}{\c_empty_tl}
%    \end{macrocode}
% \end{macro}
% 
% \subsection{章节命令加入答案写出}
% 
% \begin{macro}[added=2019-09-18]{\@chapter,\@schapter,\@sect,\@sset}
% \changes{v3.1.6}{2019/09/18}{新增命令}
% \changes{v3.1.7}{2019/09/19}{加入例题计数器随章计数器置零}
% \changes{v3.1.8}{2019/09/22}{追加重定义章节命令时的检测,适应不同文档类}
% \changes{v3.1.9}{2019/10/08}{置零计数器改用标准的\cs{int_zero:N}}
% \changes{v3.2.2}{2019/02/16}{修复题号不置零错误}
% \changes{v3.3.3}{2020/12/04}{初步解决\cs{@sect}的修改导致与hyperref宏包的冲突}
% 在章节命令的基础上追加了写出答案命令
% 
%    \begin{macrocode}
\cs_if_exist:NTF \@chapter
{
   \tex_let:D \cexam_chapter:n \@chapter
   \tex_def:D \@chapter[#1]#2{
      \cexam_chapter:n [#1]{#2}
      \int_gzero:N \example_number_int 
      \int_zero:N \cexam_number_int 
      \cexam_answer_add:p \chapter{#2(答案)}\scan_stop:
   }
}
{\c_empty_tl}
\cs_if_exist:NTF \@schapter
{
   \tex_let:D \cexam_schapter:n \@schapter
   \tex_def:D \@schapter#1{
      \cexam_schapter:n {#1}
      \int_gzero:N \example_number_int 
      \int_zero:N \cexam_number_int 
      \cexam_answer_add:p \chapter*{#1(答案)}\scan_stop:
   }
}
{\c_empty_tl}
\cs_if_exist:NTF\@sect
{
   \tex_let:D \cexam_sect:n \@sect
   \tex_def:D \@sect#1#2#3#4#5#6[#7]#8{
      \cexam_sect:n {#1}{#2}{#3}{#4}{#5}{#6}[{#7}]{#8}
      \str_if_in:nnTF {#1}{subsub}
      {\cexam_answer_add:p \subsubsection{#8}\scan_stop:}
      {
	 \str_if_in:nnTF {#1}{sub}
	 {\cexam_answer_add:p \subsection{#8}\scan_stop:}
	 {
	    \int_zero:N \cexam_number_int
	    \cs_if_exist:NTF\chapter
	    {
	       \cexam_answer_add:p \section{#8}\scan_stop:
	    }
	    {
	       \cexam_answer_add:p \section{#8(答案)}\scan_stop:
	    }
	 }
      }
   }
}
{\c_empty_tl}
\cs_if_exist:NTF\@ssect
{
   \tex_let:D \cexam_ssect:n \@ssect
   \tex_def:D \@ssect#1#2#3#4#5{
      \cexam_ssect:n {#1}{#2}{#3}{#4}{#5}
      \cs_if_exist:NTF\chapter
      {\cexam_answer_add:p \section*{#5}\scan_stop:}
      {\cexam_answer_add:p \section*{#5(答案)}\scan_stop:}
   }
}
{\c_empty_tl}
%    \end{macrocode}
% \end{macro}
%
% 
% \begin{macro}[added=2019-09-18]{answerstd , daan}
% \changes{v3.1.6}{2019/09/18}{新增命令}
% \changes{v3.2.8}{2020/07/14}{末段不必追加一个空行}
% \changes{v3.3.0}{2020/07/27}{优化末段不必追加一行功能}
% \changes{v3.4.0}{2022/04/02}{增加标准答案环境对应的汉语拼音名称daan}
% 答案排版环境,借用填空题排版环境.
% 
% 
%    \begin{macrocode}
\clist_map_inline:nn {answerstd,daan}
{
   \NewDocumentEnvironment {#1}{}
   {
      \parindent=0pt
      \everypar={\everypar_blank:p}
      \cexam_env_add_par:p
   }{}
}
%    \end{macrocode}
% \end{macro}
% 
% 
% \begin{macro}[added=2019-09-18]{\makeanswer}
% \changes{v3.1.6}{2019/09/18}{新增命令}
% \changes{v3.1.7}{2019/09/19}{新打开一页后再排版答案}
% \changes{v3.1.8}{2019/09/22}{追加超链接增加答案前对章节命令的检测}
% \changes{v3.1.9}{2019/10/08}{置零计数器改用标准的\cs{int_zero:N}}
% \changes{v3.1.9}{2019/10/08}{去除答案生成时引用答案文件的错误}
% \changes{v3.1.9}{2019/10/11}{增加目录中参考答案提示}
% \changes{v3.2.2}{2019/02/16}{修复\cs{phantomsection}不引用hypter宏包时错误}
% 答案生成命令
% 
%    \begin{macrocode}
\NewDocumentCommand \makeanswer {}
{
   \bool_if:NTF \answer_student_bool
   {
      \newpage
      \cs_if_exist:NTF \c@chapter
      {\int_zero:N \c@chapter} 
      {
	 \cs_if_exist:NTF \c@section
	 {\int_zero:N \c@section}
	 {\c_empty_tl}
      }
      \cs_if_exist:NTF \phantomsection
      {\phantomsection}
      {\c_empty_tl}
      \cs_if_exist:NTF \chapter
      {\addcontentsline{toc}{chapter}{\protect\Large【参考答案】}{}}
      {\addcontentsline{toc}{section}{\protect\Large【参考答案】}{}}
      \bool_set_false:N \answer_student_bool
      \iow_close:N \answer_write
      \file_if_exist:nTF {\jobname.ans}
      {
	 \cs_if_exist:NTF\theHchapter
	 {\tex_def:D\theHchapter{ans\arabic{chapter}}}
	 {
	    \cs_if_exist:NTF\theHsection
	    {\tex_def:D\theHsection{ans\arabic{section}}}
	    {\c_empty_tl}
	 }
	 \input{\jobname.ans}
      }
      {\c_empty_tl}
   }
   {\c_empty_tl}
}
%    \end{macrocode}
% \end{macro}
%
% \subsection{各题型与答案和解析的自动选择}
%
% \begin{macro}[added=2019-09-03]{\everypar_choice:p}
% \changes{v3.1.3}{2019/09/03}{新增命令}
% \changes{v3.3.4}{2021/02/05}{选择题加入了随机选项排列功能,同时兼容之前的选项输入格式}
% \changes{v3.3.7}{2022/03/26}{配合选择题选项为图片时的排版,增加以s开头时按原始格式排版}
% 
%    \begin{macrocode}
\cs_new:Npn \everypar_choice:p #1.#2\par
{
   \str_if_in:nnTF {#1}{a}
   {
      \bool_if:NTF \answer_student_bool
      {\vspace{-\baselineskip}\par}
      {
	 \dim_set:Nn \cexam_indent_dim {\cexam_ccwd_dim}
	 \dim_set:Nn \cexam_pswd_dim {\linewidth}
	 \dim_sub:Nn \cexam_pswd_dim {\cexam_indent_dim}
	 \tex_parshape:D~1\cexam_indent_dim~\cexam_pswd_dim
	 \str_if_in:nnTF {#2}{*}
	 {
	    \bool_if:NTF \choice_oldopt_bool
	    {
	       \ans_tag_tl
	       \color_group_begin:
	       \color_select:n {red} The~old~version~of~option~don't~supply~this~methord.
	       \color_group_end:
	       \par
	    }
	    {
	       \bool_if:NTF \cho_optstar_bool
	       {\ans_tag_tl\cexam_anspub_tl\par}
	       {\ans_tag_tl#2\par}
	    }
	 }
	 {\ans_tag_tl#2\par}
      }
   }
   {
      \str_if_in:nnTF {#1}{e}
      {\analysis_type:p #1.#2\par}
      {
	 \str_if_in:nnTF {#1}{s}
	 {#2\par}
	 {\choice_type:p #1.#2\par}
      }
   }
   \cexam_answer_iow:p #1.#2\scan_stop:
}
%    \end{macrocode}
% \end{macro}
% 
% \begin{macro}[added=2019-09-03]{\everypar_blank:p}
% \changes{v3.1.3}{2019/09/03}{新增命令}
% \changes{v3.1.7}{2019/09/19}{修复填空题排版答案置零错误}
% \changes{v3.2.2}{2019/02/16}{填空题答案输出改为字符串}
% 
%    \begin{macrocode}
\cs_new:Npn \everypar_blank:p #1.#2\par
{
  \str_if_in:nnTF {#1}{a}
  {
     \bool_if:NTF \answer_student_bool
     {\vspace{-\baselineskip}\par}
     {
	\dim_set:Nn \cexam_indent_dim {\cexam_ccwd_dim}
	\dim_set:Nn \cexam_pswd_dim {\linewidth}
	\dim_sub:Nn \cexam_pswd_dim {\cexam_indent_dim}
	\tex_parshape:D~1\cexam_indent_dim~\cexam_pswd_dim
	\ans_tag_tl\cexam_anspub_tl\par
     }
  }
  {
    \str_if_in:nnTF {#1}{e}
    {\analysis_type:p #1.#2\par}
    {
       \tl_set:Nn \cexam_anspub_tl {}
       \blank_type:p #1.#2\par
    }
  }
  \cexam_answer_iow:p #1.#2\scan_stop: 
}
%    \end{macrocode}
% \end{macro}
% 
% \begin{macro}[added=2019-09-03]{\everypar_judge:p}
% \changes{v3.1.3}{2019/09/03}{新增命令}
% 
%    \begin{macrocode}
\cs_new:Npn \everypar_judge:p #1.#2\par
{
  \str_if_in:nnTF {#1}{a}
  {
     \bool_if:NTF \answer_student_bool
     {\vspace{-\baselineskip}\par}
     {\answer_type:p #1.#2\par}
  }
  {
    \str_if_in:nnTF {#1}{e}
    {\analysis_type:p #1.#2\par}
    {\judge_type:p #1.#2\par}
  }
  \cexam_answer_iow:p #1.#2\scan_stop: 
}
%    \end{macrocode}
% \end{macro}
% 
% \begin{macro}[added=2019-09-03]{\everypar_calculate:p}
% \changes{v3.1.3}{2019/09/03}{新增命令}
% \changes{v3.3.0}{2020/07/27}{优化末段不必追加一行功能}
% 
%    \begin{macrocode}
\cs_new:Npn \everypar_calculate:p #1.#2\par
{
  \str_if_in:nnTF {#1}{a}
  {
     \bool_if:NTF \answer_student_bool
     {\vspace{-\baselineskip}\par}
     {\answer_type:p #1.#2\par}
  }
  {
    \str_if_in:nnTF {#1}{e}
    {\analysis_type:p #1.#2\par}
    {\calculate_type:p #1.#2\par}
  }
  \cexam_answer_iow:p #1.#2\scan_stop: 
}
%    \end{macrocode}
% \end{macro}
% 
% \begin{macro}[added=2019-09-03]{\everypar_proofs:p}
% \changes{v3.1.3}{2019/09/03}{新增命令}
% \changes{v3.2.8}{2020/07/14}{新增证明题命令}
% \changes{v3.2.9}{2020/07/24}{简化证明题命令}
% 
%    \begin{macrocode}
\cs_new:Npn \everypar_proofs:p #1.#2 \par
{
   \str_if_in:nnTF {#1}{>}
   {\bool_set_true:N \ctrl_end_bool}
   {\bool_set_false:N \ctrl_end_bool}
   \str_if_in:nnTF {#1}{pp}
   {\everypar_calculate:p ee.#2\ctrl_end_tl\par}
   {
      \str_if_in:nnTF {#1}{p}
      {\everypar_calculate:p e.#2\ctrl_end_tl\par}
      {\everypar_calculate:p #1.#2 \par}
   }
}
%    \end{macrocode}
% \end{macro}
% 
% \begin{macro}[added=2019-09-03]{\everypar_proof:p}
% \changes{v3.1.3}{2019/09/03}{新增命令}
% 
%    \begin{macrocode}
\cs_new:Npn \everypar_proof:p #1 \par
{
   \everypar_calculate:p e.#1\par
   \tl_clear:N \ana_tag_tl
}
%    \end{macrocode}
% \end{macro}
% 
% 
% \begin{macro}[added=2020-07-24]{\ctrl_end_tl,\cexam_env_end_tl}
% \changes{v3.3.8}{2022/03/28}{删除\cs{cexam_env_add_par:np},定义命令\cs{cexam_env_add_par:p},以方便避免题目忘记最后一行留下空行的bug}
% \changes{v3.3.9}{2022/03/29}{修复改进环境定义后带来的尾行加\cs{par}时的bug}
% \changes{v3.4.0}{2022/04/02}{增加和answerstd对应的daan环境判断}
% 此命令用来在最后一段与\cs{end}环境结尾时如果不外加一行空格和外加一行空格时得到相同的结果,这与一般的环境设置相一致。同时在最后一段中加入一个结束符号,在有需要的时候可以设置这个结束符,比如证明题中给出了设置。 
%
% 由于在v3.3.8 版中设置了兼容环境的统一定义方式,这样做的好处是可以做到升级时只修改一次代码时所有对应环境都做出对应升级。但是这也带来了麻烦,如果使用\cs{clist_map_inline:nn} 来写不必修改原来的\cs{cexam_env_add_par:np},但是这样做必须以 \pkg{\#1}的形式引用各逗号列表中的元素,于是导致定义环境时就不能在环境中使用可选选项,这样子使例题模式不能实现。为了同时保留例题模式,则必须使用变量模式\cs{clist_map_variable:nNn}来编写各环境,但是这又带来了另一个麻烦,就是如果在题目中最后一行忘记添加空行,则自动补充一个\cs{par}的操作不能固定到哪一个环境,这也就是在v3.3.9版中做出的重要升级操作,首先以一个布尔值设定为\pkg{false},然后以\cs{clist_map_variable:nNn}来筛选各环境,当符合题型对应的环境出现时则将布尔值改为\pkg{true},然后在最后就可以根据布尔值来添加\cs{par}了,这样子同时实现了统一定义试题环境,同时又能保留可选参数,但是后果就是如果本宏包增加了新的试题环境,请注意在此处\cs{cexam_env_add_par:p}的 \cs{clist_map_variable:nNn}添加上对应的环境名称才行。2022-03-29
% 
%    \begin{macrocode}
\tl_set:Nn \ctrl_end_tl 
{
   \bool_if:NTF \ctrl_end_bool
   {\cexam_end_tl}
   {\c_empty_tl}
}
\cs_new:Npn \cexam_env_add_par:p #1\end#2 
{
   \bool_set_false:N \cexam_env_add_bool
   \clist_map_variable:nNn 
   {
      choices,xuanze, 
      judgements,panduan, 
      calculations,jisuan,
      proofs,zhengming,
      answerstd,daan
   } 
   \cexam_add_tl
   {
      \tl_if_eq:NnTF \cexam_add_tl {#2}
      {\bool_set_true:N \cexam_env_add_bool}
      {}
   }
   \bool_if:NTF \cexam_env_add_bool
{\tl_put_right:No \cexam_env_end_tl {#1\par\end{#2}}\tl_use:N \cexam_env_end_tl}
   {\tl_put_right:No \cexam_env_end_tl {#1\end{#2}}\cexam_env_add_par:p}
}
%    \end{macrocode}
% \end{macro}
%
% \subsection{用户接口的各题型输入}
%
% \begin{macro}[added=2019-09-03]{\qitem}
% \changes{v3.1.3}{2019/09/03}{新增命令}
% \changes{v3.3.6}{2021/02/25}{增加了可选参数,以配合选项引用}
% 计算题中的若干小问,以\cs{qitem}加入.
%
%    \begin{macrocode}
\NewDocumentCommand \qitem { o } 
{
   \int_add:Nn \cexam_qitem_int {1}
   \int_compare:nNnTF
   {\cexam_qitem_int} = {1}
   {\c_empty_tl}
   {\newline}
   \cexam_qitem:
   \cs_if_exist:NTF \c@chapter
   {
      \IfNoValueTF {#1}
      {
	 \cs_gset:cx 
	 {
	    refitem
	    \int_use:N\c@chapter
	    \int_use:N\c@section
	    \int_use:N\cexam_number_int
	    \int_use:N\cexam_qitem_int :
	 }
	 {(??)}
      }
      {
	 \cs_gset:cx 
	 {
	    refitem
	    \int_use:N\c@chapter
	    \int_use:N\c@section
	    \int_use:N\cexam_number_int
	    #1:
	 }
	 {(\int_use:N \cexam_qitem_int)}
      }
   }
   {
      \IfNoValueTF {#1}
      {
	 \cs_gset:cx 
	 {
	    refitem
	    \int_use:N\c@section
	    \int_use:N\cexam_number_int
	    \int_use:N\cexam_qitem_int :
	 }
	 {(??)}
      }
      {
	 \cs_gset:cx 
	 {
	    refitem
	    \int_use:N\c@section
	    \int_use:N\cexam_number_int
	    #1:
	 }
	 {(\int_use:N \cexam_qitem_int)}
      }
   }
}
%    \end{macrocode}
% \end{macro}
%
% \begin{macro}[added=2019-09-03]{\refitem}
% \changes{v3.3.6}{2021/02/25}{新增命令,以便作者在编写解析时引用小问题号}
%
%    \begin{macrocode}
\NewDocumentCommand \refitem { o }
{
   \cs_if_exist:NTF \c@chapter
   {
      \use:c 
      {
	 refitem
	 \int_use:N\c@chapter
	 \int_use:N\c@section
	 \int_use:N\cexam_number_int
	 #1:
      }
   }
   {
      \use:c 
      {
	 refitem
	 \int_use:N\c@section
	 \int_use:N\cexam_number_int
	 #1:
      }
   }
}
%    \end{macrocode}
% \end{macro}
%
% \begin{macro}[added=2019-09-03]{\source}
% \changes{v3.3.6}{2021/02/25}{新增命令,以便输入题源}
% \changes{v3.3.6}{2021/02/26}{优化命令,加入星级和年份开关}
% 用来输入题源,共有三个参数分别为:星数,时间,源头
%    \begin{macrocode}
\NewDocumentCommand \source {o m m}
{
   \tl_set:Nn \cexam_source_tl{#3}
   \bool_if:nTF {!\source_display_bool && !\source_year_bool}
   {\tl_put_left:No \cexam_source_tl{#2\textperiodcentered}}
   {\c_empty_tl}
   \bool_if:nTF {!\source_display_bool && !\source_star_bool}
   {
      \IfNoValueTF {#1}
      {\c_empty_tl}
      {
	 \tl_put_left:No \cexam_source_tl{\textperiodcentered}
	 \int_zero:N\source_star_int
	 \int_while_do:nNnn {\source_star_int} < {#1}
	 {
	    \tl_put_left:No \cexam_source_tl{\ding{72}}
	    \int_incr:N \source_star_int
	 }
      }
   }
   {\c_empty_tl}
   \bool_if:nTF {!\source_display_bool}
   {
      \tl_put_left:No \cexam_source_tl 
      {
	 \color_group_begin:
	 \exp_args:Nx\color_select:n {\tl_use:N\source_color_tl}
      }
      \tl_put_right:No \cexam_source_tl 
      {\color_group_end:}
      ({\bf\cexam_source_tl})
   }
   {\c_empty_tl}
}
%    \end{macrocode}
% \end{macro}
%
% \begin{macro}[added=2019-09-03]{\blank}
% \changes{v3.1.3}{2019/09/03}{新增命令}
%  填空题中的空白输入方式.
%
%    \begin{macrocode}
\NewDocumentCommand \blank {m}
{\cexam_blank:n{#1}}
%    \end{macrocode}
% \end{macro}
% 
% 
% \begin{macro}[added=2019-09-19]{\change_example:n,\change_normal:n}
% \changes{v3.1.7}{2019/09/19}{新增命令}
% \changes{v3.1.8}{2019/09/22}{增加对章节号的检测,存在才重定义例题标签}
% \changes{v3.1.9}{2019/10/07}{修改\cs{str_if_in:nnTF}为\cs{IfNoValueTF}}
%
% 例题环境中设置例题题号的命令和还原题号命令
%    \begin{macrocode}
\cs_set:Npn \change_example:n #1
{
   \IfNoValueTF {#1}
   {\c_empty_tl}
   {
      \int_gset:Nn \cexam_numold_int{\cexam_number_int}
      \int_gset:Nn \cexam_number_int {\example_number_int}
      \bool_set_false:N \answer_student_bool
      \tl_set:Nn \cexam_number_tag_tl{{\heiti\raisebox{0.5pt}{例}}}
      \cs_if_exist:NTF\c@chapter
      {
	 \tl_set:Nn \cexam_number_tag_i_tl
	 {\int_use:N\c@chapter.\int_use:N\cexam_number_int}
      }
      {
	 \cs_if_exist:NTF\c@section
	 {
	    \tl_set:Nn \cexam_number_tag_i_tl
	    {\int_use:N\c@section.\int_use:N\cexam_number_int}
	 }
	 {\c_empty_tl}
      }
   }
}
\cs_set:Npn \change_normal:n #1
{
   \IfNoValueTF {#1}
   {\c_empty_tl}
   {
      \int_gset:Nn \example_number_int{\cexam_number_int}
      \int_gset:Nn \cexam_number_int {\cexam_numold_int}
   }
}
%    \end{macrocode}
% \end{macro}
% 
% \begin{macro}[added=2019-09-03]{choices,xuanze,blanks,tiankong,judgements,panduan,calculations,jisuan}
% \changes{v3.1.3}{2019/09/03}{新增环境}
% \changes{v3.1.7}{2019/09/03}{增加例题选项模式}
% \changes{v3.2.8}{2020/07/14}{习题环境末段可以不必须加入一个空行}
% \changes{v3.3.0}{2020/07/27}{优化末段加\cs{par}功能}
% \changes{v3.3.8}{2022/03/28}{使用clist来改写各环境定义,可以实现多名称的定义}
% 定义用户输入各题型的环境,其中兼顾了国人的输入习惯,加入了对应的汉语拼音环境.
% 
% 考虑到例题模式的转换,则加入任何一个选项符号,都以例题模式排版.这样做的好处是不同的人有不同的输入习惯,比如可以输入 Exp 等作者认为明显的字符都可以.
%
%    \begin{macrocode}
\clist_map_variable:nNn {choices, xuanze} \cexam_env_tl
{
   \exp_args:Nx \NewDocumentEnvironment {\cexam_env_tl}{o}
   {
      \bool_set_true:N\cexam_choice_bool
      \change_example:n{#1}
      \cexam_answer_add:p \begin{answerstd}\scan_stop:
	 \parindent=0pt
	 \everypar={\everypar_choice:p}
	 \cexam_env_add_par:p
      }
      {
	 \change_normal:n{#1}
      \cexam_answer_add:p \end{answerstd}\scan_stop:
   }
}
\clist_map_variable:nNn {blanks,tiankong} \cexam_env_tl
{
   \NewDocumentEnvironment {\cexam_env_tl}{o}
   {
      \change_example:n{#1}
      \cexam_answer_add:p \begin{answerstd}\scan_stop:
	 \parindent=0pt
	 \everypar={\everypar_blank:p}
	 \cexam_env_add_par:p 
   }
   {
      \change_normal:n{#1}
      \cexam_answer_add:p \end{answerstd}\scan_stop:
   }
}
\clist_map_variable:nNn {judgements,panduan} \cexam_env_tl
{
   \NewDocumentEnvironment {\cexam_env_tl}{o}
   {
      \change_example:n{#1}
      \cexam_answer_add:p \begin{answerstd}\scan_stop:
	 \parindent=0pt
	 \everypar={\everypar_judge:p}
	 \cexam_env_add_par:p 
      }
      {
	 \change_normal:n{#1}
      \cexam_answer_add:p \end{answerstd}\scan_stop:
   }
}
\clist_map_variable:nNn {calculations,jisuan} \cexam_env_tl
{
   \NewDocumentEnvironment {\cexam_env_tl}{o}
   {
      \change_example:n{#1}
      \cexam_answer_add:p \begin{answerstd}\scan_stop:
	 \parindent=0pt
	 \everypar={\everypar_calculate:p}
	 \cexam_env_add_par:p 
      }
      {
	 \change_normal:n{#1}
      \cexam_answer_add:p \end{answerstd}\scan_stop:
   }
}
%    \end{macrocode}
% \end{macro}
% \begin{macro}[added=2020/07/28]{proofs,zhengming}
% \changes{v3.2.8}{2020/07/14}{增加证明题环境}
% \changes{v3.3.0}{2020/07/28}{兼容amsthm.sty的proofs环境,但增强其排版能力符合中文多题目排版格式}
% \changes{v3.3.8}{2022/03/28}{使用clist来改写证明环境定义,可以实现多名称的定义}
%
% 在2020年7月设置了证明题环境,同时考虑到\pkg{cexam} 多题输入的设计,同时又有可能引入\pkg{amsthm},这就涉及到格式兼容问题。在\pkg{cexam} 中使用名称 \pkg{proofs} 和\pkg{zhengming} 这不与 \pkg{amsthm} 冲突,所以这可以单独设计排版模式。同时,如果作者引入\pkg{amsthm} 我将视为作者想使用 \pkg{amsthm}的格式排版证明题,故在以``>'' 开头的控制段落后加入证明结束符号,同时整个证明环境结束后也必然会带入证明结束符号。如果,没引入 \pkg{amsthm} 则视为与\pkg{cexam} 风格一致,则不设置证明结束标志。
% 
% 在\pkg{amsthm} 中定义了\pkg{proof} 环境,在引入\pkg{cexam} 后,我将其视为按\pkg{cexam}风格排版,故设置``证明'' 为黑体,以尽可能兼容\pkg{amsthm} 和 \pkg{cexam} 的排版模式,同时不影响\pkg{proof}的原始定义,这样就不会有错误出现,程序比较稳定。
%
% 输入证明题的输入格式要求以 ``p.'' 取代 ``e.'' , ``pp.''  取代 `` ee.'',因为它不是解析 explain ,而是证明 proof。
% 
%    \begin{macrocode}
\clist_map_variable:nNn {proofs,zhengming} \cexam_env_tl
{
   \NewDocumentEnvironment {\cexam_env_tl}{o}
   {
      \tl_set_eq:NN \ana_tag_tl \prf_tag_tl
      \tl_set_eq:NN \ana_tag_i_tl \prf_tag_i_tl
      \cs_if_exist:NTF \theoremstyle
      {\tl_set:Nn \cexam_end_tl {\hfill$\square$}}
      {\c_empty_tl}
      \change_example:n{#1}
      \cexam_answer_add:p \begin{answerstd}[proofs]\scan_stop:
	 \parindent=0pt
	 \everypar={\everypar_proofs:p}
	 \cexam_env_add_par:p 
      }
      {
	 \cs_if_exist:NTF \theoremstyle
	 { 
	    \everypar={}
	    \bool_if:NTF \ctrl_end_bool
	    {\c_empty_tl}
	    {\cexam_end_tl\par}
	 }
	 {\c_empty_tl}
	 \change_normal:n{#1}
      \cexam_answer_add:p \end{answerstd}\scan_stop:
   }
}
\AtBeginDocument
{
      \cs_if_exist:NTF \theoremstyle
      {\def \proofname {\mbox{\bf 证明}}}
      {\c_empty_tl}
}
%    \end{macrocode}
% \end{macro}
%
% \subsection{派生排版命令}
%
% \begin{macro}[added=2019-09-05]{\letter_sink:nnnnp}
% \changes{v3.1.3}{2019/09/05}{新增首字下沉命令}
% \changes{v3.1.4}{2019/09/10}{精简并重构部分代码}
% \changes{v3.2.4}{2020/03/22}{由于重构了图片模式模块,对其作出修改,参数调整为6个}
% \changes{v3.2.5}{2020/03/27}{恢复原来的5参量结构}
% \changes{v3.2.5}{2020/03/27}{使用\pkg{l3color}重写颜色部分}
% 五个参量:1下沉高度(文本放大高度),2字母与文本间距,3颜色,4首字母,5正文.
% 之前知道有个首字母下沉宏包:  Daniel Flipo 编写的 lettrine 宏包,
% 但是在我写成一系列排版命令后发现这个首字母下沉的格式,在这里可以更加方便的实现. 但是本程序主要是排版各种题型,所以此命令划规到了派生命令,作为附加产品出现在我的宏包中.由于颜色设置使用的是\pkg{l3color}所以此处不再依赖于\pkg{xcolor}宏包,同时也支持三种模式的颜色表达式直接输入颜色。
%
%    \begin{macrocode}
\cs_new:Npn \letter_sink:nnnnp #1#2#3#4#5\par
{
  \dim_set:Nn \cexam_indent_dim {\parindent}
  \dim_set:Nn \parindent {0pt}
  \bool_set_false:N \cexam_fmt_bool 
  \cexam_fmt_pic:nnnn {l}
  {
     \resizebox{!}{#1}{
	\color_group_begin:
	\color_select:n {#3}#4
	\color_group_end:
     }
  }{#2}{0pt}
  \cexam_get_rec:nnnnnn
  {\cexam_picmath_int}
  {\cexam_picht_dim}{\cexam_picwd_dim}
  {#2}{0pt}{#5}
  \cexam_lwr_set:nnnn
  {l}{\cexam_picwd_dim}{#2}{0pt}
  \cexam_shad_set:n {\cexam_picmath_int}
  \cexam_sha_mk:nnn
  {\cexam_picmath_int}
  {\cexam_pslin_dim}{\cexam_pswd_dim}
  \cexam_lwr_set:nnnn
  {}{}{0pt}{0pt}
  \cexam_shad_add:n {\cexam_pslin_dim}
  \cexam_shad_add:n {\cexam_pswd_dim}
  \tex_parshape:D \cexam_shape_tl
  \cexam_picture_tl
  #5\par
  \dim_set:Nn \parindent {\cexam_indent_dim}
}
%    \end{macrocode}
% \end{macro}
%
% \begin{macro}[added=2019-09-05]{\lettersink}
% \changes{v3.1.3}{2019/09/05}{新增首字下沉命令的用户接口命令}
% \changes{v3.1.9}{2019/10/07}{修改\cs{str_if_in:nnTF}为\cs{IfNoValueTF}}
% 四个参量:1文字高度,2首字母与文本间距,3首字母颜色,4.首字母
% 用户接口命令
%
%    \begin{macrocode}
\dim_new:N \letter_ht_dim
\dim_new:N \letter_ltskip_dim
\NewDocumentCommand \lettersink {O{#1} O{#2} O{#3} m}
{
  \IfNoValueTF {#1}
  {\dim_set:Nn \letter_ht_dim{2cm}}
  {\dim_set:Nn \letter_ht_dim{#1}}
  \IfNoValueTF {#2}
  {\dim_set:Nn \letter_ltskip_dim{5pt}}
  {\dim_set:Nn \letter_ltskip_dim{#2}}
  \IfNoValueTF {#3}
  {\letter_sink:nnnnp {\letter_ht_dim}{\letter_ltskip_dim}{black}{#4}}
  {\letter_sink:nnnnp {\letter_ht_dim}{\letter_ltskip_dim}{#3}{#4}}
}
%    \end{macrocode}
% \end{macro}
%
%    \begin{macrocode}
%</package>
%    \end{macrocode}
%
% 
% \section{ctrlwarning.sty代码实现}
% 
%    \begin{macrocode}
%<*ctrlwarning>
%    \end{macrocode}
% 
% \begin{macro}[added=2020-12-30]{ctrlwarning.sty}
% \changes{v3.3.3}{2020/12/29}{增加ctrlwarning.sty(v1.0)宏包}
% \changes{v3.3.4}{2021/02/05}{规范化ctrlwarning.sty选项}
% 此处代码是为了控制编译PDF文件时系统由于字体问题而导致的字体警告,这个问题是因为 ctex宏集修改了字体大小以适应中文排版,但是尚未解决数学公式排版中引用amsmath等宏包时导致的字体警告。所以做为一个省心的方案,初步编写了这个宏包以实现对系统字体警告的控制。 
% 
%    \begin{macrocode}
\bool_new:N \fontwarning_switch_bool
\keys_define:nn {fontwarning / option}
{
   fontwarning  .choice:,
   fontwarning / off .code:n =
   \bool_set_true:N  \fontwarning_switch_bool,
   fontwarning / on .code:n =
   \bool_set_false:N \fontwarning_switch_bool,
   fontwarning / unknown .code:n =
   \bool_set_false:N \fontwarning_switch_bool,
}
\ProcessKeysOptions {fontwarning / option}
\bool_if:NTF \fontwarning_switch_bool
{\def\@font@warning#1{}}
{\c_empty_tl}
%    \end{macrocode}
% \end{macro}
%
%    \begin{macrocode}
%</ctrlwarning>
%    \end{macrocode}
%
% \section{colornote.sty代码实现}
%    \begin{macrocode}
%<*colornote>
%    \end{macrocode}
% \begin{macro}[ added=2023-12-02 ]{colornote.sty}
% \changes{v1.1}{2023/12/02}{增加colornote.sty}
%  此宏包在\LaTeX{}下实现了Hexo的Next主题中的几种note模式,更加方便有特色的书写\LaTeX{}文件
%    \begin{macrocode}
\NewDocumentEnvironment {note}{o}
{
    \noindent
    \IfNoValueTF {#1}
    {
        \begin{tcolorbox}[colback=gray!12!white,colframe=gray!16!white,
            nobeforeafter, boxrule=0.5pt, arc=0mm]
            \color{gray!60!black}
    }
    {
        \str_case:nn {#1}
        {
            {danger}
            {
                \begin{tcolorbox}[colback=red!12!white,
                    colframe=red!16!white, nobeforeafter, boxrule=0.5pt, arc=0mm]
                    \color{red!65!black}
            }
            {warning}
            {
                \begin{tcolorbox}[colback=orange!12!white,
                    colframe=orange!18!white, nobeforeafter, boxrule=0.5pt, arc=0mm]
                    \color{orange!50!black}
            }
            {info}
            {
                \begin{tcolorbox}[colback=cyan!12!white,
                    colframe=cyan!18!white, nobeforeafter, boxrule=0.5pt, arc=0mm]
                    \color{cyan!70!black}
            }
            {success}
            {
                \begin{tcolorbox}[colback=green!12!white,
                    colframe=green!19!white, nobeforeafter, boxrule=0.5pt, arc=0mm]
                    \color{green!40!black}
            }
            {primary}
            {
                \begin{tcolorbox}[colback=violet!12!white,
                    colframe=violet!14!white, nobeforeafter, boxrule=0.5pt, arc=0mm]
                    \color{violet!80!blue}
            }
            {brown}
            {
                \begin{tcolorbox}[colback=brown!9!white,
                    colframe=brown!35!white, nobeforeafter, boxrule=0.5pt, arc=0mm]
                    \color{brown!90!black}
            }
            {default}
            {
                \begin{tcolorbox}[colback=gray!12!white,
                    colframe=gray!16!white,nobeforeafter, boxrule=0.5pt, arc=0mm]
                    \color{gray!60!black}
            }
        }
    }
}{\end{tcolorbox}}
%    \end{macrocode}
% \end{macro}
%
%    \begin{macrocode}
%</colornote>
%    \end{macrocode}
%
% \end{implementation}
%
% \Finale
%
% \endinput
